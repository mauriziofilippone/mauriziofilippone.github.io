% Research Activity

% TODO:
% Fellowships: aggiungere una breve descrizione di quello che hai fatto
% PhD courses: togliere
% Education: Metti le vie


\documentclass[a4paper,10pt]{article}

\addtolength{\oddsidemargin}{-60pt}
\addtolength{\textwidth}{120pt}

\begin{document}

\begin{center}
{\bf \huge Curriculum Vitae}
\end{center}

\section*{Informazioni Personali}
\begin{description}
\item Nome:               {\bf Maurizio Filippone}
\item Nazionalit\`a:        Italiana
\item Posizione Attuale:
  {\bf Lecturer} School of Computing Science, University of Glasgow.
\item Email:              maurizio.filippone@glasgow.ac.uk
\item Pagina Web:         www.dcs.gla.ac.uk/$\sim$maurizio/
\end{description}

\section*{Istruzione e Formazione}
\begin{itemize}
\item Dall'01-01-2005 al 05-05-2008 \\
Titolo conseguito: Dottorato in Informatica \\
Istituto: Dipartimento di Informatica e Scienze dell'Informazione - Universit\`a di Genova \\
Titolo della tesi: Central Clustering in Kernel-Induced Spaces \\
Argomenti trattati nella tesi: metodi kernel per il clustering, spectral clustering, clustering relazionale, fuzzy clustering.

\item Dall'01-09-1998 al 14-07-2004  \\
Laurea in Fisica  \\
Istituto: Dipartimento di Fisica - Universit\`a di Genova \\
Voto: 110/110  \\
Titolo della tesi: Metodi di Ensemble per la previsione di serie storiche \\
Argomenti trattati nella tesi: sistemi non lineari, regressione, ensemble di macchine d'apprendimento, elaborazione di segnali.

\item Dall'01-09-1993 all'01-07-1998  \\
Diploma in Elettronica e Telecomunicazioni \\
Istituto: Scuola superiore I.T.I.S. Italo Calvino - Genova \\
Volto: 60/60

\end{itemize}

\section*{Borse di Studio e Assegni di Ricerca}
\begin{itemize}
\item Dal 01-01-2010 al 31-08-2011 \\
  Department of Statistical Science - University College London (2010 with the Department of Computing Science - University of Glasgow) \\
  1-19 Torrington Place, London, WC1E 7HB - United Kingdom. \\
  Grant: The Synthesis of Probabilistic Prediction \& Mechanistic Modelling within a Computational \& Systems Biology Context \\
%  Topics of the fellowship: novelty detection, statistical testing, Bayesian inference in data modeling. \\
  PI: Prof. Mark Girolami

\item Dal 15-03-2008 al 31-05-2009 \\
  Department of Computer Science - University of Sheffield \\
  Regent Court, 211 Portobello, Sheffield, S1 4DP - United Kingdom \\
  Grant: ALMS: Advanced Lifestyle Monitoring Systems \\
  Argomenti: novelty detection, test statistici, inferenza Bayesiana in data modeling. \\
  PI: Dr Guido Sanguinetti

\item Dall'01-03-2007 al 30-10-2007 \\
  Department of Information and Software Engineering - George Mason University \\
  4400 University Drive, Fairfax, VA 22030 - USA \\
  Grant: Detecting Suspicious Behavior in Reconnaissance Images \\
  Argomenti: outlier detection, stima di densit\`a, clustering relazionale. \\
  PI e co-PI: Prof. Daniel Barbar\`a e Prof. Carlotta Domeniconi

\item Dal 20-07-2006 al 30-10-2006 \\
  presso il Consorzio Venezia Ricerche \\
  Via della Libert\`a 12, 30175 Marghera, Venezia - Italy \\
  Grant: Realizzazione di un sistema di previsione di marea. \\ 
  Argomenti: analisi e previsione di serie temporali, regressione, ensembles di macchine di apprendimento. \\
  Supervisore: Prof. Elio Canestrelli

\item Dall'01-09-2005 al 30-04-2007 \\
  presso il Dipartimento di Informatica e Scienze dell'Informazione - Universit\`a di Genova \\
  Via Dodecaneso 35, 16146 Genova - Italy \\
  Vincitore di una borsa di studio sul tema: \\
  Nuove tecniche di clustering con applicazione all'analisi e segmentazione di immagini. \\
  Supervisore: Dr Stefano Rovetta

\item Dall'01-06-2005 al 31-08-2005 \\
  presso il Dipartimento di Informatica e Scienze dell'Informazione e Dipartimento di Scienze Endocrinologiche e Metaboliche - Universit\`a di Genova \\
  Via Dodecaneso 35, 16146 Genova - Italy \\
  Vincitore di una borsa di studio sul tema: \\
  Applicazione di tecniche innovative di clustering a problemi diagnostici in ambito reumatologico. \\
  Supervisore: Prof. Guido Rovetta

\item Dall'01-09-2004 al 31-09-2004 \\
  presso il Dipartimento di Informatica e Scienze dell'Informazione - Universit\`a di Genova \\
  Via Dodecaneso 35, 16146 Genova - Italy \\
  Progetto:
  Sviluppo di un pacchetto software in linguaggio R e C per l'analisi e la previsione di serie temporali. \\
  Supervisore: Prof. Francesco Masulli

\item Dall'01-12-2003 al 31-12-2003 \\
  presso il Dipartimento di Oncologia, Biologia e Genetica (148) - Universit\`a di Genova \\
  Largo Rosanna Benzi 10, 16132 Genova - Italy \\
  Progetto:
  Sviluppo di una interfaccia in Perl Tk e Java per un simulatore di sistema immunitario in linguaggio C. \\
  Supervisore: Prof. Franco Celada

\end{itemize}

% \section*{Attivit\`a di ricerca}
% Ho iniziato la mia carriera di ricerca nel 2004 al Dipartimento di Informatica e Scienze dell'Informazione presso l'Universit\`a di Genova.
% Nella mia tesi di Laurea in Fisica, sotto la supervisione dei Prof. F.~Masulli e S.~Rovetta, ho effettuato uno studio comparativo di ensemble di macchine di apprendimento in problemi di regressione, con applicazione all'analisi e previsione di serie temporali.
% L'analisi di serie temporali ha coinvolto lo studio di alucune tecniche per la stima del numero di osservazioni passate necessarie per ricostruire fedelmente la dinamica del sistema (model order).
% Uno studio comparativo di queste tecniche \`e stato pubblicato sulla rivista {\em Neural Computing and Applications}.
% Gli ensemble che ho studiato ed implementato sono stati bagging e adaboost utilizzando reti neurali e SVM.
% Ho applicato queste tecniche al problema di previsione del livello di marea nella laguna di Venezia ottenendo un miglioramento nella accuratezza delle previsioni rispetto al metodo attualmente in uso.
% I risultati principali sono stati pubblicati sugli atti della conferenza {\em IJCNN 2007}.

% Quando ho iniziato il dottorato di ricerca, nel 2005, ho iniziato a studiare il problema del clustering.
% I risultati pi\`u importanti della tesi sono stati:
% \begin{itemize}
% \item Survey della letteratura sui metodi spettrali e kernel per il clustering, riportando le connessioni fra i due approcci.
% Questa survey \`e stata pubblicata sulla rivista {\em Pattern Recognition}, ed \`e stata scelta come miglior articolo pubblicato nel 2008 sulla rivista Pattern Recognition.
% \item Introduzione del clustering possibilistico nello spazio indotto da kernel semidefiniti positivi.
% Questo algoritmo si comporta da stimatore non parametrico di densit\`a nello spazio dei dati e mostra una buona robustezza agli outliers.
% Ho studiato le propriet\`a di questo algoritmo in termini di robustezza e stabilit\`a delle soluzioni, mostrando inoltre le connessioni con One Class SVM e Kernel Density Estimation.
% Questi risultati sono stati pubblicati sulla rivista {\em IEEE Transactions on Fuzzy Systems}.
% \item Studio del clustering relazionale quando gli oggetti da raggruppare sono rappresentati in termini di dissimilarit\`a non metriche fra essi.
% Questo studio \`e stato motivato dalla esperienza estera, durante il terzo anno di dottorato, trascorsa all'Information and Software Engineering department della George Mason University.
% Ho lavorato su un progetto finanziato dall'NRO, con i profs. D.~Barbar\`{a}, C.~Domeniconi, and Z.~Duric, sul riconoscimento di comportamenti sospetti in sequenze di immagini e ho dovuto affrontare il problema del clustering nel caso di dissimilarit\`a non metriche.
% Ho riportato gli studi riguardanti le operazioni algebriche necessarie per trasformare le dissimilarit\`a a coppie fra oggetti da non metriche a metriche e ho studiato come molti algoritmi di clustering vengono influenzati da queste operazioni.
% Questi studi mostrano inoltre una connessione diretta fra clustering relazionale e clustering nello spazio indotto da kernel semidefiniti positivi.
% Questa parte della tesi \`e stata pubblicata nella rivista {\em International Journal of Approximate Reasoning}.
% \end{itemize}

% In questi anni mi sono occupato inoltre di altri progetti legati al machine learning: riduzione di dimensionalit\`a e biclustering.
% I risultati di questi lavori sono stati pubblicati nei proceedings di diverse conferenze (serie LNCS), uno di essi \`e stato pubblicato come lettera su {\em Neurocomputing} e un'altro sulla rivista {\em International Journal of Knowledge Engineering and Soft Data Paradigms}.
% Applicazioni tipiche degli approcci di riduzione della dimensionalit\`a riguardano l'analisi di dati genomici, dove il numero di features \`e molto grande rispetto al numero di oggetti.

% Da Marzo 2008 sono stato Research Associate presso l'Universit\`a di Sheffield, nel gruppo di Machine Learning supervisionato dal prof. Guido Sanguinetti.
% La mia area di ricerca si \`e focalizzata su metodi statistici per pattern recognition, comprendendo entrambi i paradigmi frequentisti e bayesiani.
% Siccome il grant che finanziava la mia attivit\`a di ricerca riguardava l'individuazione di cambiamenti dello stile di vita di persone anziane per inferire cambiamenti sul loro stato di salute, gran parte dei miei studi sono stati concentrati su problemi di novelty detection.
% Abbiamo proposto un punto di vista basato sulla teria dell'informazione del problema di individuare novit\`a in insiemi di dati.
% Abbiamo studiato il caso gaussiano mostrando le connessioni con test statistici e evidenziando l'abilit\`a del metodo proposto di controllare in maniera precisa la percentuale di falsi positivi su dati di test anche quando il numero di dati di addestramento \`e piccolo.
% Forti di questo risultato, abbiamo esteso l'approccio al caso della mistura di gaussiane e delle serie temporali autoregressive.
% I risultati riguardanti il caso gaussiano e della mistura di gaussiane sono stati pubblicati sulla rivista {\em Pattern Recognition}, mentre l'estensione al caso di modelli autoregressivi appare sulla rivista {\em IEEE Transactions on Signal Processing}.

\newpage

\subsection*{Premi Internazionali}
\begin{itemize}
     \item Miglior articolo pubblicato nel 2008 sulla rivista Pattern Recognition:
       \\M. Filippone, F. Camastra, F. Masulli, and S. Rovetta.
       \textbf{A survey of kernel and spectral methods for clustering}.
       \emph{Pattern Recognition}, 41(1):176-190, January 2008.
       \\Manuscripts published in volume 41 (year 2008) have been judged by the Editors-in-Chief and the members of the Editorial and Advisory Boards of the journal based on the following criteria: originality of the contribution, presentation and exposition of the manuscript, and citations by other researchers.
\end{itemize}

\subsection*{Pubblicazioni}
\subsubsection*{Riviste}
\begin{itemize}
     \item  M. Filippone, G. Sanguinetti.
       \\\textbf{Approximate inference of the bandwidth in multivariate kernel density estimation}.
       \\\emph{Computational Statistics \& Data Analysis}, 55(12):3104-3122, 2011.
     \item  M. Filippone, G. Sanguinetti.
       \\\textbf{A Perturbative Approach to Novelty Detection in Autoregressive Models}.
       \\\emph{IEEE Transactions on Signal Processing}, 59(3):1027-1036, March 2011.
     \item L. Mohamed, B. Calderhead, M. Filippone, M. Christie, and M. Girolami.
       \\\textbf{Population MCMC methods for history matching and uncertainty quantification}.
       \\\emph{Computational Geosciences}. to appear.
     \item  M. Filippone, F. Masulli, and S. Rovetta.
       \\\textbf{Applying the possibilistic c-means algorithm in kernel-induced spaces}.
       \\\emph{IEEE Transactions on Fuzzy Systems}, 18(3):572-584, June 2010.
     \item  M. Filippone, F. Masulli, and S. Rovetta.
       \\\textbf{Simulated annealing for supervised gene selection}.
       \\\emph{Soft Computing}. in press.
     \item  M. Filippone, G. Sanguinetti.
       \\\textbf{Information theoretic novelty detection}.
       \\\emph{Pattern Recognition}, 43(3):805-814, March 2010.
     \item  F. Camastra and M. Filippone.
       \\\textbf{A comparative evaluation of nonlinear dynamics methods for time series prediction}.
       \\\emph{Neural Computing and Applications}, 18(8):1021-1029, November 2009.
     \item  M. Filippone, F. Masulli, and S. Rovetta.
       \\\textbf{Clustering in the membership embedding space}.
       \\\emph{International Journal of Knowledge Engineering and Soft Data Paradigms}, 4(1):363-375, 2009.
     \item  S. Rovetta, F. Masulli, M. Filippone.
       \\\textbf{Soft ranking in clustering}.
       \\\emph{Neurocomputing}, 72(7-9):2028-2031, March 2009.
     \item  M. Filippone.
       \\\textbf{Dealing with non-metric dissimilarities in fuzzy central clustering algorithms}.
       \\\emph{International Journal of Approximate Reasoning}, 50(2):363-384, February 2009.
     \item  M. Filippone, F. Camastra, F. Masulli, and S. Rovetta.
       \\\textbf{A survey of kernel and spectral methods for clustering}.
       \\\emph{Pattern Recognition}, 41(1):176-190, January 2008.
       \\{\footnotesize This paper has been chosen to be the best paper published in 2008 in the journal Pattern Recognition}
\end{itemize}

\subsubsection*{Discussioni}
\begin{itemize}
     \item M. Filippone, A. Mira, M. Girolami.
       \\Discussion of the paper \textbf{Sampling Schemes for Generalized Linear Dirichlet Process Random Effects Models} by M. Kyung, J. Gill, and G. Casella.
       \emph{Statistical Methods \& Applications} to appear.
     \item M. Filippone.
       \\Discussion of the paper \textbf{Riemann manifold Langevin and Hamiltonian Monte Carlo methods} by M. Girolami and B. Calderhead. 
       \emph{Journal of the Royal Statistical Society, Series B (Statistical Methodology)}, 73(2):123-214, 2011.
     \item V. Stathopoulos and M. Filippone.
       \\Discussion of the paper \textbf{Riemann manifold Langevin and Hamiltonian Monte Carlo methods} by M. Girolami and B. Calderhead. 
       \emph{Journal of the Royal Statistical Society, Series B (Statistical Methodology)}, 73(2):123-214, 2011.
\end{itemize}

\subsubsection*{Conferenze}
\begin{itemize}
     \item  D. Barbar\'{a}, C. Domeniconi, Z. Duric, M. Filippone, E. Lawson, and R. Mansfield.
       \\\textbf{Detecting suspicious behavior in surveillance images}.
       \\\emph{In ICDM Workshops}, pages 891-900. IEEE Computer Society, 2008. 
     \item  F. Camastra and M. Filippone.
       \\\textbf{SVM-based time series prediction with nonlinear dynamics methods}.
       \\In Bruno Apolloni, Robert~J. Howlett, and Lakhmi~C. Jain, editors,
       \\\emph{KES (3)}, volume 4694 of \emph{Lecture Notes in Computer Science}, pages 300-307. Springer, 2007.
     \item  S. Rovetta, F. Masulli, and M. Filippone.
       \\\textbf{Membership embedding space approach and spectral clustering}.
       \\In Bruno Apolloni, Robert~J. Howlett, and Lakhmi~C. Jain, editors,
       \\\emph{KES (3)}, volume 4694 of \emph{Lecture Notes in Computer Science}, pages 901-908. Springer, 2007.
     \item  E. Canestrelli, P. Canestrelli, M. Corazza, M. Filippone, S. Giove, and F. Masulli.
       \\\textbf{Local learning of tide level time series using a fuzzy approach}.
       \\In \emph{IJCNN - International Joint Conference on Neural Networks}, Orlando - Florida, 12-17 August 2007.
     \item  M. Filippone, F. Masulli, and S. Rovetta.
       \\\textbf{Possibilistic clustering in feature space}.
       \\In Francesco Masulli, Sushmita Mitra, and Gabriella Pasi, editors,
       \\\emph{WILF}, volume 4578 of \emph{Lecture Notes in Computer Science}, pages 219-226. Springer, 2007.
     \item  M. Filippone, F. Masulli, S. Rovetta, S. Mitra, and H. Banka.
       \\\textbf{Possibilistic approach to biclustering: An application to oligonucleotide microarray data analysis}.
       \\In Corrado Priami, editor, \emph{Computational Methods in Systems Biology}, 
       \\volume 4210 of \emph{Lecture Notes in Computer Science}, pages 312-322. Springer Berlin / Heidelberg, 2006.
%     \item  M. Filippone, F. Masulli, S. Rovetta, and S.-P. Constantinescu.
%       \\\textbf{Input selection with mixed data sets: A simulated annealing wrapper approach}.
%       \\In \emph{CISI 06 - Conferenza Italiana Sistemi Intelligenti},
%       \\Ancona - Italy, 27-29 September 2006.
     \item  M. Filippone, F. Masulli, and S. Rovetta.
       \\\textbf{Gene expression data analysis in the membership embedding space: A constructive approach}.
       \\In \emph{CIBB 2006 - Third International Meeting on Computational 
       Intelligence Methods for Bioinformatics and Biostatistics}, 
       \\Genova - Italy, 29-31 August 2006.
     \item  M. Filippone, F. Masulli, and S. Rovetta.
       \\\textbf{Supervised classification and gene selection using simulated annealing}.
       \\In \emph{IJCNN - International Joint Conference on Neural Networks},
       \\Vancouver - Canada, 16-21 July 2006.
     \item  M. Filippone, F. Masulli, and S. Rovetta.
       \\\textbf{Unsupervised gene selection and clustering using simulated annealing}.
       \\In Isabelle Bloch, Alfredo Petrosino, and Andrea Tettamanzi, editors,
       \\\emph{WILF}, volume 3849 of \emph{Lecture Notes in Computer Science}, pages 229-235. Springer, 2005.
     \item  F. Masulli, S. Rovetta, and M. Filippone.
       \\\textbf{Clustering genomic data in the membership embedding space}.
       \\In \emph{CI-BIO - Workshop on Computational Intelligence Approaches for the Analysis of Bioinformatics Data},
       \\Montreal - Canada, 5 August 2005.
     \item  S. Rovetta, F. Masulli, and M. Filippone.
       \\\textbf{Soft rank clustering}.
       \\In Bruno Apolloni, Maria Marinaro, Giuseppe Nicosia, and Roberto Tagliaferri, editors, 
       \\\emph{WIRN/NAIS}, volume 3931 of \emph{Lecture Notes in Computer Science}, pages 207-213. Springer, 2005.
     \item  M. Filippone, F. Masulli, and S. Rovetta.
       \\\textbf{ERAF: a R package for regression and forecasting}.
       \\In Biological and Artificial Intelligence Environments, pages 165-173, Secaucus, NJ, USA, 2005.
       \\Springer-Verlag New York, Inc.
\end{itemize}

\subsubsection*{Rapporti Tecnici}
\begin{itemize}
     \item  M. Filippone and G. Sanguinetti.
       \\\textbf{Novelty detection in autoregressive models using information theoretic measures}.
       \\Technical Report CS-09-06, Department of Computer Science, University of Sheffield, July 2009.
     \item  M. Filippone and G. Sanguinetti.
       \\\textbf{Information theoretic novelty detection}.
       \\Technical Report CS-09-02, Department of Computer Science, University of Sheffield, February 2009.
     \item  M. Filippone.
       \\\textbf{Fuzzy clustering of patterns represented by pairwise dissimilarities}.
       \\Technical Report ISE-TR-07-05, Department of Information and Software Engineering, George Mason University, October 2007.
     \item  M. Filippone, F. Camastra, F. Masulli, and S. Rovetta.
       \\\textbf{A survey of kernel and spectral methods for clustering}.
       \\Technical Report DISI-TR-06-19,
       Department of Computer and Information Sciences at the Universit\`a di Genova, Italy, 18th October 2006.
     \item  M. Filippone, F. Masulli, and S. Rovetta.
       \\\textbf{A wrapper approach to supervised input selection using simulated annealing}.
       \\Technical Report DISI-TR-06-10,
       Department of Computer and Information Sciences at the Universit\`a di Genova, Italy, 12th June 2006.
\end{itemize}

\subsubsection*{Tesi}
\begin{itemize}
      \item M. Filippone.
	\\\textbf{Central Clustering in Kernel-Induced Spaces}.
	\\Phd Thesis in Computer Science, University of Genova, February 2008.
      \item M. Filippone.
	\\\textbf{Metodi di ensemble per la previsione di serie storiche}.
	\\Master's Degree thesis in physics, University of Genova, July 2004.
\end{itemize}

\subsubsection*{Periodici}
\begin{itemize}
      \item C. Calcagno, M. Filippone, and D. Ghersi.
	\\R: Potenza e versatilit\`a per l'analisi statistica.
	\\Linux \& C., (53):41-46, June 2006.
\end{itemize}

\subsection*{Attivit\`a in qualit\`a di Referee}
\begin{itemize}
\item Riviste:
  Pattern Recognition (19),
  IEEE Transactions on Neural Networks (11), 
  Pattern Recognition Letters (9),
  IEEE Transactions on Signal Processing (4), 
  IEEE Signal Processing Letters (2), 
  IEEE Transactions on Pattern Analysis and Machine Intelligence (1), 
  Computational Intelligence (1),
  Neural Processing Letters (1),
  IEEE Transactions on Parallel and Distributed Systems (1),
  Soft Computing (1),
  EURASIP Journal on Advances in Signal Processing (1),
  Information Sciences (1).
%  \begin{itemize}
%  \item Pattern Recognition (1)
%  \item Pattern Recognition Letters (1)
%  \item IEEE Transactions on Parallel and Distributed Systems (1)
%  \item Soft Computing (1)
%  \item EURASIP Journal on Advances in Signal Processing (1)
%  \item Information Sciences (1)
%  \end{itemize}
\item Conferenze:
  PRIB 2010 (2), 
  IJCNN 2010 (2), 
  PRIB 2009 (4), 
  CIBB 2009 (4), 
  IFSA 2009 (4), 
  IJCNN 2009 (8), 
  CIBB 2008 (4), 
  WCCI 2008 (3), 
  ICDM 2007 (2), 
  IJCNN 2007 (4), 
  CIBB 2007 (1), 
  IJCNN 2006 (5), 
  CIBB 2006 (1).
%  \begin{itemize}
%  \item CIBB 2008 (4)
%    \\Fifth International Meeting on Computational Intelligence Methods for Bioinformatics and Biostatistics, 3-4 October, 2008, Vietri sul Mare, Salerno - Italy
%  \item WCCI 2008 (3)
%    \\IEEE World Congress on Computational Intelligence, 1-6 June 2008, Hong Kong - China
%  \item ICDM 2007 (2)
%    \\IEEE International Conference on Data Mining, 28-31 October 2007, Omaha, NE - USA
%  \item IJCNN 2007 (4)
%    \\International Joint Conferences on Neural Networks, 12-17 August 2007, Orlando, FL - USA
%  \item CIBB 2007 (1)
%    \\Fourth International Meeting on Computational Intelligence Methods for Bioinformatics and Biostatistics, 7-10 July 2007, Portofino, Genova - Italy
%  \item IJCNN 2006 (5)
%    \\IEEE World Congress on Computational Intelligence, 16-21 July 2006, Vancouver - Canada
%  \item CIBB 2006 (1)
%    \\Third International Meeting on Computational Intelligence Methods for Bioinformatics and Biostatistics, 29-31 August 2006, Genova - Italy
%  \end{itemize}
\end{itemize}

\subsection*{Partecipazione a Conferenze}
\begin{itemize}
\item 09 June 2011, Bologna, Italy \\
  Convegno intermedio SIS 2011.
  \\Presentazione orale: \emph{Bayesian inference in latent variable models and applications}.
\item 10-11 Dicembre 2010, Whistler, BC, Canada \\
  NIPS 2010 Workshops: Neural Information Processing Systems Conference.
  \\Presentazione del Poster: \emph{Posterior Inference in Latent Gaussian Models Using Manifold MCMC Methods}.
\item 23-26 Agosto 2010, Istanbul, Turkey \\
  ICPR 2010 - 20th International Conference on Pattern Recognition
  \\\emph{Ho ricevuto il premio per il miglior articolo pubblicato nel 2008 sulla rivista Pattern Recognition}
\item 12-14 Luglio 2010, Sheffield, United Kingdom \\
  UCM 2010 - Uncertainty in Computer Models 2010 conference
\item 5-9 Luglio 2010, Glasgow, United Kingdom \\
  IWSM 2010 - 25th International Workshop on Statistical Modelling
\item 3-8 Giugno 2010, Benidorm, Spain \\
  Ninth Valencia International Meeting on Bayesian Statistics - 2010 World Meeting of the International Society for Bayesian Analysis.
  \\Presentazione del Poster: \emph{Inference for Gaussian Process Emulation of Oil Reservoir Simulation Codes}.
\item 6-7 Aprile 2010, Warwick, United Kingdom \\
  WOGAS2 - Workshop on Geometric and Algebraic Statistics 2.
\item 3-5 Marzo 2010, Edinburgh, United Kingdom \\
  Mixture estimation and applications.
  \\Presentazione del Poster: \emph{Information Theoretic Novelty Detection for Mixtures of Gaussians}.
\item 7-9 Settembre 2009, Sheffield, United Kingdom \\
  PRIB 2009: Pattern Recognition in Bioinformatics 2009.
\item 20-21 Maggio 2009, Swansea, United Kingdom \\
  NCAF Meeting: Neural Computing and Applications. Special Theme - Grand Challenges in Information-Driven Healthcare.
\item 15-19 Dicembre 2008, Pisa, Italy \\
  ICDM 2008: IEEE International Conference on Data Mining.
  \\Presentazione orale: \emph{Detecting Suspicious Behavior in Surveillance Images}.
\item 8-13 Dicembre 2008, Vancouver, BC, Canada \\
  NIPS 2008: Neural Information Processing Systems Conference.
\item 9-10 Settembre 2008, Sheffield, United Kingdom \\
  NCAF Meeting: Neural Computing and Applications. Special Theme - Dynamics and Dynamic Systems.
  \\Presentazione orale: \emph{Information Theoretic Novelty Detection}.
\item 18-19 Giugno 2008, Oxford, United Kingdom \\
  NCAF Meeting: Neural Computing and Applications. Special Theme - Signal Processing and Biomedical Applications.
\item 31 Marzo - 2 Aprile 2008, Sheffield, United Kingdom \\
  Data Modelling Workshops \& Symposium.
\item 13-17 Agosto 2007, Orlando, FL - USA \\
  IJCNN 2007 - International Joint Conferences on Neural Networks.
  \\Presentazione del Poster: \emph{Local learning of tide level time series using a fuzzy approach}.
\item 17 Agosto 2007, Orlando, FL - USA \\
  CI-BIO 2007 - Post-Conference Workshop on Computational Intelligence Approaches for the Analysis of Bioinformatics Data.
  \\Presentazione orale dell'articolo: \emph{Aggregating Memberships in Possibilistic Biclustering}.
\item 27-29 Settembre 2006, Ancona - Italy \\
  CISI 06 Conferenza Italiana Sistemi Intelligenti.
  \\Presentazione del Poster: \emph{Input Selection with Mixed Data Sets: A Simulated Annealing Wrapper Approach}.
\item 15-16 Settembre 2006, Genova - Italy \\
  BioLeMiD 06 - Third Bioinformatics Meeting on Machine Learning for Microarray Studies of Disease: biomarker selection.
\item 29-31 Agosto 2006, Genova - Italy
  FLINS 2006 - 7th International FLINS Conference on Applied Artificial Intelligence.
\item 28 Giugno 2006, Genova - Italy \\
  Workshop on Trends in Computational Sciences.
\item 22 Giugno 2006, Genova - Italy \\
  The I LIMBS day - A free one-day workshop about intelligence.
\item 21 Giugno 2006, Genova - Italy \\
  Second workshop on Analytic Methods for Learning Theory: Learning, Regularization and Approximation
\item 15-17 Settembre 2005, Crema - Italy \\
  WILF 05 - International Workshop on Fuzzy Logic and Applications.
  \\Presentazione orale dell'articolo: \emph{Unsupervised gene selection and clustering using simulated annealing}.
\item 16-17 Giugno 2005, Genova - Italy \\
  CLIP 2005 - Workshop on Cross-language information processing.
  \\Presentazione orale dell'articolo: \emph{Soft rank clustering}.
\item 8-11 Giugno 2005, Vietri sul Mare - Italy \\
  WIRN 05 - XVI Italian Workshop on Neural Networks.
  \\Presentazione orale dell'articolo: \emph{Soft rank clustering}.
\item 14-17 Settembre 2004, Perugia - Italy \\
  WIRN 04 - XV Italian Workshop on Neural Networks.
  \\Presentazione del Poster: \emph{ERAF: a R package for regression and forecasting}.
\end{itemize}

\subsection*{Presentazioni}
\begin{itemize}
\item 08 Febbraio, 2011, University College London (CSML seminars series) - \emph{Calibration of Oil Reservoir Simulation Codes}.
\item 24 Gennaio, 2011, University College London - \emph{Classification of fMRI data using latent Gaussian models}.
\item 12 Novembre, 2010, University of Glasgow - \emph{Information Theoretic Novelty Detection}.
\item 13 Ottobre, 2010, Royal Statistical Society - \emph{Discussion of the paper ``Riemann manifold Langevin and Hamiltonian Monte Carlo methods'' by M. Girolami and B. Calderhead}.
\item 26 Marzo, 2010, Liverpool John Moores University - \emph{Information Theoretic Novelty Detection}.
\item 11 Novembre, 2009, University of Edinburgh - \emph{Information Theoretic Novelty Detection}.
\item 21 Ottobre, 2009, University of Sheffield - Tutorial per il gruppo Speech and Hearing - \emph{The probabilistic approach in data modeling}.
\item 14 Luglio, 2009, Columbia University - \emph{Information Theoretic Novelty Detection}.
\item 21 Gennaio, 2009, University of Glasgow - \emph{Information Theoretic Novelty Detection}.
\item 22 Dicembre, 2008, Universit\`a di Genova - \emph{Information Theoretic Novelty Detection}.
\item 3 Marzo, 2008, Universit\`a degli Studi di Milano - \emph{Central Clustering in Kernel-Induced Spaces}.
\item 27 Febbraio, 2008, Universit\`a degli Studi di Napoli Parthenope - \emph{Central Clustering in Kernel-Induced Spaces}.
\item 27 Settembre, 2007, George Mason University - \emph{Kernel and Spectral Methods for Clustering}.
\item 15 Novembre 2006, Universit\`a di Genova - \emph{Kernel and Spectral Methods for Clustering}.
\item 21 Marzo 2006, Universit\`a di Genova - \emph{Spectral Approach to Clustering}.
\item 20 Dicembre 2005, Universit\`a di Genova - \emph{Subsequence Matching for Time Series Forecasting}.
\end{itemize}



\section*{Attivit\`a di Insegnamento}

\subsection*{Commissioni d'esame}
\begin{itemize}
\item 2005 e 2006: \\
  Reti Neurali e Soft Computing - Universit\`a di Genova, \\ Sistemi Operativi - Universit\`a di Pisa
\end{itemize}
%\begin{itemize}
%\item 2006:
%  \begin{itemize}
%  \item Reti Neurali \\
%    IV / V anno, Corso di Laurea in Informatica - Universit\`a di Genova \\ 
%    Prof. Stefano Rovetta
%  \item Soft Computing \\
%    III anno, Corso di Laurea in Informatica - Universit\`a di Genova \\
%    Prof. Francesco Masulli
%  \item Sistemi Operativi \\
%    II anno, Corso di Laurea in Informatica Applicata - Polo di La Spezia - Universit\`a di Pisa \\
%    Prof. Francesco Masulli
%  \end{itemize}
%\item 2005:
%  \begin{itemize}
%  \item Reti Neurali \\
%    IV / V anno, Corso di Laurea in Informatica - Universit\`a di Genova \\
%    Prof. Francesco Masulli
%  \item Soft Computing \\
%    III anno, Corso di Laurea in Informatica - Universit\`a di Genova \\
%    Prof. Francesco Masulli
%  \item Sistemi Operativi \\
%    II anno, Corso di Laurea in Informatica Applicata - Polo di La Spezia - Universit\`a di Pisa \\
%    Prof. Francesco Masulli
%  \end{itemize}
%\end{itemize}

\subsection*{Insegnamento}
\begin{itemize}
\item 10-2008 \\
  Assistente di Laboratorio (2 ore)
  Bioinformatics \\
  Modulo di Computational Biology module per MSC in Biological and Bioprocess Engineering \\
  Prof. Guido Sanguinetti \\
\item 09-2005 - 12-2005 \\
  Assistente di Laboratorio (30 ore)
  Reti Neurali \\
  IV / V anno, Corso di Laurea in Informatica - Universit\`a di Genova \\
  Prof. Stefano Rovetta \\
\item 09-2005 - 12-2005 \\
  Assistente di Laboratorio (10 ore)
  Soft Computing \\
  III anno, Corso di Laurea in Informatica - Universit\`a di Genova \\
  Prof. Francesco Masulli \\
\item 09-2004 - 12-2004 \\
  Assistente di Laboratorio (10 ore) \\
  Presentazione di un mini-corso sul linguaggio R (2 ore)
  Reti Neurali \\
  IV / V anno, Corso di Laurea in Informatica - Universit\`a di Genova \\
  Prof. Francesco Masulli \\
\end{itemize}

\subsection*{Altre Attivit\`a di Insegnamento}
\begin{itemize}
\item 09-2005 - 06-2006 \\
  Insegnante di Matematica (120 ore) \\
  Insegnante di Informatica (80 ore) \\  
  IAL LIGURIA - Scuola Professionale - Programma Operatore Commerciale \\
  email: informazioni@ial.liguria.it, segreteria@ial.liguria.it
\end{itemize}

%\newpage




\section*{Capacit\`a e Competenze Personali}

\subsection*{Lingue}
\begin{itemize}
\item Madrelingua: Italiano
\item Inglese:
  \begin{itemize}
%    \item Reading ability: good
%    \item Writing ability: good
%    \item Speaking ability: good
    \item Dal 05-2007 al 10-2007 ho frequentato alcuni corsi di grammatica, conversazione e pronuncia per non nativi americani (60 ore)
      presso la George Mason University, Fairfax, VA 22030 - USA
    \item Dal 20-07-2004 al 05-08-2004 ho frequentato un corso di inglese di livello intermedio (30 ore)
      presso la Byron School, 79 Hills Road - CB2 1PG Cambridge 
  \end{itemize}
\end{itemize}

\subsection*{Capacit\`a e Competenze Tecniche}
\begin{description}
\item[Sistemi operativi:] Windows, Unix e Linux.
\item[Linguaggi di programmazione:] R, C, C++, Fortran e Assembler.
\item[Linguaggi di scripting:]  Perl.
\item[Linguaggi per il web:] HTML e PHP.
\item[Basi di dati:] SQL.
\item[Linguaggi di editing:] Latex e Word.
\end{description}

\end{document}
