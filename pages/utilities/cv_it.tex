\documentclass[a4paper,10pt]{article}

\addtolength{\oddsidemargin}{-60pt}
\addtolength{\textwidth}{120pt}

\begin{document}

\begin{center}
{\bf \huge Curriculum Vitae}
\end{center}

\section*{Informazioni Personali}
\begin{description}
\item Nome:               {\bf Maurizio Filippone}
\item Nazionalit\`a:        Italiana
\item Posizione Attuale:
  {\bf Lecturer} - School of Computing Science, University of Glasgow
\item Email:              maurizio.filippone@glasgow.ac.uk
\item Pagina Web:         www.dcs.gla.ac.uk/$\sim$maurizio/
\end{description}

\section*{Istruzione e Formazione}
\begin{itemize}
\item Dall'01-01-2005 al 05-05-2008 \\
Titolo conseguito: Dottorato in Informatica \\
Istituto: Dipartimento di Informatica e Scienze dell'Informazione - Universit\`a di Genova \\
Titolo della tesi: Central Clustering in Kernel-Induced Spaces \\
Argomenti trattati nella tesi: metodi kernel per il clustering, spectral clustering, clustering relazionale, fuzzy clustering.

\item Dall'01-09-1998 al 14-07-2004  \\
Laurea in Fisica  \\
Istituto: Dipartimento di Fisica - Universit\`a di Genova \\
Voto: 110/110  \\
Titolo della tesi: Metodi di Ensemble per la previsione di serie storiche \\
Argomenti trattati nella tesi: sistemi non lineari, regressione, ensemble di macchine d'apprendimento, elaborazione di segnali.

\item Dall'01-09-1993 all'01-07-1998  \\
Diploma in Elettronica e Telecomunicazioni \\
Istituto: Scuola superiore I.T.I.S. Italo Calvino - Genova \\
Volto: 60/60

\end{itemize}

\section*{Borse di Studio e Assegni di Ricerca}
\begin{itemize}
\item Dal 01-01-2010 al 31-08-2011 \\
  Department of Statistical Science - University College London (2010 with the Department of Computing Science - University of Glasgow) \\
  1-19 Torrington Place, London, WC1E 7HB - United Kingdom. \\
  Grant: The Synthesis of Probabilistic Prediction \& Mechanistic Modelling within a Computational \& Systems Biology Context \\
%  Topics of the fellowship: novelty detection, statistical testing, Bayesian inference in data modeling. \\
  PI: Prof. Mark Girolami

\item Dal 15-03-2008 al 31-05-2009 \\
  Department of Computer Science - University of Sheffield \\
  Regent Court, 211 Portobello, Sheffield, S1 4DP - United Kingdom \\
  Grant: ALMS: Advanced Lifestyle Monitoring Systems \\
  Argomenti: novelty detection, test statistici, inferenza Bayesiana in data modeling. \\
  PI: Dr Guido Sanguinetti

\item Dall'01-03-2007 al 30-10-2007 \\
  Department of Information and Software Engineering - George Mason University \\
  4400 University Drive, Fairfax, VA 22030 - USA \\
  Grant: Detecting Suspicious Behavior in Reconnaissance Images \\
  Argomenti: outlier detection, stima di densit\`a, clustering relazionale. \\
  PI e co-PI: Prof. Daniel Barbar\`a e Prof. Carlotta Domeniconi

\item Dal 20-07-2006 al 30-10-2006 \\
  presso il Consorzio Venezia Ricerche \\
  Via della Libert\`a 12, 30175 Marghera, Venezia - Italy \\
  Grant: Realizzazione di un sistema di previsione di marea. \\ 
  Argomenti: analisi e previsione di serie temporali, regressione, ensembles di macchine di apprendimento. \\
  Supervisore: Prof. Elio Canestrelli

\item Dall'01-09-2005 al 30-04-2007 \\
  presso il Dipartimento di Informatica e Scienze dell'Informazione - Universit\`a di Genova \\
  Via Dodecaneso 35, 16146 Genova - Italy \\
  Vincitore di una borsa di studio sul tema: \\
  Nuove tecniche di clustering con applicazione all'analisi e segmentazione di immagini. \\
  Supervisore: Dr Stefano Rovetta

\item Dall'01-06-2005 al 31-08-2005 \\
  presso il Dipartimento di Informatica e Scienze dell'Informazione e Dipartimento di Scienze Endocrinologiche e Metaboliche - Universit\`a di Genova \\
  Via Dodecaneso 35, 16146 Genova - Italy \\
  Vincitore di una borsa di studio sul tema: \\
  Applicazione di tecniche innovative di clustering a problemi diagnostici in ambito reumatologico. \\
  Supervisore: Prof. Guido Rovetta

\item Dall'01-09-2004 al 31-09-2004 \\
  presso il Dipartimento di Informatica e Scienze dell'Informazione - Universit\`a di Genova \\
  Via Dodecaneso 35, 16146 Genova - Italy \\
  Progetto:
  Sviluppo di un pacchetto software in linguaggio R e C per l'analisi e la previsione di serie temporali. \\
  Supervisore: Prof. Francesco Masulli

\item Dall'01-12-2003 al 31-12-2003 \\
  presso il Dipartimento di Oncologia, Biologia e Genetica (148) - Universit\`a di Genova \\
  Largo Rosanna Benzi 10, 16132 Genova - Italy \\
  Progetto:
  Sviluppo di una interfaccia in Perl Tk e Java per un simulatore di sistema immunitario in linguaggio C. \\
  Supervisore: Prof. Franco Celada

\end{itemize}

\section*{Attivit\`a di ricerca}
% Ho iniziato la mia carriera di ricerca nel 2004 al Dipartimento di Informatica e Scienze dell'Informazione presso l'Universit\`a di Genova.
% Nella mia tesi di Laurea in Fisica, sotto la supervisione dei Prof. F.~Masulli e S.~Rovetta, ho effettuato uno studio comparativo di ensemble di macchine di apprendimento in problemi di regressione, con applicazione all'analisi e previsione di serie temporali.
% L'analisi di serie temporali ha coinvolto lo studio di alucune tecniche per la stima del numero di osservazioni passate necessarie per ricostruire fedelmente la dinamica del sistema (model order).
% Uno studio comparativo di queste tecniche \`e stato pubblicato sulla rivista {\em Neural Computing and Applications}.
% Gli ensemble che ho studiato ed implementato sono stati bagging e adaboost utilizzando reti neurali e SVM.
% Ho applicato queste tecniche al problema di previsione del livello di marea nella laguna di Venezia ottenendo un miglioramento nella accuratezza delle previsioni rispetto al metodo attualmente in uso.
% I risultati principali sono stati pubblicati sugli atti della conferenza {\em IJCNN 2007}.

% Quando ho iniziato il dottorato di ricerca, nel 2005, ho iniziato a studiare il problema del clustering.
% I risultati pi\`u importanti della tesi sono stati:
% \begin{itemize}
% \item Survey della letteratura sui metodi spettrali e kernel per il clustering, riportando le connessioni fra i due approcci.
% Questa survey \`e stata pubblicata sulla rivista {\em Pattern Recognition}, ed \`e stata scelta come miglior articolo pubblicato nel 2008 sulla rivista Pattern Recognition.
% \item Introduzione del clustering possibilistico nello spazio indotto da kernel semidefiniti positivi.
% Questo algoritmo si comporta da stimatore non parametrico di densit\`a nello spazio dei dati e mostra una buona robustezza agli outliers.
% Ho studiato le propriet\`a di questo algoritmo in termini di robustezza e stabilit\`a delle soluzioni, mostrando inoltre le connessioni con One Class SVM e Kernel Density Estimation.
% Questi risultati sono stati pubblicati sulla rivista {\em IEEE Transactions on Fuzzy Systems}.
% \item Studio del clustering relazionale quando gli oggetti da raggruppare sono rappresentati in termini di dissimilarit\`a non metriche fra essi.
% Questo studio \`e stato motivato dalla esperienza estera, durante il terzo anno di dottorato, trascorsa all'Information and Software Engineering department della George Mason University.
% Ho lavorato su un progetto finanziato dall'NRO, con i profs. D.~Barbar\`{a}, C.~Domeniconi, and Z.~Duric, sul riconoscimento di comportamenti sospetti in sequenze di immagini e ho dovuto affrontare il problema del clustering nel caso di dissimilarit\`a non metriche.
% Ho riportato gli studi riguardanti le operazioni algebriche necessarie per trasformare le dissimilarit\`a a coppie fra oggetti da non metriche a metriche e ho studiato come molti algoritmi di clustering vengono influenzati da queste operazioni.
% Questi studi mostrano inoltre una connessione diretta fra clustering relazionale e clustering nello spazio indotto da kernel semidefiniti positivi.
% Questa parte della tesi \`e stata pubblicata nella rivista {\em International Journal of Approximate Reasoning}.
% \end{itemize}

% In questi anni mi sono occupato inoltre di altri progetti legati al machine learning: riduzione di dimensionalit\`a e biclustering.
% I risultati di questi lavori sono stati pubblicati nei proceedings di diverse conferenze (serie LNCS), uno di essi \`e stato pubblicato come lettera su {\em Neurocomputing} e un'altro sulla rivista {\em International Journal of Knowledge Engineering and Soft Data Paradigms}.
% Applicazioni tipiche degli approcci di riduzione della dimensionalit\`a riguardano l'analisi di dati genomici, dove il numero di features \`e molto grande rispetto al numero di oggetti.

% Da Marzo 2008 sono stato Research Associate presso l'Universit\`a di Sheffield, nel gruppo di Machine Learning supervisionato dal prof. Guido Sanguinetti.
% La mia area di ricerca si \`e focalizzata su metodi statistici per pattern recognition, comprendendo entrambi i paradigmi frequentisti e bayesiani.
% Siccome il grant che finanziava la mia attivit\`a di ricerca riguardava l'individuazione di cambiamenti dello stile di vita di persone anziane per inferire cambiamenti sul loro stato di salute, gran parte dei miei studi sono stati concentrati su problemi di novelty detection.
% Abbiamo proposto un punto di vista basato sulla teria dell'informazione del problema di individuare novit\`a in insiemi di dati.
% Abbiamo studiato il caso gaussiano mostrando le connessioni con test statistici e evidenziando l'abilit\`a del metodo proposto di controllare in maniera precisa la percentuale di falsi positivi su dati di test anche quando il numero di dati di addestramento \`e piccolo.
% Forti di questo risultato, abbiamo esteso l'approccio al caso della mistura di gaussiane e delle serie temporali autoregressive.
% I risultati riguardanti il caso gaussiano e della mistura di gaussiane sono stati pubblicati sulla rivista {\em Pattern Recognition}, mentre l'estensione al caso di modelli autoregressivi appare sulla rivista {\em IEEE Transactions on Signal Processing}.

\subsection*{Progetti di Ricerca Finanziati}
\begin{itemize}
  \item \emph{Computational inference of biopathway dynamics and structures} (\pounds 350K) with D. Husmeier and S. Rogers - EPSRC research grant  (2014 - 2017)
  \item \emph{Method and software integration for systems biology} (\pounds 30K) with D. Husmeier and S. Rogers - Bridging the Gap-EPSRC research grant  (2012)
\end{itemize}

\subsection*{Premi Internazionali}
\begin{itemize}
     \item Miglior articolo pubblicato nel 2008 sulla rivista Pattern Recognition:
       \\M. Filippone, F. Camastra, F. Masulli, and S. Rovetta.
       \textbf{A survey of kernel and spectral methods for clustering}.
       \emph{Pattern Recognition}, 41(1):176-190, January 2008.
       \\Manuscripts published in volume 41 (year 2008) have been judged by the Editors-in-Chief and the members of the Editorial and Advisory Boards of the journal based on the following criteria: originality of the contribution, presentation and exposition of the manuscript, and citations by other researchers.
\end{itemize}

\subsection*{Pubblicazioni}

\textbf{Journals}\begin{itemize}\item  J. Wacker, M. Kanagawa, and M. Filippone. Improved random features for dot product kernels. \emph{Journal of Machine Learning Research}, 25(235):1-75, 2024.  
\item  A. Zammit-Mangion, M. D. Kaminski, B.-H. Tran, M. Filippone, and N. Cressie. Spatial Bayesian neural networks. \emph{Spatial Statistics}, 60:100825, 2024.  
\item  G. Franzese, S. Rossi, L. Yang, A. Finamore, D. Rossi, M. Filippone, and P. Michiardi. How much is enough? a study on diffusion times in score-based generative models. \emph{Entropy}, 25(4), 2023.  
\item  B.-H. Tran, S. Rossi, D. Milios, and M. Filippone. All you need is a good functional prior for Bayesian deep learning. \emph{Journal of Machine Learning Research}, 23(74):1-56, 2022.  
\item  S. Marmin and M. Filippone. Deep Gaussian Processes for Calibration of Computer Models (with Discussion). \emph{Bayesian Analysis}, 17(4):1301 - 1350, 2022.  
\item  C. Carota, M. Filippone, and S. Polettini. Assessing Bayesian semi-parametric log-linear models: An application to disclosure risk estimation. \emph{International Statistical Review}, 90(1):165-183, 2022.  
\item  Q. V. Andrew Zammit-Mangion, Tin Lok James Ng and M. Filippone. Deep compositional spatial models. \emph{Journal of the American Statistical Association}, 117(540):1787-1808, 2022.  
\item  R. Domingues, P. Michiardi, J. Barlet, and M. Filippone. A comparative evaluation of novelty detection algorithms for discrete sequences. \emph{Artificial Intelligence Review}, 53:3787-3812, 2020.  
\item  M. Lorenzi, M. Filippone, G. B. Frisoni, D. C. Alexander, and S. Ourselin. Probabilistic disease progression modeling to characterize diagnostic uncertainty: Application to staging and prediction in alzheimer's disease. \emph{NeuroImage}, 190:56-68, 2019.  
\item  R. Domingues, P. Michiardi, J. Zouaoui, and M. Filippone. Deep Gaussian Process autoencoders for novelty detection. \emph{Machine Learning}, 107(8-10):1363-1383, 2018.  
\item  R. Domingues, M. Filippone, P. Michiardi, and J. Zouaoui. A comparative evaluation of outlier detection algorithms: Experiments and analyses. \emph{Pattern Recognition}, 74:406-421, 2018.  
\item  M. Niu, B. Macdonald, S. Rogers, M. Filippone, and D. Husmeier. Statistical inference in mechanistic models: time warping for improved gradient matching. \emph{Computational Statistics}, pages 1-33, 8 2017.  
\item  X. Xiong, V. Šm\'{\i}dl, and M. Filippone. Adaptive multiple importance sampling for Gaussian processes. \emph{Journal of Statistical Computation and Simulation}, 87(8):1644-1665, 2017.  
\item  B. Macdonald, M. Niu, S. Rogers, M. Filippone, and D. Husmeier. Approximate parameter inference in systems biology using gradient matching: a comparative evaluation. \emph{BioMedical Engineering OnLine}, 15(Suppl 1):80, 7 2016.  
\item  J. M. Rondina, M. Filippone, M. Girolami, and N. S. Ward. Decoding post-stroke motor function from structural brain imaging. \emph{NeuroImage: Clinical}, 12:372-380, 2016.  
\item  C. Carota, M. Filippone, R. Leombruni, and S. Polettini. Bayesian nonparametric disclosure risk estimation via mixed effects log-linear models. \emph{Annals of Applied Statistics}, 9(1):525-546, 2015.  
\item  M. Dell'Amico, M. Filippone, P. Michiardi, and Y. Roudier. On user availability prediction and network applications. \emph{IEEE/ACM Transactions on Networking}, 23(4):1300-1313, Aug 2015.  
\item  M. Filippone and M. Girolami. Pseudo-marginal Bayesian inference for Gaussian processes. \emph{IEEE Transactions on Pattern Analysis and Machine Intelligence}, 36(11):2214-2226, 2014.  
\item  S. Kim, F. Valente, M. Filippone, and A. Vinciarelli. Predicting continuous conflict perception with Bayesian Gaussian processes. \emph{IEEE Transactions on Affective Computing}, 5(2):187-200, 2014.  
\item  A. F. Marquand, M. Filippone, J. Ashburner, M. Girolami, J. Mour\~ao-Miranda, G. J. Barker, S. C. R. Williams, P. N. Leigh, and C. R. V. Blain. Automated, High Accuracy Classification of Parkinsonian Disorders: A Pattern Recognition Approach. \emph{PLoS ONE}, 8(7):e69237+, 2013.  
\item  M. Filippone, M. Zhong, and M. Girolami. A comparative evaluation of stochastic-based inference methods for Gaussian process models. \emph{Machine Learning}, 93(1):93-114, 2013.  
\item  Y. Zhao, J. Kim, and M. Filippone. Aggregation algorithm towards large-scale boolean network analysis. \emph{IEEE Transactions on Automatic Control}, 58(8):1976-1985, 2013.  
\item  M. Filippone, A. F. Marquand, C. R. V. Blain, S. C. R. Williams, J. Mour\~ao-Miranda, and M. Girolami. Probabilistic prediction of neurological disorders with a statistical assessment of neuroimaging data modalities. \emph{Annals of Applied Statistics}, 6(4):1883-1905, 2012.  
\item  L. Mohamed, B. Calderhead, M. Filippone, M. Christie, and M. Girolami. Population MCMC methods for history matching and uncertainty quantification. \emph{Computational Geosciences}, 16(2):423-436, 2012.  
\item  M. Filippone and G. Sanguinetti. Approximate inference of the bandwidth in multivariate kernel density estimation. \emph{Computational Statistics \& Data Analysis}, 55(12):3104-3122, 2011.  
\item  M. Filippone and G. Sanguinetti. A perturbative approach to novelty detection in autoregressive models. \emph{IEEE Transactions on Signal Processing}, 59(3):1027-1036, 2011.  
\item  M. Filippone, F. Masulli, and S. Rovetta. Simulated annealing for supervised gene selection. \emph{Soft Computing - A Fusion of Foundations, Methodologies and Applications}, 15:1471-1482, 2011.  
\item  M. Filippone, F. Masulli, and S. Rovetta. Applying the possibilistic c-means algorithm in kernel-induced spaces. \emph{IEEE Transactions on Fuzzy Systems}, 18(3):572-584, June 2010.  
\item  M. Filippone and G. Sanguinetti. Information theoretic novelty detection. \emph{Pattern Recognition}, 43(3):805-814, March 2010.  
\item  M. Filippone. Dealing with non-metric dissimilarities in fuzzy central clustering algorithms. \emph{International Journal of Approximate Reasoning}, 50(2):363-384, February 2009.  
\item  F. Camastra and M. Filippone. A comparative evaluation of nonlinear dynamics methods for time series prediction. \emph{Neural Computing and Applications}, 18(8):1021-1029, November 2009.  
\item  M. Filippone, F. Masulli, and S. Rovetta. Clustering in the membership embedding space. \emph{International Journal of Knowledge Engineering and Soft Data Paradigms}, 4(1):363-375, 2009. 
\item  S. Rovetta, F. Masulli, and M. Filippone. Soft ranking in clustering. \emph{Neurocomputing}, 72(7-9):2028-2031, March 2009.  
\item  M. Filippone, F. Camastra, F. Masulli, and S. Rovetta. A survey of kernel and spectral methods for clustering. \emph{Pattern Recognition}, 41(1):176-190, January 2008.  

\end{itemize}\textbf{Conferences}\begin{itemize}\item  M. Heinonen, B.-H. Tran, M. Kampffmeyer, and M. Filippone. Robust classification by coupling data mollification with label smoothing. In \emph{The 28th International Conference on Artificial Intelligence and Statistics}, 2025.  
\item  A. Lh\'eritier and M. Filippone. Unconditionally calibrated priors for beta mixture density networks. In \emph{The 28th International Conference on Artificial Intelligence and Statistics}, 2025.  
\item  A. Benechehab, Y. A. E. Hili, A. Odonnat, O. Zekri, A. Thomas, G. Paolo, M. Filippone, I. Redko, and B. K\'egl. Zero-shot model-based reinforcement learning using large language models. In \emph{The Thirteenth International Conference on Learning Representations}, 2025.  
\item  T. Papamarkou, M. Skoularidou, K. Palla, L. Aitchison, J. Arbel, D. Dunson, M. Filippone, V. Fortuin, P. Hennig, J. M. Hern\'andez-Lobato, A. Hubin, A. Immer, T. Karaletsos, M. E. Khan, A. Kristiadi, Y. Li, S. Mandt, C. Nemeth, M. A. Osborne, T. G. J. Rudner, D. R\"ugamer, Y. W. Teh, M. Welling, A. G. Wilson, and R. Zhang. Position: Bayesian deep learning is needed in the age of large-scale AI. In R. Salakhutdinov, Z. Kolter, K. Heller, A. Weller, N. Oliver, J. Scarlett, and F. Berkenkamp, editors, \emph{Proceedings of the 41st International Conference on Machine Learning}, volume 235 of \emph{Proceedings of Machine Learning Research}, pages 39556-39586. PMLR, 21-27 Jul 2024.  
\item  B.-H. Tran, G. Franzese, P. Michiardi, and M. Filippone. One-line-of-code data mollification improves optimization of likelihood-based generative models. In A. Oh, T. Neumann, A. Globerson, K. Saenko, M. Hardt, and S. Levine, editors, \emph{Advances in Neural Information Processing Systems}, volume 36, pages 6545-6567. Curran Associates, Inc., 2023.  
\item  G. Franzese, G. Corallo, S. Rossi, M. Heinonen, M. Filippone, and P. Michiardi. Continuous-time functional diffusion processes. In A. Oh, T. Neumann, A. Globerson, K. Saenko, M. Hardt, and S. Levine, editors, \emph{Advances in Neural Information Processing Systems}, volume 36, pages 37370-37400. Curran Associates, Inc., 2023.  
\item  J. Wacker, R. Ohana, and M. Filippone. Complex-to-real sketches for tensor products with applications to the polynomial kernel. In F. Ruiz, J. Dy, and J.-W. van de Meent, editors, \emph{Proceedings of The 26th International Conference on Artificial Intelligence and Statistics}, volume 206 of \emph{Proceedings of Machine Learning Research}, pages 5181-5212. PMLR, 25-27 Apr 2023.  
\item  G. Franzese, D. Milios, M. Filippone, and P. Michiardi. Revisiting the effects of stochasticity for Hamiltonian samplers. In K. Chaudhuri, S. Jegelka, L. Song, C. Szepesvari, G. Niu, and S. Sabato, editors, \emph{Proceedings of the 39th International Conference on Machine Learning}, volume 162 of \emph{Proceedings of Machine Learning Research}, pages 6744-6778. PMLR, 17-23 Jul 2022.  
\item  B.-H. Tran, S. Rossi, D. Milios, P. Michiardi, E. V. Bonilla, and M. Filippone. Model selection for Bayesian autoencoders. In A. Beygelzimer, Y. Dauphin, P. Liang, and J. W. Vaughan, editors, \emph{Advances in Neural Information Processing Systems}, 2021.  
\item  G.-L. Tran, D. Milios, P. Michiardi, and M. Filippone. Sparse within Sparse Gaussian Processes using Neighbor Information. In M. Meila and T. Zhang, editors, \emph{Proceedings of the 38th International Conference on Machine Learning}, volume 139 of \emph{Proceedings of Machine Learning Research}, pages 10369-10378. PMLR, 18-24 Jul 2021.  
\item  G. Mita, M. Filippone, and P. Michiardi. An Identifiable Double VAE For Disentangled Representations. In M. Meila and T. Zhang, editors, \emph{Proceedings of the 38th International Conference on Machine Learning}, volume 139 of \emph{Proceedings of Machine Learning Research}, pages 7769-7779. PMLR, 18-24 Jul 2021.  
\item  S. Rossi, M. Heinonen, E. Bonilla, Z. Shen, and M. Filippone. Sparse Gaussian Processes Revisited: Bayesian Approaches to Inducing-Variable Approximations. In A. Banerjee and K. Fukumizu, editors, \emph{Proceedings of The 24th International Conference on Artificial Intelligence and Statistics}, volume 130 of \emph{Proceedings of Machine Learning Research}, pages 1837-1845. PMLR, 13-15 Apr 2021.  
\item  S. Rossi, S. Marmin, and M. Filippone. Walsh-Hadamard variational inference for Bayesian deep learning. In H. Larochelle, M. Ranzato, R. Hadsell, M. Balcan, and H. Lin, editors, \emph{Advances in Neural Information Processing Systems}, volume 33, pages 9674-9686. Curran Associates, Inc., 2020.  
\item  G. Mita, P. Papotti, M. Filippone, and P. Michiardi. LIBRE: Learning Interpretable Boolean Rule Ensembles. In \emph{AISTATS 2020, Palermo, Italy}, 2020.  
\item  S. Rossi, S. Marmin, and M. Filippone. Efficient approximate inference with walsh-hadamard variational inference. In \emph{Bayesian Deep Learning Workshop, NeurIPS}, 2019.  
\item  C. Nemeth, F. Lindsten, M. Filippone, and J. Hensman. Pseudo-extended Markov chain Monte Carlo. In \emph{Advances in Neural Information Processing Systems 32: Annual Conference on Neural Information Processing Systems 2019, 9-12 December 2019, Vancouver, British Columbia, Canada}, 2019.  
\item  S. Rossi, P. Michiardi, and M. Filippone. Good Initializations of Variational Bayes for Deep Models. In \emph{Proceedings of the 36th International Conference on Machine Learning, ICML 2019, Long Beach, USA, 2019}, 2019.  
\item  G.-L. Tran, E. V. Bonilla, J. P. Cunningham, P. Michiardi, and M. Filippone. Calibrating Deep Convolutional Gaussian Processes. In \emph{AISTATS 2019, Naha, Japan, 2019}, 2019.  
\item  D. Nguyen, M. Filippone, and P. Michiardi. Exact Gaussian process regression with distributed computations. In C. Hung and G. A. Papadopoulos, editors, \emph{Proceedings of the 34th ACM/SIGAPP Symposium on Applied Computing, SAC 2019, Limassol, Cyprus, April 8-12, 2019}, pages 1286-1295. ACM, 2019.  
\item  D. Milios, R. Camoriano, P. Michiardi, L. Rosasco, and M. Filippone. Dirichlet-based Gaussian Processes for Large-scale Calibrated Classification. In \emph{Advances in Neural Information Processing Systems 31: Annual Conference on Neural Information Processing Systems 2018, December 3-7 2018, Montreal, Quebec, Canada}, 2018.  
\item  M. Lorenzi and M. Filippone. Constraining the Dynamics of Deep Probabilistic Models. In \emph{Proceedings of the 35th International Conference on Machine Learning, ICML 2018, Stockholm, Sweden, 2018}, 2018.  
\item  J. Fitzsimons, D. Granziol, K. Cutajar, M. Osborne, M. Filippone, and S. Roberts. Entropic Trace Estimates for Log Determinants. In \emph{Machine Learning and Knowledge Discovery in Databases - European Conference, ECML PKDD 2017, Skopje, Macedonia, September 18-22, 2017}, 2017.  
\item  J. Fitzsimons, K. Cutajar, M. Osborne, S. Roberts, and M. Filippone. Bayesian Inference of Log Determinants. In \emph{Thirty-Third Conference on Uncertainty in Artificial Intelligence, UAI 2017, August 11-15, 2017, Sydney, Australia}, 2017.  
\item  K. Krauth, E. V. Bonilla, K. Cutajar, and M. Filippone. AutoGP: Exploring the capabilities and limitations of Gaussian process models. In \emph{Thirty-Third Conference on Uncertainty in Artificial Intelligence, UAI 2017, August 11-15, 2017, Sydney, Australia}, 2017.  
\item  K. Cutajar, E. V. Bonilla, P. Michiardi, and M. Filippone. Random feature expansions for deep Gaussian processes. In \emph{Proceedings of the 34th International Conference on Machine Learning, ICML 2017, Sydney, Australia, August 6-11, 2017}, 2017. 
\item  Y. Han and M. Filippone. Mini-batch spectral clustering. In \emph{2016 International Joint Conference on Neural Networks, IJCNN 2017, Anchorage, AK, USA, May 14-19, 2017}. IEEE, 2017. 
\item  K. Cutajar, E. V. Bonilla, P. Michiardi, and M. Filippone. Accelerating deep Gaussian processes inference with arc-cosine kernels. In \emph{Bayesian Deep Learning Workshop, NIPS}, 2016.  
\item  X. Xiong, M. Filippone, and A. Vinciarelli. Looking good with flickr faves: Gaussian processes for finding difference makers in personality impressions. In \emph{ACM Multimedia}, 2016. 
\item  K. Cutajar, M. A. Osborne, J. P. Cunningham, and M. Filippone. Preconditioning kernel matrices. In \emph{Proceedings of the 33rd International Conference on Machine Learning, ICML 2016, New York City, USA, June 19-24, 2016}, 2016. 
\item  M. Niu, S. Rogers, M. Filippone, and D. Husmeier. Fast inference in nonlinear dynamical systems using gradient matching. In \emph{Proceedings of the 33rd International Conference on Machine Learning, ICML 2016, New York City, USA, June 19-24, 2016}, 2016. 
\item  J. Hensman, A. G. de G. Matthews, M. Filippone, and Z. Ghahramani. MCMC for variationally sparse Gaussian processes. In \emph{Advances in Neural Information Processing Systems 28: Annual Conference on Neural Information Processing Systems 2015, December 7-12 2015, Montreal, Quebec, Canada}, 2015. 
\item  M. Dell'Amico and M. Filippone. Monte Carlo strength evaluation: Fast and reliable password checking. In \emph{Proceedings of the 22nd ACM Conference on Computer and Communications Security}, 2015. 
\item  M. Filippone and R. Engler. Enabling scalable stochastic gradient-based inference for Gaussian processes by employing the Unbiased LInear System SolvEr (ULISSE). In \emph{Proceedings of the 32nd International Conference on Machine Learning, ICML 2015, Lille, France, July 6-11, 2015}, 2015. 
\item  M. Filippone. Bayesian inference for Gaussian process classifiers with annealing and pseudo-marginal MCMC. In \emph{22nd International Conference on Pattern Recognition, ICPR 2014, Stockholm, Sweden, August 24-28, 2014}, pages 614-619, 2014.  
\item  A. D. O'Harney, A. Marquand, K. Rubia, K. Chantiluke, A. B. Smith, A. Cubillo, C. Blain, and M. Filippone. Pseudo-marginal Bayesian multiple-class multiple-kernel learning for neuroimaging data. In \emph{22nd International Conference on Pattern Recognition, ICPR 2014, Stockholm, Sweden, August 24-28, 2014}, pages 3185-3190, 2014.  
\item  F. Dondelinger, M. Filippone, S. Rogers, and D. Husmeier. ODE parameter inference using adaptive gradient matching with Gaussian processes. In \emph{AISTATS}, 2013. 
\item  S. Kim, M. Filippone, F. Valente, and A. Vinciarelli. Predicting the conflict level in television political debates: an approach based on crowdsourcing, nonverbal communication and Gaussian processes. In \emph{Proceedings of the 20th ACM Multimedia Conference, MM '12, Nara, Japan, October 29 - November 02, 2012}, pages 793-796. ACM, 2012.  
\item  G. Mohammadi, A. Origlia, M. Filippone, and A. Vinciarelli. From speech to personality: mapping voice quality and intonation into personality differences. In \emph{Proceedings of the 20th ACM Multimedia Conference, MM '12, Nara, Japan, October 29 - November 02, 2012}, pages 789-792. ACM, 2012.  
\item  D. Barbar\'a, C. Domeniconi, Z. Duric, M. Filippone, R. Mansfield, and E. Lawson. Detecting suspicious behavior in surveillance images. In \emph{Workshops Proceedings of the 8th IEEE International Conference on Data Mining (ICDM 2008), December 15-19, 2008, Pisa, Italy}, pages 891-900. IEEE, 2008.  
\item  M. Filippone, F. Masulli, and S. Rovetta. Stability and performances in biclustering algorithms. In \emph{Computational Intelligence Methods for Bioinformatics and Biostatistics, 5th International Meeting, CIBB 2008, Vietri sul Mare, Italy, October 3-4, 2008, Revised Selected Papers}, volume 5488 of \emph{Lecture Notes in Computer Science}, pages 91-101. Springer, 2008.  
\item  M. Filippone, F. Masulli, and S. Rovetta. An experimental comparison of kernel clustering methods. In \emph{New Directions in Neural Networks - 18th Italian Workshop on Neural Networks: WIRN 2008, Vietri sul Mare, Italy, May 22-24, 2008, Revised Selected Papers}, volume 193 of \emph{Frontiers in Artificial Intelligence and Applications}, pages 118-126. IOS Press, 2008.  
\item  F. Camastra and M. Filippone. SVM-based time series prediction with nonlinear dynamics methods. In \emph{Knowledge-Based Intelligent Information and Engineering Systems, 11th International Conference, KES 2007, XVII Italian Workshop on Neural Networks, Vietri sul Mare, Italy, September 12-14, 2007, Proceedings, Part III}, volume 4694 of \emph{Lecture Notes in Computer Science}, pages 300-307. Springer, 2007.  
\item  S. Rovetta, F. Masulli, and M. Filippone. Membership embedding space approach and spectral clustering. In \emph{Knowledge-Based Intelligent Information and Engineering Systems, 11th International Conference, KES 2007, XVII Italian Workshop on Neural Networks, Vietri sul Mare, Italy, September 12-14, 2007, Proceedings, Part III}, volume 4694 of \emph{Lecture Notes in Computer Science}, pages 901-908. Springer, 2007.  
\item  E. Canestrelli, P. Canestrelli, M. Corazza, M. Filippone, S. Giove, and F. Masulli. Local learning of tide level time series using a fuzzy approach. In \emph{Proceedings of the International Joint Conference on Neural Networks, IJCNN 2007, Celebrating 20 years of neural networks, Orlando, Florida, USA, August 12-17, 2007}, pages 1813-1818. IEEE, 2007.  
\item  M. Filippone, F. Masulli, and S. Rovetta. Possibilistic clustering in feature space. In \emph{Applications of Fuzzy Sets Theory, 7th International Workshop on Fuzzy Logic and Applications, WILF 2007, Camogli, Italy, July 7-10, 2007, Proceedings}, volume 4578 of \emph{Lecture Notes in Computer Science}, pages 219-226. Springer, 2007.  
\item  M. Filippone, F. Masulli, S. Rovetta, S. Mitra, and H. Banka. Possibilistic approach to biclustering: An application to oligonucleotide microarray data analysis. In \emph{Computational Methods in Systems Biology, International Conference, CMSB 2006, Trento, Italy, October 18-19, 2006, Proceedings}, volume 4210 of \emph{Lecture Notes in Computer Science}, pages 312-322. Springer, 2006.  
\item  M. Filippone, F. Masulli, S. Rovetta, and S.-P. Constantinescu. Input selection with mixed data sets: A simulated annealing wrapper approach. In \emph{CISI 06 - Conferenza Italiana Sistemi Intelligenti}, Ancona - Italy, 27-29 September 2006. 
\item  M. Filippone, F. Masulli, and S. Rovetta. Gene expression data analysis in the membership embedding space: A constructive approach. In \emph{CIBB 2006 - Third International Meeting on Computational Intelligence Methods for Bioinformatics and Biostatistics}, Genova - Italy, 29-31 August 2006. 
\item  M. Filippone, F. Masulli, and S. Rovetta. Supervised classification and gene selection using simulated annealing. In \emph{Proceedings of the International Joint Conference on Neural Networks, IJCNN 2006, part of the IEEE World Congress on Computational Intelligence, WCCI 2006, Vancouver, BC, Canada, 16-21 July 2006}, pages 3566-3571. IEEE, 2006.  
\item  M. Filippone, F. Masulli, and S. Rovetta. Unsupervised gene selection and clustering using simulated annealing. In \emph{Fuzzy Logic and Applications, 6th International Workshop, WILF 2005, Crema, Italy, September 15-17, 2005, Revised Selected Papers}, volume 3849 of \emph{Lecture Notes in Computer Science}, pages 229-235. Springer, 2005.  
\item  F. Masulli, S. Rovetta, and M. Filippone. Clustering genomic data in the membership embedding space. In \emph{CI-BIO - Workshop on Computational Intelligence Approaches for the Analysis of Bioinformatics Data}, Montreal - Canada, 5 August 2005. 
\item  S. Rovetta, F. Masulli, and M. Filippone. Soft rank clustering. In \emph{Neural Nets, 16th Italian Workshop on Neural Nets, WIRN 2005, and International Workshop on Natural and Artificial Immune Systems, NAIS 2005, Vietri sul Mare, Italy, June 8-11, 2005, Revised Selected Papers}, volume 3931 of \emph{Lecture Notes in Computer Science}, pages 207-213. Springer, 2005.  
\item  M. Filippone, F. Masulli, and S. Rovetta. ERAF: a R package for regression and forecasting. In \emph{Biological and Artificial Intelligence Environments}, pages 165-173, Secaucus, NJ, USA, 2004. Springer-Verlag New York, Inc. 

\end{itemize}\textbf{Discussions}\begin{itemize}\item  S. Rossi, C. Rusu, L. A. Rosasco, and M. Filippone. Contributed discussion on ``A Bayesian conjugate gradient method''. \emph{Bayesian Analysis, 14(3), 2019}, 10 2019.  
\item  M. Filippone, A. Mira, and M. Girolami. Discussion of the paper: ”Sampling schemes for generalized linear Dirichlet process random effects models” by M. Kyung, J. Gill, and G. Casella. \emph{Statistical Methods \& Applications}, 20:295-297, 2011.  
\item  M. Filippone. Discussion of the paper ”Riemann manifold Langevin and Hamiltonian Monte Carlo methods” by Mark Girolami and Ben Calderhead. \emph{Journal of the Royal Statistical Society, Series B (Statistical Methodology)}, 73(2):164-165, 2011.  
\item  V. Stathopoulos and M. Filippone. Discussion of the paper ”Riemann manifold Langevin and Hamiltonian Monte Carlo methods” by Mark Girolami and Ben Calderhead. \emph{Journal of the Royal Statistical Society, Series B (Statistical Methodology)}, 73(2):167-168, March 2011.  

\end{itemize}\textbf{Theses}\begin{itemize}\item  M. Filippone. \emph{Central Clustering in Kernel-Induced Spaces}. Phd thesis in computer science, University of Genova, February 2008. 
\item  M. Filippone. Metodi di ensemble per la previsione di serie storiche. Master's degree thesis in physics, University of Genova, July 2004. 

\end{itemize}


\subsection*{Attivit\`a Professionali}
\begin{itemize}
\item Associate editor - Pattern Recognition
\item Associate editor - IEEE Transactions on Neural Networks and Learning Systems
\end{itemize}


\subsection*{Attivit\`a in qualit\`a di Referee}
\begin{itemize}
\item Riviste:
\input{reviewer_journals.txt}
\input{reviewer_journals_low.txt}
\item Conferenze:
\input{reviewer_conferences.txt}
\end{itemize}

\subsection*{Partecipazione a Conferenze}
\begin{itemize}
\item 30 Maggio - 1 Giugno 2012, Trondheim, Norway \\
  Second Workshop on Bayesian Inference for Latent Gaussian Models with Applications.
  \\Presentazione orale: \emph{On the Fully Bayesian Treatment of Latent Gaussian Models using Stochastic Simulations}.
\item 09 June 2011, Bologna, Italy \\
  Convegno intermedio SIS 2011.
  \\Presentazione orale: \emph{Bayesian inference in latent variable models and applications}.
\item 10-11 Dicembre 2010, Whistler, BC, Canada \\
  NIPS 2010 Workshops: Neural Information Processing Systems Conference.
  \\Presentazione del Poster: \emph{Posterior Inference in Latent Gaussian Models Using Manifold MCMC Methods}.
\item 23-26 Agosto 2010, Istanbul, Turkey \\
  ICPR 2010 - 20th International Conference on Pattern Recognition
  \\\emph{Ho ricevuto il premio per il miglior articolo pubblicato nel 2008 sulla rivista Pattern Recognition}
\item 12-14 Luglio 2010, Sheffield, United Kingdom \\
  UCM 2010 - Uncertainty in Computer Models 2010 conference
\item 5-9 Luglio 2010, Glasgow, United Kingdom \\
  IWSM 2010 - 25th International Workshop on Statistical Modelling
\item 3-8 Giugno 2010, Benidorm, Spain \\
  Ninth Valencia International Meeting on Bayesian Statistics - 2010 World Meeting of the International Society for Bayesian Analysis.
  \\Presentazione del Poster: \emph{Inference for Gaussian Process Emulation of Oil Reservoir Simulation Codes}.
\item 6-7 Aprile 2010, Warwick, United Kingdom \\
  WOGAS2 - Workshop on Geometric and Algebraic Statistics 2.
\item 3-5 Marzo 2010, Edinburgh, United Kingdom \\
  Mixture estimation and applications.
  \\Presentazione del Poster: \emph{Information Theoretic Novelty Detection for Mixtures of Gaussians}.
\item 7-9 Settembre 2009, Sheffield, United Kingdom \\
  PRIB 2009: Pattern Recognition in Bioinformatics 2009.
\item 20-21 Maggio 2009, Swansea, United Kingdom \\
  NCAF Meeting: Neural Computing and Applications. Special Theme - Grand Challenges in Information-Driven Healthcare.
\item 15-19 Dicembre 2008, Pisa, Italy \\
  ICDM 2008: IEEE International Conference on Data Mining.
  \\Presentazione orale: \emph{Detecting Suspicious Behavior in Surveillance Images}.
\item 8-13 Dicembre 2008, Vancouver, BC, Canada \\
  NIPS 2008: Neural Information Processing Systems Conference.
\item 9-10 Settembre 2008, Sheffield, United Kingdom \\
  NCAF Meeting: Neural Computing and Applications. Special Theme - Dynamics and Dynamic Systems.
  \\Presentazione orale: \emph{Information Theoretic Novelty Detection}.
\item 18-19 Giugno 2008, Oxford, United Kingdom \\
  NCAF Meeting: Neural Computing and Applications. Special Theme - Signal Processing and Biomedical Applications.
\item 31 Marzo - 2 Aprile 2008, Sheffield, United Kingdom \\
  Data Modelling Workshops \& Symposium.
\item 13-17 Agosto 2007, Orlando, FL - USA \\
  IJCNN 2007 - International Joint Conferences on Neural Networks.
  \\Presentazione del Poster: \emph{Local learning of tide level time series using a fuzzy approach}.
\item 17 Agosto 2007, Orlando, FL - USA \\
  CI-BIO 2007 - Post-Conference Workshop on Computational Intelligence Approaches for the Analysis of Bioinformatics Data.
  \\Presentazione orale dell'articolo: \emph{Aggregating Memberships in Possibilistic Biclustering}.
\item 27-29 Settembre 2006, Ancona - Italy \\
  CISI 06 Conferenza Italiana Sistemi Intelligenti.
  \\Presentazione del Poster: \emph{Input Selection with Mixed Data Sets: A Simulated Annealing Wrapper Approach}.
\item 15-16 Settembre 2006, Genova - Italy \\
  BioLeMiD 06 - Third Bioinformatics Meeting on Machine Learning for Microarray Studies of Disease: biomarker selection.
\item 29-31 Agosto 2006, Genova - Italy
  FLINS 2006 - 7th International FLINS Conference on Applied Artificial Intelligence.
\item 28 Giugno 2006, Genova - Italy \\
  Workshop on Trends in Computational Sciences.
\item 22 Giugno 2006, Genova - Italy \\
  The I LIMBS day - A free one-day workshop about intelligence.
\item 21 Giugno 2006, Genova - Italy \\
  Second workshop on Analytic Methods for Learning Theory: Learning, Regularization and Approximation
\item 15-17 Settembre 2005, Crema - Italy \\
  WILF 05 - International Workshop on Fuzzy Logic and Applications.
  \\Presentazione orale dell'articolo: \emph{Unsupervised gene selection and clustering using simulated annealing}.
\item 16-17 Giugno 2005, Genova - Italy \\
  CLIP 2005 - Workshop on Cross-language information processing.
  \\Presentazione orale dell'articolo: \emph{Soft rank clustering}.
\item 8-11 Giugno 2005, Vietri sul Mare - Italy \\
  WIRN 05 - XVI Italian Workshop on Neural Networks.
  \\Presentazione orale dell'articolo: \emph{Soft rank clustering}.
\item 14-17 Settembre 2004, Perugia - Italy \\
  WIRN 04 - XV Italian Workshop on Neural Networks.
  \\Presentazione del Poster: \emph{ERAF: a R package for regression and forecasting}.
\end{itemize}

\subsection*{Presentazioni}
\begin{itemize}
\item 20 Ottobre, 2011, University of Glasgow - \emph{Inference in hierarchical models using stochastic approximations}.
\item 08 Febbraio, 2011, University College London (CSML seminars series) - \emph{Calibration of Oil Reservoir Simulation Codes}.
\item 24 Gennaio, 2011, University College London - \emph{Classification of fMRI data using latent Gaussian models}.
\item 12 Novembre, 2010, University of Glasgow - \emph{Information Theoretic Novelty Detection}.
\item 13 Ottobre, 2010, Royal Statistical Society - \emph{Discussion of the paper ``Riemann manifold Langevin and Hamiltonian Monte Carlo methods'' by M. Girolami and B. Calderhead}.
\item 26 Marzo, 2010, Liverpool John Moores University - \emph{Information Theoretic Novelty Detection}.
\item 11 Novembre, 2009, University of Edinburgh - \emph{Information Theoretic Novelty Detection}.
\item 21 Ottobre, 2009, University of Sheffield - Tutorial per il gruppo Speech and Hearing - \emph{The probabilistic approach in data modeling}.
\item 14 Luglio, 2009, Columbia University - \emph{Information Theoretic Novelty Detection}.
\item 21 Gennaio, 2009, University of Glasgow - \emph{Information Theoretic Novelty Detection}.
\item 22 Dicembre, 2008, Universit\`a di Genova - \emph{Information Theoretic Novelty Detection}.
\item 3 Marzo, 2008, Universit\`a degli Studi di Milano - \emph{Central Clustering in Kernel-Induced Spaces}.
\item 27 Febbraio, 2008, Universit\`a degli Studi di Napoli Parthenope - \emph{Central Clustering in Kernel-Induced Spaces}.
\item 27 Settembre, 2007, George Mason University - \emph{Kernel and Spectral Methods for Clustering}.
\item 15 Novembre 2006, Universit\`a di Genova - \emph{Kernel and Spectral Methods for Clustering}.
\item 21 Marzo 2006, Universit\`a di Genova - \emph{Spectral Approach to Clustering}.
\item 20 Dicembre 2005, Universit\`a di Genova - \emph{Subsequence Matching for Time Series Forecasting}.
\end{itemize}



\section*{Attivit\`a di Insegnamento}

% \subsection*{Commissioni d'esame}
% \begin{itemize}
% \item 2005 e 2006: \\
%   Reti Neurali e Soft Computing - Universit\`a di Genova, \\ Sistemi Operativi - Universit\`a di Pisa
% \end{itemize}
%\begin{itemize}
%\item 2006:
%  \begin{itemize}
%  \item Reti Neurali \\
%    IV / V anno, Corso di Laurea in Informatica - Universit\`a di Genova \\ 
%    Prof. Stefano Rovetta
%  \item Soft Computing \\
%    III anno, Corso di Laurea in Informatica - Universit\`a di Genova \\
%    Prof. Francesco Masulli
%  \item Sistemi Operativi \\
%    II anno, Corso di Laurea in Informatica Applicata - Polo di La Spezia - Universit\`a di Pisa \\
%    Prof. Francesco Masulli
%  \end{itemize}
%\item 2005:
%  \begin{itemize}
%  \item Reti Neurali \\
%    IV / V anno, Corso di Laurea in Informatica - Universit\`a di Genova \\
%    Prof. Francesco Masulli
%  \item Soft Computing \\
%    III anno, Corso di Laurea in Informatica - Universit\`a di Genova \\
%    Prof. Francesco Masulli
%  \item Sistemi Operativi \\
%    II anno, Corso di Laurea in Informatica Applicata - Polo di La Spezia - Universit\`a di Pisa \\
%    Prof. Francesco Masulli
%  \end{itemize}
%\end{itemize}

% \subsection*{Insegnamento}
\begin{itemize}
\item 09-2013 - 12-2013 \\
  Lecturer (30 ore)
  Artificial Intelligence \\
  Undergraduate Degree Programme - Anno 4
\item 01-2013 - 04-2013 \\
  Lecturer (30 ore)
  Machine Learning \\
  Undergraduate/Postgraduate Degree Programmes - Anno 4/5
\item 09-2012 - 12-2012 \\
  Lecturer (30 ore)
  Artificial Intelligence \\
  Undergraduate Degree Programme - Anno 4
\item 10-2008 \\
  Assistente di Laboratorio (2 ore)
  Bioinformatics \\
  Modulo di Computational Biology per MSc in Biological and Bioprocess Engineering \\
  Prof. Guido Sanguinetti
\item 09-2005 - 12-2005 \\
  Assistente di Laboratorio (30 ore)
  Reti Neurali \\
  IV / V anno, Corso di Laurea in Informatica - Universit\`a di Genova \\
  Prof. Stefano Rovetta
\item 09-2005 - 12-2005 \\
  Assistente di Laboratorio (10 ore)
  Soft Computing \\
  III anno, Corso di Laurea in Informatica - Universit\`a di Genova \\
  Prof. Francesco Masulli
\item 09-2004 - 12-2004 \\
  Assistente di Laboratorio (10 ore) \\
  Presentazione di un mini-corso sul linguaggio R (2 ore)
  Reti Neurali \\
  IV / V anno, Corso di Laurea in Informatica - Universit\`a di Genova \\
  Prof. Francesco Masulli
\end{itemize}

% \subsection*{Altre Attivit\`a di Insegnamento}
% \begin{itemize}
% \item 09-2005 - 06-2006 \\
%   Insegnante di Matematica (120 ore) \\
%   Insegnante di Informatica (80 ore) \\  
%   IAL LIGURIA - Scuola Professionale - Programma Operatore Commerciale \\
%   email: informazioni@ial.liguria.it, segreteria@ial.liguria.it
% \end{itemize}

%\newpage




% \section*{Capacit\`a e Competenze Personali}

% \subsection*{Lingue}
% \begin{itemize}
% \item Madrelingua: Italiano
% \item Inglese:
%   \begin{itemize}
% %    \item Reading ability: good
% %    \item Writing ability: good
% %    \item Speaking ability: good
%     \item Dal 05-2007 al 10-2007 ho frequentato alcuni corsi di grammatica, conversazione e pronuncia per non nativi americani (60 ore)
%       presso la George Mason University, Fairfax, VA 22030 - USA
%     \item Dal 20-07-2004 al 05-08-2004 ho frequentato un corso di inglese di livello intermedio (30 ore)
%       presso la Byron School, 79 Hills Road - CB2 1PG Cambridge 
%   \end{itemize}
% \end{itemize}

% \subsection*{Capacit\`a e Competenze Tecniche}
% \begin{description}
% \item[Sistemi operativi:] Windows, Unix e Linux.
% \item[Linguaggi di programmazione:] R, C, C++, Fortran e Assembler.
% \item[Linguaggi di scripting:]  Perl.
% \item[Linguaggi per il web:] HTML e PHP.
% \item[Basi di dati:] SQL.
% \item[Linguaggi di editing:] Latex e Word.
% \end{description}

\end{document}
