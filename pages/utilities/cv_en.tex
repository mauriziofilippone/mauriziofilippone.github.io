\documentclass[a4paper,10pt]{article}

\usepackage{url}
\usepackage{graphicx}
\usepackage{eurosym}

\addtolength{\topmargin}{-60pt}
\addtolength{\textheight}{60pt}
\addtolength{\oddsidemargin}{-70pt}
\addtolength{\textwidth}{140pt}

\begin{document}

\begin{center}
{\bf \LARGE Curriculum Vitae}
\end{center}

\section*{Personal Information}
\begin{description}
\item Name:               {\bf Maurizio Filippone}
%\item Place of birth: Genova
\item Nationality:        Italian
\item Current position:
{\em AXA Chair of Computational Statistics \& Associate Professor} at EURECOM, Sophia Antipolis, France
\item Email:              \texttt{maurizio.filippone@eurecom.fr}
\item Web Page:           \url{http://www.eurecom.fr/~filippon}
\end{description}

\section*{Education}
\begin{itemize}
\item From 01-01-2005 to 05-05-2008 \\
Ph.D. in Computer Science \\
Department of Computer and Information Sciences - University of Genova \\
Thesis title: Central Clustering in Kernel-Induced Spaces \\
Thesis topics: kernel methods for clustering, spectral clustering, relational clustering, fuzzy clustering.

\item From 01-09-1998 to 14-07-2004  \\
Master's Degree in Physics  \\
Department of Physics - University of Genova \\
Grades: 110/110  \\
Thesis title: Ensemble methods for time series analysis and forecasting \\
Thesis topics: Non linear systems, regression, ensemble of learning machines, signal processing.

\item From 01-09-1993 to 01-07-1998  \\
Diploma in electronics and telecommunications \\
High School I.T.I.S. Italo Calvino - Genova \\
Grades: 60/60

%\item From 01-03-1997 to 15-03-1997  \\
%Corporate stage in electronics and telecommunications \\
%at EBTech snc %, Via Greto di Cornigliano 6R 16152 
%- Genova
\end{itemize}

\section*{Research Experience}
%Period (da - a)
%Nome e indirizzo del datore di lavoro
%Tipo di azienda o settore
%Tipo di impiego
%Principali mansioni e responsabilità
\begin{itemize}
\item From Fall 2015 to Spring 2018 - {\em Associate Professor} \\
  EURECOM, Sophia Antipolis, France \\
  Keywords: Bayesian inference, Nonparametric modeling, Scalable inference

\item From Fall 2015 to Spring 2018 - {\em Ma\^{i}tre de Conf\'{e}rence} \\
  EURECOM, Sophia Antipolis, France \\
  Keywords: Bayesian inference, Nonparametric modeling, Scalable inference

\item From Fall 2011 to Fall 2015 - {\em Lecturer} \\
  School of Computing Science - University of Glasgow \\
  Keywords: Bayesian inference, Gaussian Processes, Markov chain Monte Carlo


\item From 01-01-2010 to 31-08-2011 \\
  Department of Statistical Science - University College London (2010 with the Department of Computing Science - University of Glasgow) \\
  1-19 Torrington Place, London, WC1E 7HB - United Kingdom. \\
  Grant: The Synthesis of Probabilistic Prediction \& Mechanistic Modelling within a Computational \& Systems Biology Context \\
%  Topics of the fellowship: novelty detection, statistical testing, Bayesian inference in data modeling. \\
  PI: Prof. Mark Girolami

\item From 15-03-2008 to 30-11-2009 \\
  Department of Computer Science - University of Sheffield \\
  Regent Court, 211 Portobello, Sheffield, S1 4DP - United Kingdom \\
  Grant: ALMS: Advanced Lifestyle Monitoring Systems \\
  Topics of the fellowship: novelty detection, statistical testing, Bayesian inference in data modeling. \\
  PI: Dr Guido Sanguinetti

\item From 01-03-2007 to 30-10-2007 \\
  Department of Information and Software Engineering - George Mason University \\
  4400 University Drive, Fairfax, VA 22030 - USA \\
  Grant: Detecting Suspicious Behavior in Reconnaissance Images \\
  Topics of the fellowship: outlier detection, density estimation, relational clustering. \\
  PI, co-PI: Prof. Daniel Barbar\`a and Prof. Carlotta Domeniconi

\item From 20-07-2006 to 30-10-2006 \\
  Consorzio Venezia Ricerche \\
  Via della Libert\`a 12, 30175 Marghera, Venezia - Italy \\
  Grant: Tide level forecasting in the lagoon of Venezia \\ 
  Topics of the fellowship: time series analysis and forecasting, regression, ensembles of learning machines. \\
  Supervisor: Prof. Elio Canestrelli

\item From 01-09-2005 to 30-04-2007 \\
  Department of Computer Science - University of Genova \\
  Via Dodecaneso 35, 16146 Genova - Italy \\
  Topic of the fellowship: \\
  Novel clustering techniques with applications in image segmentation and analysis \\
  Supervisor: Dr Stefano Rovetta

\item From 01-06-2005 to 31-08-2005 \\
  Department of Computer Science and Department of Endocrinologic and Metabolic Sciences - University of Genova \\
  Via Dodecaneso 35, 16146 Genova - Italy \\
  Topic of the fellowship: \\
  Application of advanced clustering techniques in diagnostic problems in rheumatology \\
  Supervisor: Prof. Guido Rovetta

\item From 01-09-2004 to 31-09-2004 \\
  Department of Computer Science - University of Genova \\
  Via Dodecaneso 35, 16146 Genova - Italy \\
  Topic of the fellowship: \\
  Development of a package in R and C languages for time series analysis and forecasting \\
  Supervisor: Prof. Francesco Masulli

\item From 01-12-2003 to 31-12-2003 \\
  Department of Oncology, Biology and Genetic (148) - University of Genova \\
  Largo Rosanna Benzi 10, 16132 Genova - Italy \\
  Topic of the fellowship: \\
  Development of a front-end in Perl Tk and Java languages for an immunitary system simulator in C language \\
  Supervisor: Prof. Franco Celada

\end{itemize}



\section*{Research Activity}



% I started my research career in 2004, in the Department of Computer and Information Sciences at the University of Genova.
% I worked on my Master's thesis in Physics, supervised by Profs. F.~Masulli and S.~Rovetta, on a comparative study of ensembles of learning machines for regression tasks, with a special interest in time series analysis and forecasting.
% The analysis of the time series involved the study of some techniques to estimate the number of past observations needed to obtain a faithful reconstruction of the dynamic of the system (model order).
% A comparative study of some of these methods has been published on the {\em Neural Computing and Applications} journal.
% The studied and implemented ensemble methods were bagging and adaboost using neural networks and SVMs.
% I applied these methods to the tide level forecasting problem in the lagoon of Venice obtaining an improvement in the forecasting accuracy with respect to the methods currently in use.
% The main results have been published in the {\em IJCNN 2007} conference proceedings.

% When I started the Ph.D. in 2005, I focused on clustering problems.
% The main results of my Ph.D. thesis are the following:
% \begin{itemize}
% \item Survey of the literature on kernel and spectral approaches to clustering, reporting the links between them.
% The review has been published as a survey paper on {\em Pattern Recognition} journal, and it has been selected to be the best paper published in 2008 in the journal Pattern Recognition.
% \item Proposing the possibilistic $c$-means in the space induced by positive semidefinite kernels.
% This algorithm is essentially a non parametric estimator of densities in the data space and shows high robustness to outliers.
% I studied its properties in terms of robustness and stability of the solutions, highlighting the connections to One Class SVMs and Kernel Density Estimation.
% These results have been published on {\em IEEE Transactions on Fuzzy Systems}.
% \item Studying the relational clustering when the objects to cluster are described in term of non-metric pairwise dissimilarities.
% This study has been motivated by my experience abroad during the third year at the Information and Software Engineering department at George Mason University.
% I worked on a project founded by NRO, with profs. D.~Barbar\`{a}, C.~Domeniconi, and Z.~Duric, on the detection of suspicious behaviors in reconnaissance images, and I dealt with the clustering problem in the case of non-metric dissimilarities.
% I reported the studies on the algebraic operations transforming the dissimilarities between patterns from non-metric to metric, and I studied the effects of these oprations on many clustering algorithms.
% These studies showed also a direct link between relational clustering and clustering in the space induced by positive semidefinite kernels.
% This part of the thesis has been published as a regular paper in the {\em International Journal of Approximate Reasoning}.
% \end{itemize}

% During these years I also worked on other machine learning related projects, such as dimensionality reduction and biclustering.
% The results of these works have been published in several refereed conference proceedings (LNCS series), one of them has been published as a letter on {\em Neurocomputing}, and another one will appear on {\em International Journal of Knowledge Engineering and Soft Data Paradigms}.
% Typical applications of dimensionality reduction approaches can be found in genomic data analysis, where the number of features is very large with respect to the number of patterns.

% In March 2008 I was Research Associate at the University of Sheffield with the Machine Learning group, supervised by Dr Guido Sanguinetti.
% My research area was on statistical methods for pattern recognition, including both the frequentist and Bayesian paradigms.
% Since the grant that was funding my research activity was about detecting changes in the lifestyle of elderly people to infer changes in their health conditions, much of my studies were on the novelty detection problem.
% We recast the novelty detection problem in the framework of information theory.
% We studied the Gaussian case, showing that the proposed method leads to a statistical test that is able to control the false positive rate on test data even in the case of small training sets.
% Based on this remarkable result, we extended the approach to the mixture of Gaussians and to linear autoregressive time series.
% The results for the Gaussian and mixture of Gaussians cases have been collected in a paper that has been published on {\em Pattern Recognition}, and the extension to autoregressive time-series have been published on the {\em IEEE Transactions on Signal Processing}.

\subsection*{Research Grants}
\begin{itemize}
  \item \emph{ECO-ML: Rethinking Modern Machine Learning Tools for a New Generation of Low-Power Large-Scale Modeling Systems} (300K\euro), 2018--2021, ANR-JCJC 
  \item AXA Chair of Computational Statistics: \emph{New Computational Approaches to Risk Modeling} (600K\euro), 2016--2023, AXA Research Fund 
  \item \emph{Computational inference of biopathway dynamics and structures} (\pounds 350K) with D. Husmeier and S. Rogers - EPSRC research grant (2014 -- 2017)
  \item \emph{Method and software integration for systems biology} (\pounds 30K) with D. Husmeier and S. Rogers - Bridging the Gap-EPSRC research grant (2012)
\end{itemize}

\subsection*{Awards}
\begin{itemize}
     \item Best paper published in 2008 in the journal Pattern Recognition:
       \\M. Filippone, F. Camastra, F. Masulli, and S. Rovetta.
       \textbf{A survey of kernel and spectral methods for clustering}.
       \emph{Pattern Recognition}, 41(1):176-190, January 2008.
       \\Manuscripts published in volume 41 (year 2008) have been judged by the Editors-in-Chief and the members of the Editorial and Advisory Boards of the journal based on the following criteria: originality of the contribution, presentation and exposition of the manuscript, and citations by other researchers.
\end{itemize}

\subsection*{Publications}

\textbf{Journals}\begin{itemize}\item  J. Wacker, M. Kanagawa, and M. Filippone. Improved random features for dot product kernels. \emph{Journal of Machine Learning Research}, 25(235):1-75, 2024.  
\item  A. Zammit-Mangion, M. D. Kaminski, B.-H. Tran, M. Filippone, and N. Cressie. Spatial Bayesian neural networks. \emph{Spatial Statistics}, 60:100825, 2024.  
\item  G. Franzese, S. Rossi, L. Yang, A. Finamore, D. Rossi, M. Filippone, and P. Michiardi. How much is enough? a study on diffusion times in score-based generative models. \emph{Entropy}, 25(4), 2023.  
\item  B.-H. Tran, S. Rossi, D. Milios, and M. Filippone. All you need is a good functional prior for Bayesian deep learning. \emph{Journal of Machine Learning Research}, 23(74):1-56, 2022.  
\item  S. Marmin and M. Filippone. Deep Gaussian Processes for Calibration of Computer Models (with Discussion). \emph{Bayesian Analysis}, 17(4):1301 - 1350, 2022.  
\item  C. Carota, M. Filippone, and S. Polettini. Assessing Bayesian semi-parametric log-linear models: An application to disclosure risk estimation. \emph{International Statistical Review}, 90(1):165-183, 2022.  
\item  Q. V. Andrew Zammit-Mangion, Tin Lok James Ng and M. Filippone. Deep compositional spatial models. \emph{Journal of the American Statistical Association}, 117(540):1787-1808, 2022.  
\item  R. Domingues, P. Michiardi, J. Barlet, and M. Filippone. A comparative evaluation of novelty detection algorithms for discrete sequences. \emph{Artificial Intelligence Review}, 53:3787-3812, 2020.  
\item  M. Lorenzi, M. Filippone, G. B. Frisoni, D. C. Alexander, and S. Ourselin. Probabilistic disease progression modeling to characterize diagnostic uncertainty: Application to staging and prediction in alzheimer's disease. \emph{NeuroImage}, 190:56-68, 2019.  
\item  R. Domingues, P. Michiardi, J. Zouaoui, and M. Filippone. Deep Gaussian Process autoencoders for novelty detection. \emph{Machine Learning}, 107(8-10):1363-1383, 2018.  
\item  R. Domingues, M. Filippone, P. Michiardi, and J. Zouaoui. A comparative evaluation of outlier detection algorithms: Experiments and analyses. \emph{Pattern Recognition}, 74:406-421, 2018.  
\item  M. Niu, B. Macdonald, S. Rogers, M. Filippone, and D. Husmeier. Statistical inference in mechanistic models: time warping for improved gradient matching. \emph{Computational Statistics}, pages 1-33, 8 2017.  
\item  X. Xiong, V. Šm\'{\i}dl, and M. Filippone. Adaptive multiple importance sampling for Gaussian processes. \emph{Journal of Statistical Computation and Simulation}, 87(8):1644-1665, 2017.  
\item  B. Macdonald, M. Niu, S. Rogers, M. Filippone, and D. Husmeier. Approximate parameter inference in systems biology using gradient matching: a comparative evaluation. \emph{BioMedical Engineering OnLine}, 15(Suppl 1):80, 7 2016.  
\item  J. M. Rondina, M. Filippone, M. Girolami, and N. S. Ward. Decoding post-stroke motor function from structural brain imaging. \emph{NeuroImage: Clinical}, 12:372-380, 2016.  
\item  C. Carota, M. Filippone, R. Leombruni, and S. Polettini. Bayesian nonparametric disclosure risk estimation via mixed effects log-linear models. \emph{Annals of Applied Statistics}, 9(1):525-546, 2015.  
\item  M. Dell'Amico, M. Filippone, P. Michiardi, and Y. Roudier. On user availability prediction and network applications. \emph{IEEE/ACM Transactions on Networking}, 23(4):1300-1313, Aug 2015.  
\item  M. Filippone and M. Girolami. Pseudo-marginal Bayesian inference for Gaussian processes. \emph{IEEE Transactions on Pattern Analysis and Machine Intelligence}, 36(11):2214-2226, 2014.  
\item  S. Kim, F. Valente, M. Filippone, and A. Vinciarelli. Predicting continuous conflict perception with Bayesian Gaussian processes. \emph{IEEE Transactions on Affective Computing}, 5(2):187-200, 2014.  
\item  A. F. Marquand, M. Filippone, J. Ashburner, M. Girolami, J. Mour\~ao-Miranda, G. J. Barker, S. C. R. Williams, P. N. Leigh, and C. R. V. Blain. Automated, High Accuracy Classification of Parkinsonian Disorders: A Pattern Recognition Approach. \emph{PLoS ONE}, 8(7):e69237+, 2013.  
\item  M. Filippone, M. Zhong, and M. Girolami. A comparative evaluation of stochastic-based inference methods for Gaussian process models. \emph{Machine Learning}, 93(1):93-114, 2013.  
\item  Y. Zhao, J. Kim, and M. Filippone. Aggregation algorithm towards large-scale boolean network analysis. \emph{IEEE Transactions on Automatic Control}, 58(8):1976-1985, 2013.  
\item  M. Filippone, A. F. Marquand, C. R. V. Blain, S. C. R. Williams, J. Mour\~ao-Miranda, and M. Girolami. Probabilistic prediction of neurological disorders with a statistical assessment of neuroimaging data modalities. \emph{Annals of Applied Statistics}, 6(4):1883-1905, 2012.  
\item  L. Mohamed, B. Calderhead, M. Filippone, M. Christie, and M. Girolami. Population MCMC methods for history matching and uncertainty quantification. \emph{Computational Geosciences}, 16(2):423-436, 2012.  
\item  M. Filippone and G. Sanguinetti. Approximate inference of the bandwidth in multivariate kernel density estimation. \emph{Computational Statistics \& Data Analysis}, 55(12):3104-3122, 2011.  
\item  M. Filippone and G. Sanguinetti. A perturbative approach to novelty detection in autoregressive models. \emph{IEEE Transactions on Signal Processing}, 59(3):1027-1036, 2011.  
\item  M. Filippone, F. Masulli, and S. Rovetta. Simulated annealing for supervised gene selection. \emph{Soft Computing - A Fusion of Foundations, Methodologies and Applications}, 15:1471-1482, 2011.  
\item  M. Filippone, F. Masulli, and S. Rovetta. Applying the possibilistic c-means algorithm in kernel-induced spaces. \emph{IEEE Transactions on Fuzzy Systems}, 18(3):572-584, June 2010.  
\item  M. Filippone and G. Sanguinetti. Information theoretic novelty detection. \emph{Pattern Recognition}, 43(3):805-814, March 2010.  
\item  M. Filippone. Dealing with non-metric dissimilarities in fuzzy central clustering algorithms. \emph{International Journal of Approximate Reasoning}, 50(2):363-384, February 2009.  
\item  F. Camastra and M. Filippone. A comparative evaluation of nonlinear dynamics methods for time series prediction. \emph{Neural Computing and Applications}, 18(8):1021-1029, November 2009.  
\item  M. Filippone, F. Masulli, and S. Rovetta. Clustering in the membership embedding space. \emph{International Journal of Knowledge Engineering and Soft Data Paradigms}, 4(1):363-375, 2009. 
\item  S. Rovetta, F. Masulli, and M. Filippone. Soft ranking in clustering. \emph{Neurocomputing}, 72(7-9):2028-2031, March 2009.  
\item  M. Filippone, F. Camastra, F. Masulli, and S. Rovetta. A survey of kernel and spectral methods for clustering. \emph{Pattern Recognition}, 41(1):176-190, January 2008.  

\end{itemize}\textbf{Conferences}\begin{itemize}\item  M. Heinonen, B.-H. Tran, M. Kampffmeyer, and M. Filippone. Robust classification by coupling data mollification with label smoothing. In \emph{The 28th International Conference on Artificial Intelligence and Statistics}, 2025.  
\item  A. Lh\'eritier and M. Filippone. Unconditionally calibrated priors for beta mixture density networks. In \emph{The 28th International Conference on Artificial Intelligence and Statistics}, 2025.  
\item  A. Benechehab, Y. A. E. Hili, A. Odonnat, O. Zekri, A. Thomas, G. Paolo, M. Filippone, I. Redko, and B. K\'egl. Zero-shot model-based reinforcement learning using large language models. In \emph{The Thirteenth International Conference on Learning Representations}, 2025.  
\item  T. Papamarkou, M. Skoularidou, K. Palla, L. Aitchison, J. Arbel, D. Dunson, M. Filippone, V. Fortuin, P. Hennig, J. M. Hern\'andez-Lobato, A. Hubin, A. Immer, T. Karaletsos, M. E. Khan, A. Kristiadi, Y. Li, S. Mandt, C. Nemeth, M. A. Osborne, T. G. J. Rudner, D. R\"ugamer, Y. W. Teh, M. Welling, A. G. Wilson, and R. Zhang. Position: Bayesian deep learning is needed in the age of large-scale AI. In R. Salakhutdinov, Z. Kolter, K. Heller, A. Weller, N. Oliver, J. Scarlett, and F. Berkenkamp, editors, \emph{Proceedings of the 41st International Conference on Machine Learning}, volume 235 of \emph{Proceedings of Machine Learning Research}, pages 39556-39586. PMLR, 21-27 Jul 2024.  
\item  B.-H. Tran, G. Franzese, P. Michiardi, and M. Filippone. One-line-of-code data mollification improves optimization of likelihood-based generative models. In A. Oh, T. Neumann, A. Globerson, K. Saenko, M. Hardt, and S. Levine, editors, \emph{Advances in Neural Information Processing Systems}, volume 36, pages 6545-6567. Curran Associates, Inc., 2023.  
\item  G. Franzese, G. Corallo, S. Rossi, M. Heinonen, M. Filippone, and P. Michiardi. Continuous-time functional diffusion processes. In A. Oh, T. Neumann, A. Globerson, K. Saenko, M. Hardt, and S. Levine, editors, \emph{Advances in Neural Information Processing Systems}, volume 36, pages 37370-37400. Curran Associates, Inc., 2023.  
\item  J. Wacker, R. Ohana, and M. Filippone. Complex-to-real sketches for tensor products with applications to the polynomial kernel. In F. Ruiz, J. Dy, and J.-W. van de Meent, editors, \emph{Proceedings of The 26th International Conference on Artificial Intelligence and Statistics}, volume 206 of \emph{Proceedings of Machine Learning Research}, pages 5181-5212. PMLR, 25-27 Apr 2023.  
\item  G. Franzese, D. Milios, M. Filippone, and P. Michiardi. Revisiting the effects of stochasticity for Hamiltonian samplers. In K. Chaudhuri, S. Jegelka, L. Song, C. Szepesvari, G. Niu, and S. Sabato, editors, \emph{Proceedings of the 39th International Conference on Machine Learning}, volume 162 of \emph{Proceedings of Machine Learning Research}, pages 6744-6778. PMLR, 17-23 Jul 2022.  
\item  B.-H. Tran, S. Rossi, D. Milios, P. Michiardi, E. V. Bonilla, and M. Filippone. Model selection for Bayesian autoencoders. In A. Beygelzimer, Y. Dauphin, P. Liang, and J. W. Vaughan, editors, \emph{Advances in Neural Information Processing Systems}, 2021.  
\item  G.-L. Tran, D. Milios, P. Michiardi, and M. Filippone. Sparse within Sparse Gaussian Processes using Neighbor Information. In M. Meila and T. Zhang, editors, \emph{Proceedings of the 38th International Conference on Machine Learning}, volume 139 of \emph{Proceedings of Machine Learning Research}, pages 10369-10378. PMLR, 18-24 Jul 2021.  
\item  G. Mita, M. Filippone, and P. Michiardi. An Identifiable Double VAE For Disentangled Representations. In M. Meila and T. Zhang, editors, \emph{Proceedings of the 38th International Conference on Machine Learning}, volume 139 of \emph{Proceedings of Machine Learning Research}, pages 7769-7779. PMLR, 18-24 Jul 2021.  
\item  S. Rossi, M. Heinonen, E. Bonilla, Z. Shen, and M. Filippone. Sparse Gaussian Processes Revisited: Bayesian Approaches to Inducing-Variable Approximations. In A. Banerjee and K. Fukumizu, editors, \emph{Proceedings of The 24th International Conference on Artificial Intelligence and Statistics}, volume 130 of \emph{Proceedings of Machine Learning Research}, pages 1837-1845. PMLR, 13-15 Apr 2021.  
\item  S. Rossi, S. Marmin, and M. Filippone. Walsh-Hadamard variational inference for Bayesian deep learning. In H. Larochelle, M. Ranzato, R. Hadsell, M. Balcan, and H. Lin, editors, \emph{Advances in Neural Information Processing Systems}, volume 33, pages 9674-9686. Curran Associates, Inc., 2020.  
\item  G. Mita, P. Papotti, M. Filippone, and P. Michiardi. LIBRE: Learning Interpretable Boolean Rule Ensembles. In \emph{AISTATS 2020, Palermo, Italy}, 2020.  
\item  S. Rossi, S. Marmin, and M. Filippone. Efficient approximate inference with walsh-hadamard variational inference. In \emph{Bayesian Deep Learning Workshop, NeurIPS}, 2019.  
\item  C. Nemeth, F. Lindsten, M. Filippone, and J. Hensman. Pseudo-extended Markov chain Monte Carlo. In \emph{Advances in Neural Information Processing Systems 32: Annual Conference on Neural Information Processing Systems 2019, 9-12 December 2019, Vancouver, British Columbia, Canada}, 2019.  
\item  S. Rossi, P. Michiardi, and M. Filippone. Good Initializations of Variational Bayes for Deep Models. In \emph{Proceedings of the 36th International Conference on Machine Learning, ICML 2019, Long Beach, USA, 2019}, 2019.  
\item  G.-L. Tran, E. V. Bonilla, J. P. Cunningham, P. Michiardi, and M. Filippone. Calibrating Deep Convolutional Gaussian Processes. In \emph{AISTATS 2019, Naha, Japan, 2019}, 2019.  
\item  D. Nguyen, M. Filippone, and P. Michiardi. Exact Gaussian process regression with distributed computations. In C. Hung and G. A. Papadopoulos, editors, \emph{Proceedings of the 34th ACM/SIGAPP Symposium on Applied Computing, SAC 2019, Limassol, Cyprus, April 8-12, 2019}, pages 1286-1295. ACM, 2019.  
\item  D. Milios, R. Camoriano, P. Michiardi, L. Rosasco, and M. Filippone. Dirichlet-based Gaussian Processes for Large-scale Calibrated Classification. In \emph{Advances in Neural Information Processing Systems 31: Annual Conference on Neural Information Processing Systems 2018, December 3-7 2018, Montreal, Quebec, Canada}, 2018.  
\item  M. Lorenzi and M. Filippone. Constraining the Dynamics of Deep Probabilistic Models. In \emph{Proceedings of the 35th International Conference on Machine Learning, ICML 2018, Stockholm, Sweden, 2018}, 2018.  
\item  J. Fitzsimons, D. Granziol, K. Cutajar, M. Osborne, M. Filippone, and S. Roberts. Entropic Trace Estimates for Log Determinants. In \emph{Machine Learning and Knowledge Discovery in Databases - European Conference, ECML PKDD 2017, Skopje, Macedonia, September 18-22, 2017}, 2017.  
\item  J. Fitzsimons, K. Cutajar, M. Osborne, S. Roberts, and M. Filippone. Bayesian Inference of Log Determinants. In \emph{Thirty-Third Conference on Uncertainty in Artificial Intelligence, UAI 2017, August 11-15, 2017, Sydney, Australia}, 2017.  
\item  K. Krauth, E. V. Bonilla, K. Cutajar, and M. Filippone. AutoGP: Exploring the capabilities and limitations of Gaussian process models. In \emph{Thirty-Third Conference on Uncertainty in Artificial Intelligence, UAI 2017, August 11-15, 2017, Sydney, Australia}, 2017.  
\item  K. Cutajar, E. V. Bonilla, P. Michiardi, and M. Filippone. Random feature expansions for deep Gaussian processes. In \emph{Proceedings of the 34th International Conference on Machine Learning, ICML 2017, Sydney, Australia, August 6-11, 2017}, 2017. 
\item  Y. Han and M. Filippone. Mini-batch spectral clustering. In \emph{2016 International Joint Conference on Neural Networks, IJCNN 2017, Anchorage, AK, USA, May 14-19, 2017}. IEEE, 2017. 
\item  K. Cutajar, E. V. Bonilla, P. Michiardi, and M. Filippone. Accelerating deep Gaussian processes inference with arc-cosine kernels. In \emph{Bayesian Deep Learning Workshop, NIPS}, 2016.  
\item  X. Xiong, M. Filippone, and A. Vinciarelli. Looking good with flickr faves: Gaussian processes for finding difference makers in personality impressions. In \emph{ACM Multimedia}, 2016. 
\item  K. Cutajar, M. A. Osborne, J. P. Cunningham, and M. Filippone. Preconditioning kernel matrices. In \emph{Proceedings of the 33rd International Conference on Machine Learning, ICML 2016, New York City, USA, June 19-24, 2016}, 2016. 
\item  M. Niu, S. Rogers, M. Filippone, and D. Husmeier. Fast inference in nonlinear dynamical systems using gradient matching. In \emph{Proceedings of the 33rd International Conference on Machine Learning, ICML 2016, New York City, USA, June 19-24, 2016}, 2016. 
\item  J. Hensman, A. G. de G. Matthews, M. Filippone, and Z. Ghahramani. MCMC for variationally sparse Gaussian processes. In \emph{Advances in Neural Information Processing Systems 28: Annual Conference on Neural Information Processing Systems 2015, December 7-12 2015, Montreal, Quebec, Canada}, 2015. 
\item  M. Dell'Amico and M. Filippone. Monte Carlo strength evaluation: Fast and reliable password checking. In \emph{Proceedings of the 22nd ACM Conference on Computer and Communications Security}, 2015. 
\item  M. Filippone and R. Engler. Enabling scalable stochastic gradient-based inference for Gaussian processes by employing the Unbiased LInear System SolvEr (ULISSE). In \emph{Proceedings of the 32nd International Conference on Machine Learning, ICML 2015, Lille, France, July 6-11, 2015}, 2015. 
\item  M. Filippone. Bayesian inference for Gaussian process classifiers with annealing and pseudo-marginal MCMC. In \emph{22nd International Conference on Pattern Recognition, ICPR 2014, Stockholm, Sweden, August 24-28, 2014}, pages 614-619, 2014.  
\item  A. D. O'Harney, A. Marquand, K. Rubia, K. Chantiluke, A. B. Smith, A. Cubillo, C. Blain, and M. Filippone. Pseudo-marginal Bayesian multiple-class multiple-kernel learning for neuroimaging data. In \emph{22nd International Conference on Pattern Recognition, ICPR 2014, Stockholm, Sweden, August 24-28, 2014}, pages 3185-3190, 2014.  
\item  F. Dondelinger, M. Filippone, S. Rogers, and D. Husmeier. ODE parameter inference using adaptive gradient matching with Gaussian processes. In \emph{AISTATS}, 2013. 
\item  S. Kim, M. Filippone, F. Valente, and A. Vinciarelli. Predicting the conflict level in television political debates: an approach based on crowdsourcing, nonverbal communication and Gaussian processes. In \emph{Proceedings of the 20th ACM Multimedia Conference, MM '12, Nara, Japan, October 29 - November 02, 2012}, pages 793-796. ACM, 2012.  
\item  G. Mohammadi, A. Origlia, M. Filippone, and A. Vinciarelli. From speech to personality: mapping voice quality and intonation into personality differences. In \emph{Proceedings of the 20th ACM Multimedia Conference, MM '12, Nara, Japan, October 29 - November 02, 2012}, pages 789-792. ACM, 2012.  
\item  D. Barbar\'a, C. Domeniconi, Z. Duric, M. Filippone, R. Mansfield, and E. Lawson. Detecting suspicious behavior in surveillance images. In \emph{Workshops Proceedings of the 8th IEEE International Conference on Data Mining (ICDM 2008), December 15-19, 2008, Pisa, Italy}, pages 891-900. IEEE, 2008.  
\item  M. Filippone, F. Masulli, and S. Rovetta. Stability and performances in biclustering algorithms. In \emph{Computational Intelligence Methods for Bioinformatics and Biostatistics, 5th International Meeting, CIBB 2008, Vietri sul Mare, Italy, October 3-4, 2008, Revised Selected Papers}, volume 5488 of \emph{Lecture Notes in Computer Science}, pages 91-101. Springer, 2008.  
\item  M. Filippone, F. Masulli, and S. Rovetta. An experimental comparison of kernel clustering methods. In \emph{New Directions in Neural Networks - 18th Italian Workshop on Neural Networks: WIRN 2008, Vietri sul Mare, Italy, May 22-24, 2008, Revised Selected Papers}, volume 193 of \emph{Frontiers in Artificial Intelligence and Applications}, pages 118-126. IOS Press, 2008.  
\item  F. Camastra and M. Filippone. SVM-based time series prediction with nonlinear dynamics methods. In \emph{Knowledge-Based Intelligent Information and Engineering Systems, 11th International Conference, KES 2007, XVII Italian Workshop on Neural Networks, Vietri sul Mare, Italy, September 12-14, 2007, Proceedings, Part III}, volume 4694 of \emph{Lecture Notes in Computer Science}, pages 300-307. Springer, 2007.  
\item  S. Rovetta, F. Masulli, and M. Filippone. Membership embedding space approach and spectral clustering. In \emph{Knowledge-Based Intelligent Information and Engineering Systems, 11th International Conference, KES 2007, XVII Italian Workshop on Neural Networks, Vietri sul Mare, Italy, September 12-14, 2007, Proceedings, Part III}, volume 4694 of \emph{Lecture Notes in Computer Science}, pages 901-908. Springer, 2007.  
\item  E. Canestrelli, P. Canestrelli, M. Corazza, M. Filippone, S. Giove, and F. Masulli. Local learning of tide level time series using a fuzzy approach. In \emph{Proceedings of the International Joint Conference on Neural Networks, IJCNN 2007, Celebrating 20 years of neural networks, Orlando, Florida, USA, August 12-17, 2007}, pages 1813-1818. IEEE, 2007.  
\item  M. Filippone, F. Masulli, and S. Rovetta. Possibilistic clustering in feature space. In \emph{Applications of Fuzzy Sets Theory, 7th International Workshop on Fuzzy Logic and Applications, WILF 2007, Camogli, Italy, July 7-10, 2007, Proceedings}, volume 4578 of \emph{Lecture Notes in Computer Science}, pages 219-226. Springer, 2007.  
\item  M. Filippone, F. Masulli, S. Rovetta, S. Mitra, and H. Banka. Possibilistic approach to biclustering: An application to oligonucleotide microarray data analysis. In \emph{Computational Methods in Systems Biology, International Conference, CMSB 2006, Trento, Italy, October 18-19, 2006, Proceedings}, volume 4210 of \emph{Lecture Notes in Computer Science}, pages 312-322. Springer, 2006.  
\item  M. Filippone, F. Masulli, S. Rovetta, and S.-P. Constantinescu. Input selection with mixed data sets: A simulated annealing wrapper approach. In \emph{CISI 06 - Conferenza Italiana Sistemi Intelligenti}, Ancona - Italy, 27-29 September 2006. 
\item  M. Filippone, F. Masulli, and S. Rovetta. Gene expression data analysis in the membership embedding space: A constructive approach. In \emph{CIBB 2006 - Third International Meeting on Computational Intelligence Methods for Bioinformatics and Biostatistics}, Genova - Italy, 29-31 August 2006. 
\item  M. Filippone, F. Masulli, and S. Rovetta. Supervised classification and gene selection using simulated annealing. In \emph{Proceedings of the International Joint Conference on Neural Networks, IJCNN 2006, part of the IEEE World Congress on Computational Intelligence, WCCI 2006, Vancouver, BC, Canada, 16-21 July 2006}, pages 3566-3571. IEEE, 2006.  
\item  M. Filippone, F. Masulli, and S. Rovetta. Unsupervised gene selection and clustering using simulated annealing. In \emph{Fuzzy Logic and Applications, 6th International Workshop, WILF 2005, Crema, Italy, September 15-17, 2005, Revised Selected Papers}, volume 3849 of \emph{Lecture Notes in Computer Science}, pages 229-235. Springer, 2005.  
\item  F. Masulli, S. Rovetta, and M. Filippone. Clustering genomic data in the membership embedding space. In \emph{CI-BIO - Workshop on Computational Intelligence Approaches for the Analysis of Bioinformatics Data}, Montreal - Canada, 5 August 2005. 
\item  S. Rovetta, F. Masulli, and M. Filippone. Soft rank clustering. In \emph{Neural Nets, 16th Italian Workshop on Neural Nets, WIRN 2005, and International Workshop on Natural and Artificial Immune Systems, NAIS 2005, Vietri sul Mare, Italy, June 8-11, 2005, Revised Selected Papers}, volume 3931 of \emph{Lecture Notes in Computer Science}, pages 207-213. Springer, 2005.  
\item  M. Filippone, F. Masulli, and S. Rovetta. ERAF: a R package for regression and forecasting. In \emph{Biological and Artificial Intelligence Environments}, pages 165-173, Secaucus, NJ, USA, 2004. Springer-Verlag New York, Inc. 

\end{itemize}\textbf{Discussions}\begin{itemize}\item  S. Rossi, C. Rusu, L. A. Rosasco, and M. Filippone. Contributed discussion on ``A Bayesian conjugate gradient method''. \emph{Bayesian Analysis, 14(3), 2019}, 10 2019.  
\item  M. Filippone, A. Mira, and M. Girolami. Discussion of the paper: ”Sampling schemes for generalized linear Dirichlet process random effects models” by M. Kyung, J. Gill, and G. Casella. \emph{Statistical Methods \& Applications}, 20:295-297, 2011.  
\item  M. Filippone. Discussion of the paper ”Riemann manifold Langevin and Hamiltonian Monte Carlo methods” by Mark Girolami and Ben Calderhead. \emph{Journal of the Royal Statistical Society, Series B (Statistical Methodology)}, 73(2):164-165, 2011.  
\item  V. Stathopoulos and M. Filippone. Discussion of the paper ”Riemann manifold Langevin and Hamiltonian Monte Carlo methods” by Mark Girolami and Ben Calderhead. \emph{Journal of the Royal Statistical Society, Series B (Statistical Methodology)}, 73(2):167-168, March 2011.  

\end{itemize}\textbf{Theses}\begin{itemize}\item  M. Filippone. \emph{Central Clustering in Kernel-Induced Spaces}. Phd thesis in computer science, University of Genova, February 2008. 
\item  M. Filippone. Metodi di ensemble per la previsione di serie storiche. Master's degree thesis in physics, University of Genova, July 2004. 

\end{itemize}


\subsection*{Professional Activities}
\begin{itemize}
\item {\em Associate Editor} for Pattern Recognition (end 2012 - end 2016)
\item {\em Associate Editor} for the IEEE Transactions on Neural Networks and Learning Systems (2013 - end 2016)
\item {\em Technical Program Chair} for IJCNN 2014
\end{itemize}

\subsection*{Referee Activity}
\begin{itemize}
\item Journals:
\input{reviewer_journals.txt}
\input{reviewer_journals_low.txt}
\item Conferences: 
\input{reviewer_conferences.txt}
\end{itemize}

\subsection*{Conferences Attended}
\begin{itemize}
\item 30 May - 1 June 2012, Trondheim, Norway \\
  Second Workshop on Bayesian Inference for Latent Gaussian Models with Applications.
  \\Oral presentation: \emph{On the Fully Bayesian Treatment of Latent Gaussian Models using Stochastic Simulations}.
\item 09 June 2011, Bologna, Italy \\
  Italian Statistical Society Conference (SIS).
  \\Oral presentation: \emph{Bayesian inference in latent variable models and applications}.
\item 10-11 December 2010, Whistler, BC, Canada \\
  NIPS 2010 Workshops: Neural Information Processing Systems Conference.
  \\Poster presentation: \emph{Posterior Inference in Latent Gaussian Models Using Manifold MCMC Methods}.
\item 23-26 August 2010, Istanbul, Turkey \\
  ICPR 2010 - 20th International Conference on Pattern Recognition
  \\\emph{I received the award for the best paper published in 2008 in Pattern Recognition journal}
\item 12-14 July 2010, Sheffield, United Kingdom \\
  UCM 2010 - Uncertainty in Computer Models 2010 conference
\item 5-9 July 2010, Glasgow, United Kingdom \\
  IWSM 2010 - 25th International Workshop on Statistical Modelling
\item 3-8 June 2010, Benidorm, Spain \\
  Ninth Valencia International Meeting on Bayesian Statistics - 2010 World Meeting of the International Society for Bayesian Analysis.
  \\Poster presentation: \emph{Inference for Gaussian Process Emulation of Oil Reservoir Simulation Codes}.
\item 6-7 April 2010, Warwick, United Kingdom \\
  WOGAS2 - Workshop on Geometric and Algebraic Statistics 2.
\item 3-5 March 2010, Edinburgh, United Kingdom \\
  Mixture estimation and applications.
  \\Poster presentation: \emph{Information Theoretic Novelty Detection for Mixtures of Gaussians}.
\item 7-9 September 2009, Sheffield, United Kingdom \\
  PRIB 2009: Pattern Recognition in Bioinformatics 2009.
\item 20-21 May 2009, Swansea, United Kingdom \\
  NCAF Meeting: Neural Computing and Applications. Special Theme - Grand Challenges in Information-Driven Healthcare.
\item 15-19 December 2008, Pisa, Italy \\
  ICDM 2008: IEEE International Conference on Data Mining.
  \\Oral presentation: \emph{Detecting Suspicious Behavior in Surveillance Images}.
\item 8-13 December 2008, Vancouver, BC, Canada \\
  NIPS 2008: Neural Information Processing Systems Conference.
\item 9-10 September 2008, Sheffield, United Kingdom \\
  NCAF Meeting: Neural Computing and Applications. Special Theme - Dynamics and Dynamic Systems.
  \\Oral presentation: \emph{Information Theoretic Novelty Detection}.
\item 18-19 June 2008, Oxford, United Kingdom \\
  NCAF Meeting: Neural Computing and Applications. Special Theme - Signal Processing and Biomedical Applications.
\item 31 March - 2 April 2008, Sheffield, United Kingdom \\
  Data Modelling Workshops \& Symposium.
\item 13-17 August 2007, Orlando, FL - USA \\
  IJCNN 2007 - International Joint Conferences on Neural Networks.
  \\Poster presentation: \emph{Local learning of tide level time series using a fuzzy approach}.
\item 17 August 2007, Orlando, FL - USA \\
  CI-BIO 2007 - Post-Conference Workshop on Computational Intelligence Approaches for the Analysis of Bioinformatics Data.
  \\Oral presentation of the paper: \emph{Aggregating Memberships in Possibilistic Biclustering}.
\item 27-29 September 2006, Ancona - Italy \\
  CISI 06 Conferenza Italiana Sistemi Intelligenti.
  \\Poster presentation: \emph{Input Selection with Mixed Data Sets: A Simulated Annealing Wrapper Approach}.
\item 15-16 September 2006, Genova - Italy \\
  BioLeMiD 06 - Third Bioinformatics Meeting on Machine Learning for Microarray Studies of Disease: biomarker selection.
\item 29-31 August 2006, Genova - Italy
  FLINS 2006 - 7th International FLINS Conference on Applied Artificial Intelligence.
\item 28 June 2006, Genova - Italy \\
  Workshop on Trends in Computational Sciences.
\item 22 June 2006, Genova - Italy \\
  The I LIMBS day - A free one-day workshop about intelligence.
\item 21 June 2006, Genova - Italy \\
  Second workshop on Analytic Methods for Learning Theory: Learning, Regularization and Approximation
\item 15-17 September 2005, Crema - Italy \\
  WILF 05 - International Workshop on Fuzzy Logic and Applications.
  \\Oral presentation of the paper: \emph{Unsupervised gene selection and clustering using simulated annealing}.
\item 16-17 June 2005, Genova - Italy \\
  CLIP 2005 - Workshop on Cross-language information processing.
  \\Oral presentation of the paper: \emph{Soft rank clustering}.
\item 8-11 June 2005, Vietri sul Mare - Italy \\
  WIRN 05 - XVI Italian Workshop on Neural Networks.
  \\Oral presentation of the paper: \emph{Soft rank clustering}.
\item 14-17 September 2004, Perugia - Italy \\
  WIRN 04 - XV Italian Workshop on Neural Networks.
  \\Poster presentation: \emph{ERAF: a R package for regression and forecasting}.
\end{itemize}

\subsection*{Talks}
\begin{itemize}
\item 20 October, 2011, University of Glasgow - \emph{Inference in hierarchical models using stochastic approximations}.
\item 08 February, 2011, University College London (CSML seminars series) - \emph{Calibration of Oil Reservoir Simulation Codes}.
\item 24 January, 2011, University College London - \emph{Classification of fMRI data using latent Gaussian models}.
\item 12 November, 2010, University of Glasgow - \emph{Information Theoretic Novelty Detection}.
\item 13 October, 2010, Royal Statistical Society - \emph{Discussion of the paper ``Riemann manifold Langevin and Hamiltonian Monte Carlo methods'' by M. Girolami and B. Calderhead}.
\item 26 March, 2010, Liverpool John Moores University - \emph{Information Theoretic Novelty Detection}.
\item 11 November, 2009, University of Edinburgh - \emph{Information Theoretic Novelty Detection}.
\item 21 October, 2009, University of Sheffield - Tutorial presentation for the Speech and Hearing group - \emph{The probabilistic approach in data modeling}.
\item 14 July, 2009, Columbia University - \emph{Information Theoretic Novelty Detection}.
\item 21 January, 2009, University of Glasgow - \emph{Information Theoretic Novelty Detection}.
\item 22 December, 2008, University of Genova - \emph{Information Theoretic Novelty Detection}.
\item 3 March, 2008, University of Milano - \emph{Central Clustering in Kernel-Induced Spaces}.
\item 27 February, 2008, University of Napoli Parthenope - \emph{Central Clustering in Kernel-Induced Spaces}.
\item 27 September, 2007, George Mason University - \emph{Kernel and Spectral Methods for Clustering}.
\item 15 November 2006, University of Genova - \emph{Kernel and Spectral Methods for Clustering}.
\item 21 March 2006, University of Genova - \emph{Spectral Approach to Clustering}.
\item 20 December 2005, University of Genova - \emph{Subsequence Matching for Time Series Forecasting}.
\end{itemize}


\section*{Teaching Activity}

% \subsection*{Exam boards}
% \begin{itemize}
% \item 2005 and 2006: \\
%   Neural Networks and Soft Computing - University of Genova, \\ Operating Systems - University of Pisa
% \end{itemize}

%\begin{itemize}
%\item 2006:
%  \begin{itemize}
%  \item Neural Networks \\
%    IV / V year, Master's Degree in Computer Science - University of Genova \\
%    Lecturer: Stefano Rovetta
%  \item Soft Computing \\
%    III year, Master's Degree in Computer Science - University of Genova \\
%    Lecturer: Francesco Masulli
%  \item Operating Systems \\
%    II year, Master's Degree in Applied Computer Science - Polo di La Spezia - University of Pisa \\
%    Lecturer: Francesco Masulli
%  \end{itemize}
%\item 2005:
%  \begin{itemize}
%  \item Neural Networks \\
%    IV / V year, Master's Degree in Computer Science - University of Genova \\
%    Lecturer: Francesco Masulli
%  \item Soft Computing \\
%    III year, Master's Degree in Computer Science - University of Genova \\
%    Lecturer: Francesco Masulli
%  \item Operating Systems \\
%    II year, Master's Degree in Applied Computer Science - Polo di La Spezia - University of Pisa \\
%    Lecturer: Francesco Masulli
%  \end{itemize}
%\end{itemize}

% \subsection*{Teaching}
\begin{itemize}
\item 09-2013 - 12-2013 \\
  Lecturer (30 hours)
  Artificial Intelligence \\
  Undergraduate Degree Programme - Year 4
\item 01-2013 - 04-2013 \\
  Lecturer (30 hours)
  Machine Learning \\
  Undergraduate/Postgraduate Degree Programmes - Years 4/5
\item 09-2012 - 12-2012 \\
  Lecturer (30 hours)
  Artificial Intelligence \\
  Undergraduate Degree Programme - Year 4
\item 10-2008 \\
  Lab assistant (2 hours)
  Bioinformatics \\
  Computational Biology module for the MSC in Biological and Bioprocess Engineering \\
  Lecturer: Guido Sanguinetti
\item 09-2005 - 12-2005 \\
  Lab assistant (30 hours)
  Neural Networks \\
  IV / V year, Master's Degree in Computer Science - University of Genova \\
  Lecturer: Stefano Rovetta
\item 09-2005 - 12-2005 \\
  Lab assistant (10 hours)
  Soft Computing \\
  III year, Master's Degree in Computer Science - University of Genova \\
  Lecturer: Francesco Masulli
\item 09-2004 - 12-2004 \\
  Lab assistant (10 hours) \\
  Introduction to R language (2 hours)
  Neural Networks \\
  IV / V year, Master's Degree in Computer Science - University of Genova \\
  Lecturer: Francesco Masulli
\end{itemize}

% \subsection*{Other Teaching Experiences}
% \begin{itemize}
% \item 09-2005 - 06-2006 \\
%   Teacher of Mathematics (120 hours) \\
%   Teacher of Computer Science (80 hours) \\  
%   IAL LIGURIA - High School (age 14) - Commercial Operator program \\
%   email: informazioni@ial.liguria.it, segreteria@ial.liguria.it
% \end{itemize}

%\newpage


% \section*{Personal Skills}

% \subsection*{Languages}
% \begin{itemize}
% \item Mother tongue: Italian
% \item English:
%   \begin{itemize}
% %    \item Reading ability: good
% %    \item Writing ability: good
% %    \item Speaking ability: good
%     \item From 05-2007 to 10-2007 I attended the grammar, conversation, and pronunciation courses for non-native speakers (60 hours)
%       at George Mason University, Fairfax, VA 22030 - USA
%     \item From 20-07-2004 to 05-08-2004 I attended an intermediate level course (30 hours)
%       at Byron School, 79, Hills Road - CB2 1PG Cambridge 
%   \end{itemize}
% \end{itemize}

% \subsection*{Computing Skills}
% \begin{description}
% \item[Operating systems:] Windows, Unix and Linux.
% \item[Programming languages:] R, C, C++, Fortran and Assembler languages.
% \item[Scripting:]  Perl language.
% \item[Web:] HTML and PHP languages.
% \item[Databases:] SQL language.
% \item[Editing:] Latex and Word.
% \end{description}

\end{document}
