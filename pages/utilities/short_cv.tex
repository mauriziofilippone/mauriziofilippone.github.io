\documentclass[oneside, a4paper, onecolumn, 10pt]{article}

%% \usepackage{fontspec}
%% \setmainfont{arial}

%%% PACKAGES %%%

%% \usepackage{aistats2021}
% If your paper is accepted, change the options for the package
% aistats2020 as follows:
%
% \usepackage[accepted]{aistats2020}
%
% This option will print headings for the title of your paper and
% headings for the authors names, plus a copyright note at the end of
% the first column of the first page.

% If you set papersize explicitly, activate the following three lines:
%\special{papersize = 8.5in, 11in}
%\setlength{\pdfpageheight}{11in}
%\setlength{\pdfpagewidth}{8.5in}


% If you use natbib package, activate the following three lines: 
\usepackage[round]{natbib}
\renewcommand{\bibname}{References}
\renewcommand{\bibsection}{\subsubsection*{\bibname}}

% Acronyms
\usepackage{scalefnt,letltxmacro}
\LetLtxMacro{\oldtextsc}{\textsc}
\renewcommand{\textsc}[1]{\oldtextsc{\scalefont{1.10}#1}}
\usepackage[acronym,smallcaps,nowarn]{glossaries}
\glsdisablehyper
\makeglossaries
\usepackage{xspace}
\usepackage{float}
\usepackage{physics}

% Maths
\usepackage{amssymb}
\usepackage{mathtools}
\usepackage{amsfonts}
\usepackage{amsmath}
\usepackage{amsthm}
\usepackage{booktabs}
\usepackage[ruled,vlined]{algorithm2e}
\usepackage[]{microtype}


% color
\usepackage[usenames,dvipsnames]{xcolor}
\definecolor{shadecolor}{gray}{0.9}
\newcommand{\red}[1]{\textcolor{BrickRed}{#1}}
\newcommand{\orange}[1]{\textcolor{BurntOrange}{#1}}
\newcommand{\green}[1]{\textcolor{OliveGreen}{#1}}
\newcommand{\blue}[1]{\textcolor{MidnightBlue}{#1}}
\newcommand{\sky}[1]{\textcolor{SkyBlue}{#1}}
\newcommand{\gray}[1]{\textcolor{black!60}{#1}}

% hyperref
\usepackage[colorlinks,linktoc=all]{hyperref}
\usepackage[all]{hypcap}
\hypersetup{citecolor=red}
\hypersetup{linkcolor=red}
\hypersetup{urlcolor=red}
%% \hypersetup{citecolor=MidnightBlue}
%% \hypersetup{linkcolor=MidnightBlue}
%% \hypersetup{urlcolor=MidnightBlue}



% cleverref
\usepackage[capitalize,nameinlink]{cleveref}
\crefname{section}{\S}{\S\S}
\Crefname{section}{\S}{\S\S}
\creflabelformat{equation}{#2\textup{#1}#3}
\providecommand\algorithmname{algorithm}


% Other           
\DeclareRobustCommand{\parhead}[1]{\textbf{#1}~}

% For crossferencing across documents
\makeatletter
\newcommand*{\addFileDependency}[1]{% argument=file name and extension
	\typeout{(#1)}
	\@addtofilelist{#1}
	\IfFileExists{#1}{}{\typeout{No file #1.}}
}
\makeatother
\newcommand*{\myexternaldocument}[1]{%
	\externaldocument{#1}%
	\addFileDependency{#1.tex}%
	\addFileDependency{#1.aux}%
}
% Figures
\usepackage{nidanfloat}
\usepackage[font=small,labelfont=bf,tableposition=top]{caption}
\usepackage{tikz}
\usepackage{graphicx}
\usepackage[format=hang]{subcaption}
\usepackage{wrapfig}

\usepackage{pgfplots}
\pgfplotsset{compat=1.6}
\usepgfplotslibrary{groupplots}
%% Use sans serif font for pictures
\tikzstyle{every picture}+=[font=\sffamily]
\tikzstyle{optimized} = [circle,fill=white,draw=black, dashed,inner sep=1pt, minimum size=20pt, font=\fontsize{10}{10}\selectfont, node distance=1]
\pgfkeys{/pgf/number format/.cd,1000 sep={}}
\pgfplotsset{
	tick label style = {font=\sffamily},
	every axis label/.append style={font=\sffamily},
	typeset ticklabels with strut,
}
\pgfplotsset{every axis/.append style={
			every x tick label/.append style={font=\fontsize{3.5pt}{3pt}\sffamily, yshift=.5ex,},
			every y tick label/.append style={font=\fontsize{3.5pt}{3pt}\sffamily, xshift=.8ex},
			every y label/.append style={xshift=10ex}
		},
}
\newlength\figureheight
\newlength\figurewidth
\usepgfplotslibrary{external}
\tikzexternalize[mode=list and make]
% \tikzexternalize
\newcommand{\tikzfile}[1]{
	\tikzsetnextfilename{#1}%  
	\input{#1.tikz}
}



% ********************************************************************
% Notes and todos
% ********************************************************************
\usepackage[colorinlistoftodos,
	% disable,
	textsize=scriptsize,
	linecolor=red!30,
	bordercolor=red!30,
	backgroundcolor=red!10]{todonotes}
% \makeatletter
\renewcommand{\todo}[2][]{\tikzexternaldisable\@todo[#1]{#2}\tikzexternalenable}
% \makeatother
\newcommand{\note}[1]{\todo[linecolor=Plum,backgroundcolor=Plum!25,bordercolor=Plum, inline, ]{\color{Plum!80!black}{\textbf{\textit{#1}}}}}


\usepackage[most]{tcolorbox}
\newenvironment{highlight}%
{\begin{tcolorbox}[toprule=2mm,left=4pt,right=4pt,top=0pt,bottom=1pt,boxsep=2pt, before skip = 2ex, after skip =2ex]}{\end{tcolorbox}}
\tcbset{boxsep=4pt,left=2pt,right=2pt,top=-0pt,bottom=0pt}


\newcommand\noteMF[1]{{\color{red}{ \footnotesize {\textbf{\textsc{MF}} - ``#1''}\xspace}}}
\newcommand\notePM[1]{{\color{blue}{ \footnotesize {\textbf{\textsc{PM}} - ``#1''}\xspace}}}
\newcommand\noteDM[1]{{\color{magenta}{ \footnotesize {\textbf{\textsc{DM}} - ``#1''}\xspace}}}
\newcommand\noteBH[1]{{\color{violet}{ \footnotesize {\textbf{\textsc{BH}} - ``#1''}\xspace}}}



\usepackage{url}

\usepackage{xr}
\externaldocument{supplementary}

%\usepackage{algorithm}
%\usepackage{algorithmic}
\usepackage{subcaption}

\usepackage{tikz}
\usepackage{graphicx}
\usepackage{xcolor}
\usepackage{amsmath}
\usepackage{amssymb}
\usepackage{bm}
% \usepackage{amsthm} % theorems and proofs
% \newtheorem{theorem}{Theorem}
% \newtheorem{proposition}{Proposition}

\usepackage{adjustbox}
\usepackage{multirow}
\usepackage{mathtools}
%\DeclareMathOperator{\Hessian}{H}
%\DeclareMathOperator{\Hessian}{\mathbf{H}}
\DeclareMathOperator{\Hessian}{Hess}

\usepackage{xspace}

%\hypersetup{draft}
\newcommand{\redquestionmark}{\textbf{\textcolor{red!60!black}{?}}}





%%%%%%%%%%%%%%%%%%%%%%%%%%%%%%%%%%%%%%%%%%%%%%%%%%%%%%%%%%%%%
% ======= DEBUG FOR MISSING REFERENCES (Uncomment if needed)
%%%%%%%%%%%%%%%%%%%%%%%%%%%%%%%%%%%%%%%%%%%%%%%%%%%%%%%%%%%%%

% \makeatletter
% \let\oldcitation\citation
% \def\xcomma{}
% \def\bibList{}
% \def\citation#1{%
%     \@for\xx:=#1\do{%
%       \expandafter\ifx\csname bbl@\xx\endcsname\relax%
%         \expandafter\gdef\csname bbl@\xx\endcsname{0}%
%         \xdef\bibList{\bibList\xcomma\xx}%
%         \def\xcomma{,}%
%       \fi%
%     }%
%     \oldcitation{#1}%
% }
% \AtEndDocument{\@bsphack
%   \protected@write\@auxout{}%
%          {\string\BibList{\bibList}}%
%   \@esphack}
% \def\xComma{}
% \def\biblostkeys{}
% \def\BibList#1{%
%    \@for\xx:=#1\do{%
%       \expandafter\ifx\csname BBL@\xx\endcsname\relax%
%         \xdef\biblostkeys{\biblostkeys\xComma\xx}%
%         \def\xComma{,\space}%
%       \fi%
%    }%
% }
% \def\refnotcalled{}
% \def\Xcomma{}
% \let\oldbibcite\bibcite
% \def\bibcite#1{%
%    \expandafter\gdef\csname BBL@#1\endcsname{0}%
%    \expandafter\ifx\csname bbl@#1\endcsname\relax% Not cited
%      \xdef\refnotcalled{\refnotcalled\Xcomma#1}%
%      \def\Xcomma{,\space}%
%    \fi%
%    \oldbibcite{#1}%
% }
% \def\NAT@citex%
%   [#1][#2]#3{%
%   \NAT@reset@parser
%   \NAT@sort@cites{#3}%
%   \NAT@reset@citea
%   \@cite{\let\NAT@nm\@empty\let\NAT@year\@empty
%     \@for\@citeb:=\NAT@cite@list\do
%     {\@safe@activestrue
%      \edef\@citeb{\expandafter\@firstofone\@citeb\@empty}%
%      \@safe@activesfalse
%      \@ifundefined{b@\@citeb\@extra@b@citeb}{\@citea%
%        {\reset@font\bfseries?\@citeb?}\NAT@citeundefined
%                  \PackageWarning{natbib}%
%        {Citation `\@citeb' on page \thepage \space undefined}\def\NAT@date{}}%
%      {\let\NAT@last@nm=\NAT@nm\let\NAT@last@yr=\NAT@year
%       \NAT@parse{\@citeb}%
%       \ifNAT@longnames\@ifundefined{bv@\@citeb\@extra@b@citeb}{%
%         \let\NAT@name=\NAT@all@names
%         \global\@namedef{bv@\@citeb\@extra@b@citeb}{}}{}%
%       \fi
%      \ifNAT@full\let\NAT@nm\NAT@all@names\else
%        \let\NAT@nm\NAT@name\fi
%      \ifNAT@swa\ifcase\NAT@ctype
%        \if\relax\NAT@date\relax
%          \@citea\NAT@hyper@{\NAT@nmfmt{\NAT@nm}\NAT@date}%
%        \else
%          \ifx\NAT@last@nm\NAT@nm\NAT@yrsep
%             \ifx\NAT@last@yr\NAT@year
%               \def\NAT@temp{{?}}%
%               \ifx\NAT@temp\NAT@exlab\PackageWarningNoLine{natbib}%
%                {Multiple citation on page \thepage: same authors and
%                year\MessageBreak without distinguishing extra
%                letter,\MessageBreak appears as question mark}\fi
%               \NAT@hyper@{\NAT@exlab}%
%             \else\unskip\NAT@spacechar
%               \NAT@hyper@{\NAT@date}%
%             \fi
%          \else
%            \@citea\NAT@hyper@{%
%              \NAT@nmfmt{\NAT@nm}%
%              \hyper@natlinkbreak{%
%                \NAT@aysep\NAT@spacechar}{\@citeb\@extra@b@citeb
%              }%
%              \NAT@date
%            }%
%          \fi
%        \fi
%      \or\@citea\NAT@hyper@{\NAT@nmfmt{\NAT@nm}}%
%      \or\@citea\NAT@hyper@{\NAT@date}%
%      \or\@citea\NAT@hyper@{\NAT@alias}%
%      \fi \NAT@def@citea
%      \else
%        \ifcase\NAT@ctype
%         \if\relax\NAT@date\relax
%           \@citea\NAT@hyper@{\NAT@nmfmt{\NAT@nm}}%
%         \else
%          \ifx\NAT@last@nm\NAT@nm\NAT@yrsep
%             \ifx\NAT@last@yr\NAT@year
%               \def\NAT@temp{{?}}%
%               \ifx\NAT@temp\NAT@exlab\PackageWarningNoLine{natbib}%
%                {Multiple citation on page \thepage: same authors and
%                year\MessageBreak without distinguishing extra
%                letter,\MessageBreak appears as question mark}\fi
%               \NAT@hyper@{\NAT@exlab}%
%             \else
%               \unskip\NAT@spacechar
%               \NAT@hyper@{\NAT@date}%
%             \fi
%          \else
%            \@citea\NAT@hyper@{%
%              \NAT@nmfmt{\NAT@nm}%
%              \hyper@natlinkbreak{\NAT@spacechar\NAT@@open\if*#1*\else#1\NAT@spacechar\fi}%
%                {\@citeb\@extra@b@citeb}%
%              \NAT@date
%            }%
%          \fi
%         \fi
%        \or\@citea\NAT@hyper@{\NAT@nmfmt{\NAT@nm}}%
%        \or\@citea\NAT@hyper@{\NAT@date}%
%        \or\@citea\NAT@hyper@{\NAT@alias}%
%        \fi
%        \if\relax\NAT@date\relax
%          \NAT@def@citea
%        \else
%          \NAT@def@citea@close
%        \fi
%      \fi
%      }}\ifNAT@swa\else\if*#2*\else\NAT@cmt#2\fi
%      \if\relax\NAT@date\relax\else\NAT@@close\fi\fi}{#1}{#2}}

\newacronym{MAP}{map}{maximum-a-posteriori}
\newacronym{MLE}{mle}{maximum likelihood estimation}
\newacronym{MNLL}{mnll}{mean negative loglikelihood}
\newacronym{NLL}{nll}{negative loglikelihood}
\newacronym{RMSE}{rmse}{root mean square error}
\newacronym{ECE}{ece}{expected calibration error}

\newacronym{VAE}{vae}{variational autoencoder}

\newacronym{MC}{mc}{Monte Carlo}
\newacronym{MCMC}{mcmc}{Markov chain Monte Carlo}
\newacronym{HMC}{hmc}{Hamiltonian Monte Carlo}
\newacronym{MH}{mh}{Metropolis-Hastings}
\newacronym{NUTS}{nuts}{no-u-turn sampler}
\newacronym{SGHMC}{sghmc}{stochastic gradient Hamiltonian Monte Carlo}

% \newacronym{GP}{gp}{Gaussian process}
\newacronym{DGP}{dgp}{deep Gaussian process} % use glspl for plural
\newacronym{GPLVM}{gplvm}{Gaussian process latent variable model}

\newacronym{VFE}{vfe}{variational free energy}

\newacronym[firstplural=Gaussian Processes]{GP}{gp}{Gaussian Process}
%\newacronym[plural=dgp\textnormal{s}, firstplural=Deep Gaussian Processes]{DGP}{dgp}{Deep Gaussian Process}

\newacronym{VI}{vi}{variational inference}

\newacronym{ELBO}{elbo}{evidence lower bound}
\newacronym{NELBO}{nelbo}{negative evidence lower bound}
\newacronym{ELL}{ell}{expected log likelihood}
\newacronym{KL}{kl}{Kullback-Leibler divergence}
\newacronym{AUC}{auc}{area under the curve}

\newacronym[firstplural=Bayesian neural networks]{BNN}{bnn}{Bayesian neural network}
\newacronym[firstplural=deep neural networks]{DNN}{dnn}{deep neural network}
\newacronym[]{CNN}{cnn}{convolutional neural network}
\newacronym{MLP}{mlp}{multilayer perceptron}
\newacronym{NN}{nn}{neural network}
\newacronym{RELU}{ReLU}{rectified linear unit}

\newacronym{NF}{nf}{normalizing flow}

\newacronym{RBF}{rbf}{radial basis function}
\newacronym{ARD}{ard}{automatic relevance determination}

\newacronym{RKHS}{rkhs}{reproducing kernel Hilbert space}

\newcommand{\iid}{i.i.d~} 


\newcommand{\mathbold}[1]{\ensuremath{\boldsymbol{\mathbf{#1}}}}

% # PROBABILITY
\newcommand{\g}{\,|\,}
% \renewcommand{\gg}{\,\|\,}
\renewcommand{\d}[1]{\ensuremath{\operatorname{d}\!{#1}}}
\newcommand{\nestedmathbold}[1]{{\mathbold{#1}}}

% # BOLD MATHEMATICS

\newcommand{\mba}{\nestedmathbold{a}}
\newcommand{\mbb}{\nestedmathbold{b}}
\newcommand{\mbc}{\nestedmathbold{c}}
\newcommand{\mbd}{\nestedmathbold{d}}
\newcommand{\mbe}{\nestedmathbold{e}}
\newcommand{\mbf}{\nestedmathbold{f}}
\newcommand{\mbg}{\nestedmathbold{g}}
\newcommand{\mbh}{\nestedmathbold{h}}
\newcommand{\mbi}{\nestedmathbold{i}}
\newcommand{\mbj}{\nestedmathbold{j}}
\newcommand{\mbk}{\nestedmathbold{k}}
\newcommand{\mbl}{\nestedmathbold{l}}
\newcommand{\mbm}{\nestedmathbold{m}}
\newcommand{\mbn}{\nestedmathbold{n}}
\newcommand{\mbo}{\nestedmathbold{o}}
\newcommand{\mbp}{\nestedmathbold{p}}
\newcommand{\mbq}{\nestedmathbold{q}}
\newcommand{\mbr}{\nestedmathbold{r}}
\newcommand{\mbs}{\nestedmathbold{s}}
\newcommand{\mbt}{\nestedmathbold{t}}
\newcommand{\mbu}{\nestedmathbold{u}}
\newcommand{\mbv}{\nestedmathbold{v}}
\newcommand{\mbw}{\nestedmathbold{W}}
\newcommand{\mbx}{\nestedmathbold{x}}
\newcommand{\mby}{\nestedmathbold{y}}
\newcommand{\mbz}{\nestedmathbold{z}}

\newcommand{\mbA}{\nestedmathbold{A}}
\newcommand{\mbB}{\nestedmathbold{B}}
\newcommand{\mbC}{\nestedmathbold{C}}
\newcommand{\mbD}{\nestedmathbold{D}}
\newcommand{\mbE}{\nestedmathbold{E}}
\newcommand{\mbF}{\nestedmathbold{F}}
\newcommand{\mbG}{\nestedmathbold{G}}
\newcommand{\mbH}{\nestedmathbold{H}}
\newcommand{\mbI}{\nestedmathbold{I}}
\newcommand{\mbJ}{\nestedmathbold{J}}
\newcommand{\mbK}{\nestedmathbold{K}}
\newcommand{\mbL}{\nestedmathbold{L}}
\newcommand{\mbM}{\nestedmathbold{M}}
\newcommand{\mbN}{\nestedmathbold{N}}
\newcommand{\mbO}{\nestedmathbold{O}}
\newcommand{\mbP}{\nestedmathbold{P}}
\newcommand{\mbQ}{\nestedmathbold{Q}}
\newcommand{\mbR}{\nestedmathbold{R}}
\newcommand{\mbS}{\nestedmathbold{S}}
\newcommand{\mbT}{\nestedmathbold{T}}
\newcommand{\mbU}{\nestedmathbold{U}}
\newcommand{\mbV}{\nestedmathbold{V}}
\newcommand{\mbW}{\nestedmathbold{W}}
\newcommand{\mbX}{\nestedmathbold{X}}
\newcommand{\mbY}{\nestedmathbold{Y}}
\newcommand{\mbZ}{\nestedmathbold{Z}}

\newcommand{\mbalpha}{\nestedmathbold{\alpha}}
\newcommand{\mbbeta}{\nestedmathbold{\beta}}
\newcommand{\mbdelta}{\nestedmathbold{\delta}}
\newcommand{\mbepsilon}{\nestedmathbold{\epsilon}}
\newcommand{\mbchi}{\nestedmathbold{\chi}}
\newcommand{\mbeta}{\nestedmathbold{\eta}}
\newcommand{\mbgamma}{\nestedmathbold{\gamma}}
\newcommand{\mbiota}{\nestedmathbold{\iota}}
\newcommand{\mbkappa}{\nestedmathbold{\kappa}}
\newcommand{\mblambda}{\nestedmathbold{\lambda}}
\newcommand{\mbmu}{\nestedmathbold{\mu}}
\newcommand{\mbnu}{\nestedmathbold{\nu}}
\newcommand{\mbomega}{\nestedmathbold{\omega}}
\newcommand{\mbphi}{\nestedmathbold{\phi}}
\newcommand{\mbpi}{\nestedmathbold{\pi}}
\newcommand{\mbpsi}{\nestedmathbold{\psi}}
\newcommand{\mbrho}{\nestedmathbold{\rho}}
\newcommand{\mbsigma}{\nestedmathbold{\sigma}}
\newcommand{\mbtau}{\nestedmathbold{\tau}}
\newcommand{\mbtheta}{\nestedmathbold{\theta}}
\newcommand{\mbupsilon}{\nestedmathbold{\upsilon}}
\newcommand{\mbvarepsilon}{\nestedmathbold{\varepsilon}}
\newcommand{\mbvarphi}{\nestedmathbold{\varphi}}
\newcommand{\mbvartheta}{\nestedmathbold{\vartheta}}
\newcommand{\mbvarrho}{\nestedmathbold{\varrho}}
\newcommand{\mbxi}{\nestedmathbold{\xi}}
\newcommand{\mbzeta}{\nestedmathbold{\zeta}}

\newcommand{\mbDelta}{\nestedmathbold{\Delta}}
\newcommand{\mbGamma}{\nestedmathbold{\Gamma}}
\newcommand{\mbLambda}{\nestedmathbold{\Lambda}}
\newcommand{\mbOmega}{\nestedmathbold{\Omega}}
\newcommand{\mbPhi}{\nestedmathbold{\Phi}}
\newcommand{\mbPi}{\nestedmathbold{\Pi}}
\newcommand{\mbPsi}{\nestedmathbold{\Psi}}
\newcommand{\mbSigma}{\nestedmathbold{\Sigma}}
\newcommand{\mbTheta}{\nestedmathbold{\Theta}}
\newcommand{\mbUpsilon}{\nestedmathbold{\Upsilon}}
\newcommand{\mbXi}{\nestedmathbold{\Xi}}

\newcommand{\mbzero}{\nestedmathbold{0}}
\newcommand{\mbone}{\nestedmathbold{1}}
\newcommand{\mbtwo}{\nestedmathbold{2}}
\newcommand{\mbthree}{\nestedmathbold{3}}
\newcommand{\mbfour}{\nestedmathbold{4}}
\newcommand{\mbfive}{\nestedmathbold{5}}
\newcommand{\mbsix}{\nestedmathbold{6}}
\newcommand{\mbseven}{\nestedmathbold{7}}
\newcommand{\mbeight}{\nestedmathbold{8}}
\newcommand{\mbnine}{\nestedmathbold{9}}

% # MISCELLANEOUS
\newcommand{\Lelbo}{\cL_{\textsc{elbo}}}

\newcommand{\scH}{\textsc{h}}
\DeclareRobustCommand{\KL}[2]{\ensuremath{\textsc{kl}\left[#1\;\|\;#2\right]}}
\DeclarePairedDelimiterX{\infdivx}[2]{[}{]}{%
  #1\;\delimsize\|\;#2%
}
\newcommand{\KLnew}{\ensuremath{\textsc{kl}}\infdivx}

\DeclareRobustCommand{\Df}[2]{\ensuremath{\mathcal{D}_f\left[#1\;\|\;#2\right]}}

\DeclareMathOperator*{\argmax}{arg\,max}
\DeclareMathOperator*{\argmin}{arg\,min}
\newcommand\indep{\protect\mathpalette{\protect\independenT}{\perp}}
\def\independenT#1#2{\mathrel{\rlap{$#1#2$}\mkern2mu{#1#2}}}

\newcommand{\cD}{\mathcal{D}}
\newcommand{\cL}{\mathcal{L}}
\newcommand{\cN}{\mathcal{N}}
\newcommand{\cP}{\mathcal{P}}
\newcommand{\cQ}{\mathcal{Q}}
\newcommand{\cR}{\mathcal{R}}
\newcommand{\cF}{\mathcal{F}}
\newcommand{\cI}{\mathcal{I}}
\newcommand{\cT}{\mathcal{T}}
\newcommand{\cV}{\mathcal{V}}
\newcommand{\cE}{\mathcal{E}}
\newcommand{\cG}{\mathcal{G}}
\newcommand{\cX}{\mathcal{X}}
\newcommand{\cY}{\mathcal{Y}}
\newcommand{\cH}{\mathcal{H}}
\newcommand{\cW}{\mathcal{W}}

\newcommand{\E}{\mathbb{E}}
\newcommand{\bbH}{\mathbb{H}}
\newcommand{\bbR}{\mathbb{R}}
 

 % GP stuff
\newcommand{\bigO}{\mathcal{O}}
\newcommand{\kernel}{\kappa} 

% covariances
\newcommand{\Kxx}{\mbK_{\mathrm{xx}}}
\newcommand{\Kzz}{\mbK_{\mathrm{zz}}}
\newcommand{\Kxz}{\mbK_{\mathrm{xz}}}
\newcommand{\Kzx}{\mbK_{\mathrm{zx}}}
\newcommand{\Kzzinv}{\mbK_{\mathrm{zz}}^{-1}}

% distros
\newcommand{\Normal}{\cN}
\newcommand{\xstar}{\mbx_{\star}}
\newcommand{\defeq}{\stackrel{\text{\tiny def}}{=}}


% operations
\newcommand{\inv}{{-1}}
% \newcommand{\Tr}{\mathrm{Tr}}
\newcommand{\const}{\mathrm{const.}}
\newcommand{\diag}{\textrm{diag}}
\newcommand{\supp}{\textrm{supp}}
\newcommand{\sub}[1]{{\texttt{\textit{\scriptsize {#1}}}}}


\usepackage[framemethod=latex,everyline=true]{mdframed}

\usepackage[labelformat=empty]{caption}

\usepackage{wrapfig}

\usepackage[round]{natbib}
\renewcommand{\bibname}{}
\renewcommand{\bibsection}{\subsubsection*{\bibname}}

\usepackage[left=2.54cm,top=2.54cm,bottom=2.54cm,right=2.54cm]{geometry}
%% \usepackage{hyperref}
%% \usepackage[utf8]{inputenc}
%% \usepackage{graphicx} 		% Add graphics capabilities
%% \usepackage{amsmath}  		% Better maths support
%% \usepackage{natbib}	% bibliography style
%% \setlength{\bibsep}{0.0pt}
%% \usepackage{eurosym}

%% \usepackage[acronym,smallcaps,nowarn]{glossaries}
%% \glsdisablehyper
%% \makeglossaries

%% \newacronym{MAP}{map}{maximum-a-posteriori}
\newacronym{MLE}{mle}{maximum likelihood estimation}
\newacronym{MNLL}{mnll}{mean negative loglikelihood}
\newacronym{NLL}{nll}{negative loglikelihood}
\newacronym{RMSE}{rmse}{root mean square error}
\newacronym{ECE}{ece}{expected calibration error}

\newacronym{VAE}{vae}{variational autoencoder}

\newacronym{MC}{mc}{Monte Carlo}
\newacronym{MCMC}{mcmc}{Markov chain Monte Carlo}
\newacronym{HMC}{hmc}{Hamiltonian Monte Carlo}
\newacronym{MH}{mh}{Metropolis-Hastings}
\newacronym{NUTS}{nuts}{no-u-turn sampler}
\newacronym{SGHMC}{sghmc}{stochastic gradient Hamiltonian Monte Carlo}

% \newacronym{GP}{gp}{Gaussian process}
\newacronym{DGP}{dgp}{deep Gaussian process} % use glspl for plural
\newacronym{GPLVM}{gplvm}{Gaussian process latent variable model}

\newacronym{VFE}{vfe}{variational free energy}

\newacronym[firstplural=Gaussian Processes]{GP}{gp}{Gaussian Process}
%\newacronym[plural=dgp\textnormal{s}, firstplural=Deep Gaussian Processes]{DGP}{dgp}{Deep Gaussian Process}

\newacronym{VI}{vi}{variational inference}

\newacronym{ELBO}{elbo}{evidence lower bound}
\newacronym{NELBO}{nelbo}{negative evidence lower bound}
\newacronym{ELL}{ell}{expected log likelihood}
\newacronym{KL}{kl}{Kullback-Leibler divergence}
\newacronym{AUC}{auc}{area under the curve}

\newacronym[firstplural=Bayesian neural networks]{BNN}{bnn}{Bayesian neural network}
\newacronym[firstplural=deep neural networks]{DNN}{dnn}{deep neural network}
\newacronym[]{CNN}{cnn}{convolutional neural network}
\newacronym{MLP}{mlp}{multilayer perceptron}
\newacronym{NN}{nn}{neural network}
\newacronym{RELU}{ReLU}{rectified linear unit}

\newacronym{NF}{nf}{normalizing flow}

\newacronym{RBF}{rbf}{radial basis function}
\newacronym{ARD}{ard}{automatic relevance determination}

\newacronym{RKHS}{rkhs}{reproducing kernel Hilbert space}

\newcommand{\iid}{i.i.d~} 


%% \newcommand{\mathbold}[1]{\ensuremath{\boldsymbol{\mathbf{#1}}}}

% # PROBABILITY
\newcommand{\g}{\,|\,}
% \renewcommand{\gg}{\,\|\,}
\renewcommand{\d}[1]{\ensuremath{\operatorname{d}\!{#1}}}
\newcommand{\nestedmathbold}[1]{{\mathbold{#1}}}

% # BOLD MATHEMATICS

\newcommand{\mba}{\nestedmathbold{a}}
\newcommand{\mbb}{\nestedmathbold{b}}
\newcommand{\mbc}{\nestedmathbold{c}}
\newcommand{\mbd}{\nestedmathbold{d}}
\newcommand{\mbe}{\nestedmathbold{e}}
\newcommand{\mbf}{\nestedmathbold{f}}
\newcommand{\mbg}{\nestedmathbold{g}}
\newcommand{\mbh}{\nestedmathbold{h}}
\newcommand{\mbi}{\nestedmathbold{i}}
\newcommand{\mbj}{\nestedmathbold{j}}
\newcommand{\mbk}{\nestedmathbold{k}}
\newcommand{\mbl}{\nestedmathbold{l}}
\newcommand{\mbm}{\nestedmathbold{m}}
\newcommand{\mbn}{\nestedmathbold{n}}
\newcommand{\mbo}{\nestedmathbold{o}}
\newcommand{\mbp}{\nestedmathbold{p}}
\newcommand{\mbq}{\nestedmathbold{q}}
\newcommand{\mbr}{\nestedmathbold{r}}
\newcommand{\mbs}{\nestedmathbold{s}}
\newcommand{\mbt}{\nestedmathbold{t}}
\newcommand{\mbu}{\nestedmathbold{u}}
\newcommand{\mbv}{\nestedmathbold{v}}
\newcommand{\mbw}{\nestedmathbold{W}}
\newcommand{\mbx}{\nestedmathbold{x}}
\newcommand{\mby}{\nestedmathbold{y}}
\newcommand{\mbz}{\nestedmathbold{z}}

\newcommand{\mbA}{\nestedmathbold{A}}
\newcommand{\mbB}{\nestedmathbold{B}}
\newcommand{\mbC}{\nestedmathbold{C}}
\newcommand{\mbD}{\nestedmathbold{D}}
\newcommand{\mbE}{\nestedmathbold{E}}
\newcommand{\mbF}{\nestedmathbold{F}}
\newcommand{\mbG}{\nestedmathbold{G}}
\newcommand{\mbH}{\nestedmathbold{H}}
\newcommand{\mbI}{\nestedmathbold{I}}
\newcommand{\mbJ}{\nestedmathbold{J}}
\newcommand{\mbK}{\nestedmathbold{K}}
\newcommand{\mbL}{\nestedmathbold{L}}
\newcommand{\mbM}{\nestedmathbold{M}}
\newcommand{\mbN}{\nestedmathbold{N}}
\newcommand{\mbO}{\nestedmathbold{O}}
\newcommand{\mbP}{\nestedmathbold{P}}
\newcommand{\mbQ}{\nestedmathbold{Q}}
\newcommand{\mbR}{\nestedmathbold{R}}
\newcommand{\mbS}{\nestedmathbold{S}}
\newcommand{\mbT}{\nestedmathbold{T}}
\newcommand{\mbU}{\nestedmathbold{U}}
\newcommand{\mbV}{\nestedmathbold{V}}
\newcommand{\mbW}{\nestedmathbold{W}}
\newcommand{\mbX}{\nestedmathbold{X}}
\newcommand{\mbY}{\nestedmathbold{Y}}
\newcommand{\mbZ}{\nestedmathbold{Z}}

\newcommand{\mbalpha}{\nestedmathbold{\alpha}}
\newcommand{\mbbeta}{\nestedmathbold{\beta}}
\newcommand{\mbdelta}{\nestedmathbold{\delta}}
\newcommand{\mbepsilon}{\nestedmathbold{\epsilon}}
\newcommand{\mbchi}{\nestedmathbold{\chi}}
\newcommand{\mbeta}{\nestedmathbold{\eta}}
\newcommand{\mbgamma}{\nestedmathbold{\gamma}}
\newcommand{\mbiota}{\nestedmathbold{\iota}}
\newcommand{\mbkappa}{\nestedmathbold{\kappa}}
\newcommand{\mblambda}{\nestedmathbold{\lambda}}
\newcommand{\mbmu}{\nestedmathbold{\mu}}
\newcommand{\mbnu}{\nestedmathbold{\nu}}
\newcommand{\mbomega}{\nestedmathbold{\omega}}
\newcommand{\mbphi}{\nestedmathbold{\phi}}
\newcommand{\mbpi}{\nestedmathbold{\pi}}
\newcommand{\mbpsi}{\nestedmathbold{\psi}}
\newcommand{\mbrho}{\nestedmathbold{\rho}}
\newcommand{\mbsigma}{\nestedmathbold{\sigma}}
\newcommand{\mbtau}{\nestedmathbold{\tau}}
\newcommand{\mbtheta}{\nestedmathbold{\theta}}
\newcommand{\mbupsilon}{\nestedmathbold{\upsilon}}
\newcommand{\mbvarepsilon}{\nestedmathbold{\varepsilon}}
\newcommand{\mbvarphi}{\nestedmathbold{\varphi}}
\newcommand{\mbvartheta}{\nestedmathbold{\vartheta}}
\newcommand{\mbvarrho}{\nestedmathbold{\varrho}}
\newcommand{\mbxi}{\nestedmathbold{\xi}}
\newcommand{\mbzeta}{\nestedmathbold{\zeta}}

\newcommand{\mbDelta}{\nestedmathbold{\Delta}}
\newcommand{\mbGamma}{\nestedmathbold{\Gamma}}
\newcommand{\mbLambda}{\nestedmathbold{\Lambda}}
\newcommand{\mbOmega}{\nestedmathbold{\Omega}}
\newcommand{\mbPhi}{\nestedmathbold{\Phi}}
\newcommand{\mbPi}{\nestedmathbold{\Pi}}
\newcommand{\mbPsi}{\nestedmathbold{\Psi}}
\newcommand{\mbSigma}{\nestedmathbold{\Sigma}}
\newcommand{\mbTheta}{\nestedmathbold{\Theta}}
\newcommand{\mbUpsilon}{\nestedmathbold{\Upsilon}}
\newcommand{\mbXi}{\nestedmathbold{\Xi}}

\newcommand{\mbzero}{\nestedmathbold{0}}
\newcommand{\mbone}{\nestedmathbold{1}}
\newcommand{\mbtwo}{\nestedmathbold{2}}
\newcommand{\mbthree}{\nestedmathbold{3}}
\newcommand{\mbfour}{\nestedmathbold{4}}
\newcommand{\mbfive}{\nestedmathbold{5}}
\newcommand{\mbsix}{\nestedmathbold{6}}
\newcommand{\mbseven}{\nestedmathbold{7}}
\newcommand{\mbeight}{\nestedmathbold{8}}
\newcommand{\mbnine}{\nestedmathbold{9}}

% # MISCELLANEOUS
\newcommand{\Lelbo}{\cL_{\textsc{elbo}}}

\newcommand{\scH}{\textsc{h}}
\DeclareRobustCommand{\KL}[2]{\ensuremath{\textsc{kl}\left[#1\;\|\;#2\right]}}
\DeclarePairedDelimiterX{\infdivx}[2]{[}{]}{%
  #1\;\delimsize\|\;#2%
}
\newcommand{\KLnew}{\ensuremath{\textsc{kl}}\infdivx}

\DeclareRobustCommand{\Df}[2]{\ensuremath{\mathcal{D}_f\left[#1\;\|\;#2\right]}}

\DeclareMathOperator*{\argmax}{arg\,max}
\DeclareMathOperator*{\argmin}{arg\,min}
\newcommand\indep{\protect\mathpalette{\protect\independenT}{\perp}}
\def\independenT#1#2{\mathrel{\rlap{$#1#2$}\mkern2mu{#1#2}}}

\newcommand{\cD}{\mathcal{D}}
\newcommand{\cL}{\mathcal{L}}
\newcommand{\cN}{\mathcal{N}}
\newcommand{\cP}{\mathcal{P}}
\newcommand{\cQ}{\mathcal{Q}}
\newcommand{\cR}{\mathcal{R}}
\newcommand{\cF}{\mathcal{F}}
\newcommand{\cI}{\mathcal{I}}
\newcommand{\cT}{\mathcal{T}}
\newcommand{\cV}{\mathcal{V}}
\newcommand{\cE}{\mathcal{E}}
\newcommand{\cG}{\mathcal{G}}
\newcommand{\cX}{\mathcal{X}}
\newcommand{\cY}{\mathcal{Y}}
\newcommand{\cH}{\mathcal{H}}
\newcommand{\cW}{\mathcal{W}}

\newcommand{\E}{\mathbb{E}}
\newcommand{\bbH}{\mathbb{H}}
\newcommand{\bbR}{\mathbb{R}}
 

 % GP stuff
\newcommand{\bigO}{\mathcal{O}}
\newcommand{\kernel}{\kappa} 

% covariances
\newcommand{\Kxx}{\mbK_{\mathrm{xx}}}
\newcommand{\Kzz}{\mbK_{\mathrm{zz}}}
\newcommand{\Kxz}{\mbK_{\mathrm{xz}}}
\newcommand{\Kzx}{\mbK_{\mathrm{zx}}}
\newcommand{\Kzzinv}{\mbK_{\mathrm{zz}}^{-1}}

% distros
\newcommand{\Normal}{\cN}
\newcommand{\xstar}{\mbx_{\star}}
\newcommand{\defeq}{\stackrel{\text{\tiny def}}{=}}


% operations
\newcommand{\inv}{{-1}}
% \newcommand{\Tr}{\mathrm{Tr}}
\newcommand{\const}{\mathrm{const.}}
\newcommand{\diag}{\textrm{diag}}
\newcommand{\supp}{\textrm{supp}}
\newcommand{\sub}[1]{{\texttt{\textit{\scriptsize {#1}}}}}


%% \usepackage[capitalize,nameinlink]{cleveref}
%% \crefname{section}{\S}{\S\S}
%% \Crefname{section}{\S}{\S\S}
%% \creflabelformat{equation}{#2\textup{#1}#3}
%% \providecommand\algorithmname{algorithm}


%% \usepackage{bold-extra}


%% \usepackage{wrapfig}

%% \newif\ifdraftmode
%% % \draftmodetrue

%% \usepackage[framemethod=latex,everyline=true]{mdframed}

\usepackage{titlesec}

\definecolor{araorange}{rgb}{0.1, 0.1, 1.0}

\titleformat{\section}
{\color{MidnightBlue}\normalfont\large\bfseries}
{\color{MidnightBlue}\thesection}{1em}{}
\titleformat{\subsection}
{\color{MidnightBlue}\normalfont\bfseries}
{\color{MidnightBlue}\thesubsection}{1em}{}

% To fix list things: 
\usepackage{enumitem}
\setitemize{noitemsep,topsep=0pt,parsep=0pt,partopsep=0pt,leftmargin=*}
%% \usepackage{amssymb}
%% \renewcommand{\labelitemi}{\tiny$\blacksquare$}

%% \usepackage{nopageno}
%% \usepackage{enumitem}

%% \usepackage{color,colortbl}
%% \usepackage{xspace}

\definecolor{grey}{rgb}{0.9,0.9,0.9}

\newcommand{\mytitle}[0]{CURRICULUM VITAE}
\newcommand{\myacro}[0]{\name{prism}}

%% \input{notation.tex}

%% \usepackage{fancyhdr}
%% \pagestyle{fancy}
%% \renewcommand{\headrulewidth}{0pt} % Remove line at top

\usepackage{eurosym}

%% \input{notation.tex}

\usepackage{lipsum}

%% \renewcommand\thesection{\Alph{section}}
%% \renewcommand\thesubsection{\arabic{subsection}}
%% \renewcommand\thesubsubsection{\arabic{subsection}.\arabic{subsubsection}}

\newenvironment{itemize*}%
  {\begin{itemize}%
    \setlength{\itemsep}{0pt}%
    \setlength{\parskip}{0pt}}%
  {\end{itemize}}

\usepackage{enumitem}

%%% COMMANDS %%%

% Define the title, author and date of the document.
\title{\mytitle}
\author{Maurizio Filippone} %\\ Department of Data Science, EURECOM}

\begin{document}

\begin{center}
  {\Large {\color{MidnightBlue}{\bf \mytitle}}\par}
\end{center}

\begin{center}
  {{\color{MidnightBlue}{\bf Maurizio Filippone}}\par}
\end{center}


\linespread{0.9}
\selectfont

\begin{itemize}
%\item Nationality:        Italian
%% \item {\em AXA Chair of Computational Statistics \& Professor} at EURECOM, Sophia Antipolis, France
\item Email:              \texttt{maurizio.filippone@kaust.edu.sa}
\item Web Page:           \url{https://mauriziofilippone.github.io/}
\end{itemize}



\subsection*{Education}

I received a Master's degree in Physics and a Ph.D. in Computer Science from the University of Genoa, Italy, in 2004 and 2008, respectively.
The reason to pursue a Ph.D. in Computer Science was to deepen my interest in Computational Statistics and Machine Learning, which sparked towards the end of my studies in Physics.
I hold a ``Post-graduate certificate in academic practice'', which I obtained during my Lectureship at Glasgow.
I also hold the French abilitation ``Habilitation \'a diriger des recherches'', which is the highest degree in the French academic system to allow someone to become a Professor.

%% \begin{itemize}
%% \item Ph.D. in Computer Science - University of Genoa -- 2008 \\
%% Thesis title: Central Clustering in Kernel-Induced Spaces 
%% %% Keywords: kernel methods for clustering, spectral clustering, relational clustering.

%% \item Master's Degree in Physics % (full marks: 110/110) 
%% -- University of Genoa -- 2004
%% %% Thesis title: Ensemble methods for time series analysis and forecasting \\
%% %% Keywords: Nonlinear systems, regression, ensemble of learning machines, signal processing.
%% \end{itemize}
%% •	EDUCATION

%% 199? 	Ph.D.
%% 	Name of Faculty/ Department, Name of University/ Institution, Country
%% 	Name of Ph.D. Supervisor 
%% 199? 	Master
%% 	Name of Faculty/ Department, Name of University/ Institution, Country
	 

\subsection*{Academic Positions}
%% •	CURRENT POSITION(S)

%% 201? –  	Current Position
%% 	Name of Faculty/ Department, Name of University/ Institution/ Country
	
%% 200? – 	Current Position
%% 	Name of Faculty/ Department, Name of University/ Institution/ Country
                          
\begin{itemize}
\item Current position -- {\bf Associate Professor} -- % \\
  Statistics Program, KAUST, Saudi Arabia % \\

\item Fall 2018 to end of 2023 -- {\em Professeur Classe 2} -- % \\
  Department of Data Science, EURECOM, Biot, France % \\
  %% Keywords: Bayesian inference, Nonparametric modeling, Scalable inference

%% \vspace{3mm}

%% \item Current position -- {\em Associate Professor} -- % \\
%%   %% Department of Data Science, 
%% EURECOM, Sophia Antipolis, France % \\
%%   %% Keywords: Bayesian inference, Nonparametric modeling, Scalable inference

%% \vspace{3mm}

\item Fall 2015 to Fall 2018 -- {\em Ma\^{i}tre de Conf\'{e}rence Classe 1} -- % \\
%  Department of Data Science,
  EURECOM, Biot, France % \\
  %% Keywords: Bayesian inference, Nonparametric modeling, Scalable inference

\item Fall 2011 to Fall 2015 -- {\em Lecturer} -- % \\
  School of Computing Science -- University of Glasgow, UK % \\
  %% Keywords: Bayesian inference, Gaussian Processes, Markov chain Monte Carlo

\item Fall 2010 to Fall 2011 -- {\em Research Associate} (PI: Prof. M. Girolami) -- % \\
  University College London (2011) %, UK
  and % \\
  University of Glasgow (2010), UK % \\
%  Grant: The Synthesis of Probabilistic Prediction \& Mechanistic Modelling within a Computational \& Systems Biology Context (
  %% Keywords: Bayesian inference, Gaussian Processes, Markov chain Monte Carlo
  
\item Spring 2008 to Fall 2009 -- {\em Research Associate} (PI: Prof. G. Sanguinetti) -- % \\
  University of Sheffield, UK % \\
%  Grant: ALMS: Advanced Lifestyle Monitoring Systems 
  %% Keywords: novelty detection, statistical testing, Bayesian inference

\item Spring 2007 to Fall 2007 -- {\em Research Scholar} (PIs: Profs. D. Barbar\`a, C. Domeniconi) -- % \\
  %% Department of Information and Software Engineering --
  George Mason University, Fairfax VA, USA % \\
%  Grant: Detecting Suspicious Behavior in Reconnaissance Images 
  %% Keywords: outlier detection, density estimation, relational clustering
  
\end{itemize}


%% •	PREVIOUS POSITIONS

%% 200? – 200? 	Position held 
%% 	Name of Faculty/ Department, Name of University/ Institution/ Country
	
%% 200? – 200? 	Position held
%% 	Name of Faculty/ Department, Name of University/ Institution/ Country
	
\subsection*{Research contributions and impact}

My research interests are in the field of Bayesian Statistics. 
Over the last twelve years, I have focused on Bayesian models based on Gaussian processes, proposing a number of fundamental contributions to their applicability to large-scale problems.
Due to the advancements in the field, myself and other groups working in the domain have established ways to cast Gaussian processes and Deep Gaussian processes as Bayesian Neural Networks. 
As a result, my research is currently focusing on Bayesian Deep Learning, and techniques at the interface between Gaussian processes and Deep Neural Networks. 

After my Ph.D., I've published 100+ papers almost equally split between journals and conference, and I have been the leading author (either first or last) in most of these.
As of Feb 2025, I've received about 5000 citations and my $h$-index is 34 (source Google Scholar).

\subsection*{Selected Research Grants}
\begin{itemize}
  \item PI: \emph{MUSE-COM$^2$: AI-enabled MUltimodal SEmantic COMmunications and COMputing} (300K\euro), 2023--2026, ANR and EU (Chist-ERA call)
  \item PI: AXA Chair of Computational Statistics: \emph{New Computational Approaches to Risk Modeling} (600K\euro), 2016--2023, AXA Research Fund 
  \item PI: \emph{ECO-ML: Rethinking Modern Machine Learning Tools for a New Generation of Low-Power Large-Scale Modeling Systems} (300K\euro), 2018--2021, National French funding agency, JCJC research grant
  \item PI: 3IA Chair: \emph{Deep Probabilistic Modeling on Novel Hardware} ($\sim$200K\euro), 2020--2023, AI center funded by the National French funding agency ANR
  \item Partner: European Training Network: \emph{WindMill: Machine Learning for Wireless Communications} ($\sim$200K\euro), 2018--2022, EU Horizon 2020.
  \item Co-PI: \emph{Computational inference of biopathway dynamics and structures} (\pounds 340K), 2014--2017, (PI) D.~Husmeier and (Co-PI) S.~Rogers - EPSRC (UK) research grant 
\end{itemize}


%% •	FELLOWSHIPS 
%% 200? – 200? 	Scholarship, Name of Faculty/ Department/Centre, Name of University/ Institution/ Country 
%% 199? – 199? 	Scholarship, Name of Faculty/ Department/Centre, Name of University/ Institution/ Country

%% •	SUPERVISION OF GRADUATE STUDENTS AND POSTDOCTORAL FELLOWS
\subsection*{Supervision of Postdoctoral Fellows and Graduate Students}

%% \paragraph{Post-Doc Supervision:}
I've recently completed the supervision of two {\bf post-docs} at EURECOM, {\bf Simone Rossi} and {\bf Dimitrios Milios}, who were partially funded by two ongoing grants on fundamentals of Bayesian Statistics.
%% Previously, I hosted a visiting post-doc {\bf Roberto Visintainer} shortly after I moved to EURECOM in 2015.
%% Finally, 
Shortly after joining EURECOM, I supervised another post-doc {\bf Sebastien Marmin} on the same grants, and before joining EURECOM, I co-supervised {\bf Mu Niu} as a post-doc for three years at the University of Glasgow, funded by a grant from the UK research council EPSRC. 
%% Previously, I hosted a visiting post-doc {\bf Roberto Visintainer} shortly after I moved to EURECOM in 2015.
%% Finally, before joining EURECOM I co-supervised {\bf Mu Niu} as a post-doc for three years at the University of Glasgow, funded by a grant from the UK research council EPSRC. 
%% \begin{itemize}
%% \item Dr Sebastien Marmin: EURECOM. Fall 2017 - Fall 2020
%% \item Dr Dimitrios Milios: EURECOM. Fall 2017 - Fall 2020
%% \item Dr Mu Niu: School of Mathematics and Statistics, University of Glasgow. Fall 2014 - Fall 2017
%% \item Dr Roberto Visintainer: Fondazione Bruno Kessler, Trento. Visiting: Fall 2015 - Spring 2016
%% \end{itemize}

%% \paragraph{Ph.D. Supervision:}
I have recently completed the supervision of three {\bf Ph.D. students} at EURECOM, {\bf Bogdan Kozyrskiy} (ended in Spring 2023), and {\bf Ba-Hien Tran} (ended in Fall 2023) who were partially funded by two ongoing grants on fundamentals of Bayesian Statistics; the third Ph.D. student {\bf Davit Gogolashvili} was funded by European training network on Machine Learning for Wireless Communication Networks (EU grant WindMILL).
In the previous couple of years, five Ph.D. students under my supervision at EURECOM, {\bf Jonas Wacker}, {\bf Kurt Cutajar}, {\bf Gia-Lac Tran}, {\bf Simone Rossi}, and {\bf Remi Domingues}, successfully defended their theses, the first four on fundamentals of Bayesian Statistics, and the last one on statistical methods for fraud detection in collaboration with the company Amadeus. % in Sophia Antipolis, France. 
%% \paragraph*{Ph.D. Co-Supervision:}
I am helping my colleague {\bf Pietro Michiardi} to co-supervise one post-doc {\bf Giulio Franzese} on representation learning for networking data.
I've also co-supervised Ph.D. students funded by industry with Amadeus on time series ({\bf Rosa Candela}) and with SAP on interpretability ({\bf Graziano Mita}). % , and with Renault Software Labs on representation and reinforcement learning for vehicular technologies ({\bf Ugo Lecerf} and {\bf Matthieu Da Silva Filarder}).
%% , I'm co-supervising {\bf Rosa Candela} on time-series with Amadeus, {\bf Graziano Mita} on interpretable \ml with SAP, and {\bf Ugo Lecerf} and {\bf Matthieu Da Silva Filarder} on machine learning for vehicular technologies with Renault Software Labs.
Prior to joining EURECOM, I supervised a self-funded Ph.D. student {\bf Xiaoyu Xiong} at the University of Glasgow.

At KAUST, I'm currently supervising two Ph.D. students ({\bf Mattia Rosso} and {\bf Madi Matymov}) and one Ms/Ph.D. student ({\bf Yiting Lu}), and co-supervising one post-doc ({\bf Emmanuel Ambriz}).


%% \begin{itemize}
%% \item Jonas Wacker: Dept. of Data Science, EURECOM. Spring 2019 - Spring 2022
%% \item Simone Rossi: Dept. of Data Science, EURECOM. Spring 2018 - Spring 2021
%% \item Gia-Lac Tran: Dept. of Data Science, EURECOM. Fall 2017 - Fall 2020
%% \item Kurt Cutajar: Dept. of Data Science, EURECOM. Fall 2015 - Spring 2019
%% \item Remi Domingues: Dept. of Data Science, EURECOM and Amadeus. Spring 2016 - Spring 2019
%% \item Xiaoyu Xiong: School of Computing Science, University of Glasgow. Fall 2013 - Spring 2017
%% \end{itemize}

%% \subsubsection*{Ph.D. Co-Supervision}
%% \begin{itemize}
%% \item Rosa Candela: Dept. of Data Science, EURECOM and Amadeus. Spring 2018 - Spring 2021
%% \item Graziano Mita: Dept. of Data Science, EURECOM and SAP. Fall 2017 - Fall 2020
%% \end{itemize}

%% 200? – 200? 	Number of Postdocs/ Ph.D./ Master Students
%% Name of Faculty/ Department/ Centre, Name of University/ Institution/ Country


\subsection*{Teaching Activities}
%% •	TEACHING ACTIVITIES (if applicable) 

I started teaching when I joined the University of Glasgow as a lecturer in 2011, where I taught under-graduate and post-graduate courses in {\bf Algorithmic Foundations} and {\bf Machine Learning}.
In Glasgow, I also designed and created the material of a new course in {\bf Artificial Intelligence}. 
Since joining EURECOM, I have been giving lectures on Bayesian Statistics in a course named {\bf Advanced Statistical Inference}.
At KAUST I am teaching a new course on {\bf Bayesian Deep Learning} and one on {\bf Bayesian Statistics}.
Between 2018 and 2019, I delivered lectures at the MLCC {\bf summer school} in Genoa, Italy, and I designed and created the material for a {\bf tutorial} on Gaussian processes at the IJCNN 2019 conference in collaboration with E.~V.~Bonilla, and another one on Bayesian Deep Learning at the IJCAI 2021 conference in collaboration with my (at the time) Ph.D. student S.~Rossi. 

%% \begin{itemize}
%% \item Spring 2016--2018 - Lecturer (42 h) 
%%   Advanced Statistical Inference (MSc) - EURECOM
%% \item Spring 2013--2015 - Lecturer (30 h) 
%%   Machine Learning (Year 4) - University of Glasgow
%% \item Fall 2012--2014 - Lecturer (30 h) 
%%   Artificial Intelligence (Year 4) - University of Glasgow
%% \item Fall 2014 - Lecturer (30 h) 
%%   Algorithmic Foundations (Year 2) - University of Glasgow
%% \end{itemize}


%% 200? – 	Teaching position – Topic, Name of University/ Institution/ Country
%% 200? – 200? 	Teaching position – Topic, Name of University/ Institution/ Country



\subsection*{Service to the Scientific Community}

I am and I have been a {\bf Program committee} member for several international conferences. 
Here is a selection including the most prestigious ones:
  NeurIPS (2014--2019),
  ICML (2015--2020),
  ECML (2016--2017),
  AISTATS (2012--2013, 2016--2023), 
  IJCAI (2016),
  IJCNN (2006--2010, 2015).
I'm acting as an {\bf Area Chair} for AISTATS since 2020 and as a {\bf Senior Area Chair} for AISTATS since 2023, and I was {\bf Guest Editor} for the ECML/PKDD Machine Learning Journal track in 2020.
Between 2013 and 2016 I served as an {\bf Associate Editor} for the journals Pattern Recognition and the IEEE Transactions on Neural Networks and Learning Systems.

%% \begin{itemize}
%% \item {\em Area Chair} for AISTATS 2020
%% \item {\em Guest Editor} of ECML/PKDD journal track, Machine Learning Journal, 2020
%% \item {\em Program committee} member of
%%   NuerIPS (2014--2019),
%%   ICML (2015--2019),
%%   ECML (2016--2017),
%%   AISTATS (2012, 2013, 2016--2019), 
%%   IJCAI (2016),
%%   IJCNN (2006--2010, 2015),
%% \item {\em Associate Editor}: Pattern Recognition (end 2012 - end 2016)
%% \item {\em Associate Editor}: IEEE Transactions on Neural Networks and Learning Systems (2013 - end 2016)
%% \item {\em Technical Program Chair} for IJCNN 2014
%% \end{itemize}

%% •	ORGANISATION OF SCIENTIFIC MEETINGS (if applicable)
%% \subsection*{Organisation of Scientific Meetings}

%% 201?	Please specify your role and the name of event / Country 
%% 200? 	Please specify type of event / number of participants / Country


%% •	INSTITUTIONAL RESPONSIBILITIES (if applicable)

%% 201? – 	Faculty member, Name of University/ Institution/ Country
%% 201? – 201? 	Graduate Student Advisor, Name of University/ Institution/ Country
%% 200? – 200? 	Member of the Faculty Committee, Name of University/ Institution/ Country 
%% 200? – 200? 	Organiser of the Internal Seminar, Name of University/ Institution/ Country
%% 200? – 200? 	Member of a Committee; role, Name of University/ Institution/ Country


%% •	COMMISSIONS OF TRUST (if applicable)

%% 201? – 	Scientific Advisory Board, Name of University/ Institution/ Country
%% 201? – 	Review Board, Name of University/ Institution/ Country
%% 201? –	Review panel member, Name of University/ Institution/ Country
%% 201? – 	Editorial Board, Name of University/ Institution/ Country
%% 200? – 	Scientific Advisory Board, Name of University/ Institution/ Country
%% 200? –	Reviewer, Name of University/ Institution/ Country 
%% 200? –	Scientific Evaluation, Name of University/ Institution/ Country
%% 200? –	Evaluator, Name of University/ Institution/ Country


%% •	MEMBERSHIPS OF SCIENTIFIC SOCIETIES (if applicable)

%% 201? –	Member, Research Network “Name of Research Network”
%% 200? –	Associated Member, Name of Faculty/ Department/Centre, Name of University/ Institution/ Country
%% 200? –	Funding Member, Name of Faculty/ Department/Centre, Name of University/ Institution/ Country 

\subsection*{Selected Presentations}
%% \begin{itemize}
%% \item Aug 2018 and Aug 2019 - Deep Bayes Summer School, Moscow, Russia.
%% \item June 2019 - Machine Learning Crash Course MLCC 2019, Genova, Italy

I receive regular invitations to deliver {\bf keynote} presentations at international events. 
In 2024, I gave an invited talk at the Approximate Inference in Theory and Practice workshop in Paris, and at two sessions on Advances in Inference and Theory for Bayesian Neural Networks and Recent Advances on High Dimensional Models at the JSM conference and at the ICSDS conference, respectively. 
In 2023, I was invited to present my work at the symposium on Advances in Approximate Bayesian Inference (AABI) at ICML and at the Generative Modeling and Uncertainty Quantification (GenU) workshop in Copenhagen. 
In 2022, I opened the workshop on Statistical Deep Learning in Sydney, Australia, in 2019 I presented at the Northern Lights Deep Learning Workshop in Troms\o, Norway, and in 2018 I presented at the Workshop on Surrogate models for Uncertainty Quantification in Complex Systems in Cambridge, UK.
%% \item 5 Dec 2013, Conference on Electronics, Telecommunications and Computers 2013, Lisbon, Portugal.
%% \item 9 Jun 2011, Italian Statistical Society Conference, Bologna, Italy. %- \emph{Bayesian inference in latent variable models and applications}.
%% \end{itemize}

I've also been actively promoting my research through {\bf invited talks}. 
Here is a selected list over the past few years:
NTNU (2023), Aalto University (2023), Stanford University (2022), University California Irvine (2022), University of Wollongong (2022), Data61, Sydney (2022), University of Oxford (2019, 2015), Imperial College (2018), Google Research NYC (2017), Yandex Moscow (2017), University of Sheffield (2015), Columbia University (2017, 2014), Bristol University (2014), University of Edinburgh (2014). %, UTIA Prague (2014), University of Turin (2014, 2012).

In addition, I gave invited lectures at the Deep Bayes summer school in Moscow, Russia (2018, 2019), at the MLCC summer school in Genoa, Italy (2019),
%% In 2019 I presented a tutorial entitled \href{https://ebonilla.github.io/gaussianprocesses/}{``Modern Gaussian Processes: Scalable Inference and Novel Applications''} 
and I delivered a tutorial on Gaussian processes at the IJCNN 2019 conference and a tutorial on Bayesian Deep Learning at the IJCAI 2021 conference. 



%% \begin{itemize}
%% \item Aug 2018 and Aug 2019 - Deep Bayes Summer School, Moscow, Russia.
%% \item June 2019 - Machine Learning Crash Course MLCC 2019, Genova, Italy
%% \item Jan 2019 - Northern Lights Deep Learning Workshop in Troms\o, Norway.
%% \item Feb 2018 - Workshop on Surrogate models for UQ in complex systems, Cambridge, UK.
%% \item 5 Dec 2013, Conference on Electronics, Telecommunications and Computers 2013, Lisbon, Portugal.
%% \item 9 Jun 2011, Italian Statistical Society Conference, Bologna, Italy. %- \emph{Bayesian inference in latent variable models and applications}.
%% \end{itemize}

%% \subsection*{Selected Invited Presentations}
%% \begin{itemize}
%% \item University of Oxford (2019, 2015), Imperial College (2018), Google Research NYC (2017), Yandex Moscow (2017), University of Sheffield (2015), Columbia University (2014, 2009), Bristol University (2014), University of Edinburgh (2014, 2009), UTIA Prague (2014), University of Turin (2014, 2012)
%% \end{itemize}


\subsection*{Media Coverage}

%% My research has been the subject of some media articles in the past.
%% In particular, in July 2019, the article ``Light, a possible solution for a sustainable AI'' featured in {\em The Conversation}, and it is about the topics put forward by \myacro.  
%% Previously, in October 2015 my paper on ``Monte Carlo strength evaluation: Fast and reliable password checking'' has been discussed in an article in the {\em MIT Technology Review website}, 
%% and in March 2012 another one on ``Predicting the conflict level in television political debates: an approach based on crowdsourcing, nonverbal communication and Gaussian processes'' in the {\em New Scientist website}.


\begin{itemize}
\item {\em The Conversation} - 26 July 2019 - ``Light, a possible solution for a sustainable AI''
\item {\em MIT Technology Review website} - 20 October 2015 based on ``Monte Carlo strength evaluation: Fast and reliable password checking''
\item {\em New Scientist website} - 03 March 2012 based on ``Predicting the conflict level in television political debates: an approach based on crowdsourcing, nonverbal communication and Gaussian processes''
\end{itemize}

\subsection*{Major Collaborations}
%% •	MAJOR COLLABORATIONS (if applicable)

I have a number of international collaborations, which developed out of shared scientific interests with {\bf Stephan Mandt} and {\bf Babak Shahbaba} (Computer Science and Statistics departments at University California Irvine), {\bf John P. Cunningham} (Department of Statistics, Columbia University), {\bf Lorenzo A. Rosasco}, (University of Genoa and MIT), {\bf Edwin V. Bonilla} (Data61, Sydney, Australia), {\bf Michael A. Osborne} (University of Oxford, UK), and {\bf Markus Heinonen} (Aalto University, Helsinki, Finland).
I'm also named collaborator in two grants to develop the application of Bayesian Statistics to neuroscience and spatial statistics. 
In particular, the Wellcome trust grant ``BRAINCHART: Normative brain charting for predicting and stratifying psychosis'' with PI {\bf Andre Marquand} (Donders Institute, Nijmegen, The Netherlands), and on the Australian Research Council Discovery Early Career Researcher Award (DECRA) grant with PI {\bf Andrew Zammit-Mangion} (University of Wollongong, Australia).

%% \begin{itemize}
%% \item 2019-2021 - Named Collaborator in Wellcome trust grant: ``BRAINCHART: Normative brain charting for predicting and stratifying psychosis'' - PI: Andre Marquand, Donders Institute, Nijmegen, Netherlands
%% \item 2017-2021 - Named Collaborator in Australian Research Council Discovery Early Career Researcher Award (DECRA) grant - PI: Andrew Zammit-Mangion, University of Wollongong, Australia.
%% \item John Patrick Cunningham, Department of Statistics, Columbia University
%% \item Lorenzo A. Rosasco, University of Genoa and MIT
%% \item Edwin V. Bonilla, Data61, Sydney, Australia
%% \item Michael Osborne, University of Oxford, UK
%% \item James Hensman, Prowler.io, Cambridge, UK
%% \end{itemize}

%% Name of collaborators, Topic, Name of Faculty/ Department/Centre, Name of University/ Institution/ Country

\subsection*{Awards}
\begin{itemize}
     \item Best Ph.D. Thesis Award obtained by former student Ba-Hien Tran, Doctoral School of Sorbonne University, France, December 2023
     \item 
International Association of Pattern Recognition best paper award: 
       %% \begin{itemize}
       %% \item
       M. Filippone, et al. % F. Camastra, F. Masulli, and S. Rovetta.
       \textbf{A survey of kernel and spectral methods for clustering}.
       \emph{Pattern Recognition}, 41(1):176-190, January 2008.
%        Manuscripts published in volume 41 (year 2008) were judged by the Editor-in-Chief and the members of the Editorial and Advisory Boards of the journal based on the following criteria: originality of the contribution, presentation and exposition of the manuscript, and citations by other researchers.
%% \end{itemize}
%% \item I received a ``Special Mention'' award for a poster at the Autumn meeting on Latent Gaussian Models in Trondheim, Norway in 2015. 
\end{itemize}


%% \subsection*{Selected Publications}
%% %\subsubsection*{Journals}

%% %% \footnotesize
%% \linespread{0.9}
%% \selectfont

%% \begin{itemize}
%%   \setlength\itemsep{4pt}
%% %% \item  J.~Wacker, M.~Kanagawa, M.~Filippone. Improved random features for dot product kernels. \emph{Journal of Machine Learning Research}, to appear.
%% \item  A. Zammit-Mangion, M. D. Kaminski, B.-H. Tran, M. Filippone, and N. Cressie. Spatial Bayesian neural networks. {\em Spatial Statistics}, 60:100825, 2024.
%% \item  B.-H. Tran, G. Franzese, P. Michiardi, M. Filippone. One-line-of-code data mollification improves optimization of likelihood-based generative models. {\em NeurIPS 2023}.
%% \item  B.-H. Tran, S. Rossi, D. Milios, and M. Filippone. All you need is a good functional prior for Bayesian deep learning. \emph{Journal of Machine Learning Research}, 23(74):1--56, 2022.
%% \item  S. Marmin and M. Filippone. Deep Gaussian processes for calibration of computer models (with discussion). \emph{Bayesian Analysis}, 17(4): 1301-1350, 2022.
%% \item  A. Zammit-Mangion, T.-L. J. Ng, Q. Vu, and M. Filippone. Deep compositional spatial models. \emph{Journal of the American Statistical Association}, 117(540):1787-1808, 2022.
%% \item  G. Franzese, D. Milios, M. Filippone, P. Michiardi. Revisiting the effects of stochasticity for Hamiltonian samplers. \emph{ICML 2022}.
%% \item  B.-H. Tran, S. Rossi, D. Milios, and M. Filippone. Model Selection for Bayesian Autoencoders. \emph{NeurIPS 2021}.  
%% \item  G.-L. Tran, D. Milios, P. Michiardi, and M. Filippone. Sparse within sparse Gaussian processes using neighbor information. \emph{ICML 2021}.
%% \item  S. Rossi, S. Marmin, and M. Filippone. Walsh-Hadamard Variational Inference for Bayesian Deep Learning. \emph{NeurIPS 2020}.  
%% \item  C. Nemeth, F. Lindsten, M. Filippone, and J. Hensman. Pseudo-extended Markov chain Monte Carlo. \emph{NeurIPS 2019}. 
%% \item  S. Rossi, P. Michiardi, and M. Filippone. Good Initializations of Variational Bayes for Deep Models. \emph{ICML 2019}.  
%% \item  G.-L. Tran, E. V. Bonilla, J. P. Cunningham, P. Michiardi, and M. Filippone. Calibrating Deep Convolutional Gaussian Processes. \emph{AISTATS 2019}.  
%% \item  D. Milios, R. Camoriano, P. Michiardi, L. Rosasco, and M. Filippone. Dirichlet-based Gaussian Processes for Large-scale Calibrated Classification. \emph{NeurIPS 2018}. 
%% \item  M. Lorenzi and M. Filippone. Constraining the Dynamics of Deep Probabilistic Models. \emph{ICML 2018}.  
%% \item  K. Cutajar, E. V. Bonilla, P. Michiardi, and M. Filippone. Random feature expansions for deep Gaussian processes. \emph{ICML 2017}.  
%% \item  K. Cutajar, M. A. Osborne, J. P. Cunningham, and M. Filippone. Preconditioning kernel matrices. \emph{ICML 2016}.  
%% \item  J. Hensman, A. G. de G. Matthews, M. Filippone, and Z. Ghahramani. MCMC for variationally sparse Gaussian processes. \emph{NeurIPS 2015}.
%% \item  M. Filippone and R. Engler. Enabling scalable stochastic gradient-based inference for Gaussian processes by employing the Unbiased LInear System SolvEr (ULISSE). \emph{ICML 2015}.  
%% \item  M. Filippone and M. Girolami. Pseudo-marginal Bayesian inference for Gaussian processes. \emph{IEEE Transactions on Pattern Analysis and Machine Intelligence}, 36(11):2214-2226, 2014.  
%% % \item  F. Dondelinger, M. Filippone, S. Rogers, and D. Husmeier. ODE parameter inference using adaptive gradient matching with Gaussian processes. \emph{AISTATS}, 2013.  
%% \item  M. Filippone, M. Zhong, and M. Girolami. A comparative evaluation of stochastic-based inference methods for Gaussian process models. \emph{Machine Learning}, 93(1):93-114, 2013.  
%% \item  M. Filippone, A. F. Marquand, C. R. V. Blain, S. C. R. Williams, J. Mour\~ao-Miranda, and M. Girolami. Probabilistic prediction of neurological disorders with a statistical assessment of neuroimaging data modalities. \emph{Annals of Applied Statistics}, 6(4):1883-1905, 2012.  
%% %% %% \item  M. Filippone and G. Sanguinetti. Approximate inference of the bandwidth in multivariate kernel density estimation. \emph{Computational Statistics \& Data Analysis}, 55(12):3104-3122, 2011.  
%% %% \item  M. Filippone, F. Masulli, and S. Rovetta. Applying the possibilistic c-means algorithm in kernel-induced spaces. \emph{IEEE Transactions on Fuzzy Systems}, 18(3):572-584, June 2010.  
%% % \item  M. Filippone and G. Sanguinetti. Information theoretic novelty detection. \emph{Pattern Recognition}, 43(3):805-814, March 2010.  
%% %% %% \item  M. Filippone. Dealing with non-metric dissimilarities in fuzzy central clustering algorithms. \emph{International Journal of Approximate Reasoning}, 50(2):363-384, February 2009.  
%% %% \item  M. Filippone, F. Camastra, F. Masulli, and S. Rovetta. A survey of kernel and spectral methods for clustering. \emph{Pattern Recognition}, 41(1):176-190, January 2008.  
%% \end{itemize}


%% •	CAREER BREAKS (if applicable)

%% Exact dates	Please indicate the reason and the duration in months.

%% ************************************************** OLD SHORT CV ***************************************************

\subsection*{Full list of publications}

%% %% \footnotesize
%% \linespread{0.9}
%% \selectfont

\textbf{Journals}\begin{itemize}\item  J. Wacker, M. Kanagawa, and M. Filippone. Improved random features for dot product kernels. \emph{Journal of Machine Learning Research}, 25(235):1-75, 2024.  
\item  A. Zammit-Mangion, M. D. Kaminski, B.-H. Tran, M. Filippone, and N. Cressie. Spatial Bayesian neural networks. \emph{Spatial Statistics}, 60:100825, 2024.  
\item  G. Franzese, S. Rossi, L. Yang, A. Finamore, D. Rossi, M. Filippone, and P. Michiardi. How much is enough? a study on diffusion times in score-based generative models. \emph{Entropy}, 25(4), 2023.  
\item  B.-H. Tran, S. Rossi, D. Milios, and M. Filippone. All you need is a good functional prior for Bayesian deep learning. \emph{Journal of Machine Learning Research}, 23(74):1-56, 2022.  
\item  S. Marmin and M. Filippone. Deep Gaussian Processes for Calibration of Computer Models (with Discussion). \emph{Bayesian Analysis}, 17(4):1301 - 1350, 2022.  
\item  C. Carota, M. Filippone, and S. Polettini. Assessing Bayesian semi-parametric log-linear models: An application to disclosure risk estimation. \emph{International Statistical Review}, 90(1):165-183, 2022.  
\item  Q. V. Andrew Zammit-Mangion, Tin Lok James Ng and M. Filippone. Deep compositional spatial models. \emph{Journal of the American Statistical Association}, 117(540):1787-1808, 2022.  
\item  R. Domingues, P. Michiardi, J. Barlet, and M. Filippone. A comparative evaluation of novelty detection algorithms for discrete sequences. \emph{Artificial Intelligence Review}, 53:3787-3812, 2020.  
\item  M. Lorenzi, M. Filippone, G. B. Frisoni, D. C. Alexander, and S. Ourselin. Probabilistic disease progression modeling to characterize diagnostic uncertainty: Application to staging and prediction in alzheimer's disease. \emph{NeuroImage}, 190:56-68, 2019.  
\item  R. Domingues, P. Michiardi, J. Zouaoui, and M. Filippone. Deep Gaussian Process autoencoders for novelty detection. \emph{Machine Learning}, 107(8-10):1363-1383, 2018.  
\item  R. Domingues, M. Filippone, P. Michiardi, and J. Zouaoui. A comparative evaluation of outlier detection algorithms: Experiments and analyses. \emph{Pattern Recognition}, 74:406-421, 2018.  
\item  M. Niu, B. Macdonald, S. Rogers, M. Filippone, and D. Husmeier. Statistical inference in mechanistic models: time warping for improved gradient matching. \emph{Computational Statistics}, pages 1-33, 8 2017.  
\item  X. Xiong, V. Šm\'{\i}dl, and M. Filippone. Adaptive multiple importance sampling for Gaussian processes. \emph{Journal of Statistical Computation and Simulation}, 87(8):1644-1665, 2017.  
\item  B. Macdonald, M. Niu, S. Rogers, M. Filippone, and D. Husmeier. Approximate parameter inference in systems biology using gradient matching: a comparative evaluation. \emph{BioMedical Engineering OnLine}, 15(Suppl 1):80, 7 2016.  
\item  J. M. Rondina, M. Filippone, M. Girolami, and N. S. Ward. Decoding post-stroke motor function from structural brain imaging. \emph{NeuroImage: Clinical}, 12:372-380, 2016.  
\item  C. Carota, M. Filippone, R. Leombruni, and S. Polettini. Bayesian nonparametric disclosure risk estimation via mixed effects log-linear models. \emph{Annals of Applied Statistics}, 9(1):525-546, 2015.  
\item  M. Dell'Amico, M. Filippone, P. Michiardi, and Y. Roudier. On user availability prediction and network applications. \emph{IEEE/ACM Transactions on Networking}, 23(4):1300-1313, Aug 2015.  
\item  M. Filippone and M. Girolami. Pseudo-marginal Bayesian inference for Gaussian processes. \emph{IEEE Transactions on Pattern Analysis and Machine Intelligence}, 36(11):2214-2226, 2014.  
\item  S. Kim, F. Valente, M. Filippone, and A. Vinciarelli. Predicting continuous conflict perception with Bayesian Gaussian processes. \emph{IEEE Transactions on Affective Computing}, 5(2):187-200, 2014.  
\item  A. F. Marquand, M. Filippone, J. Ashburner, M. Girolami, J. Mour\~ao-Miranda, G. J. Barker, S. C. R. Williams, P. N. Leigh, and C. R. V. Blain. Automated, High Accuracy Classification of Parkinsonian Disorders: A Pattern Recognition Approach. \emph{PLoS ONE}, 8(7):e69237+, 2013.  
\item  M. Filippone, M. Zhong, and M. Girolami. A comparative evaluation of stochastic-based inference methods for Gaussian process models. \emph{Machine Learning}, 93(1):93-114, 2013.  
\item  Y. Zhao, J. Kim, and M. Filippone. Aggregation algorithm towards large-scale boolean network analysis. \emph{IEEE Transactions on Automatic Control}, 58(8):1976-1985, 2013.  
\item  M. Filippone, A. F. Marquand, C. R. V. Blain, S. C. R. Williams, J. Mour\~ao-Miranda, and M. Girolami. Probabilistic prediction of neurological disorders with a statistical assessment of neuroimaging data modalities. \emph{Annals of Applied Statistics}, 6(4):1883-1905, 2012.  
\item  L. Mohamed, B. Calderhead, M. Filippone, M. Christie, and M. Girolami. Population MCMC methods for history matching and uncertainty quantification. \emph{Computational Geosciences}, 16(2):423-436, 2012.  
\item  M. Filippone and G. Sanguinetti. Approximate inference of the bandwidth in multivariate kernel density estimation. \emph{Computational Statistics \& Data Analysis}, 55(12):3104-3122, 2011.  
\item  M. Filippone and G. Sanguinetti. A perturbative approach to novelty detection in autoregressive models. \emph{IEEE Transactions on Signal Processing}, 59(3):1027-1036, 2011.  
\item  M. Filippone, F. Masulli, and S. Rovetta. Simulated annealing for supervised gene selection. \emph{Soft Computing - A Fusion of Foundations, Methodologies and Applications}, 15:1471-1482, 2011.  
\item  M. Filippone, F. Masulli, and S. Rovetta. Applying the possibilistic c-means algorithm in kernel-induced spaces. \emph{IEEE Transactions on Fuzzy Systems}, 18(3):572-584, June 2010.  
\item  M. Filippone and G. Sanguinetti. Information theoretic novelty detection. \emph{Pattern Recognition}, 43(3):805-814, March 2010.  
\item  M. Filippone. Dealing with non-metric dissimilarities in fuzzy central clustering algorithms. \emph{International Journal of Approximate Reasoning}, 50(2):363-384, February 2009.  
\item  F. Camastra and M. Filippone. A comparative evaluation of nonlinear dynamics methods for time series prediction. \emph{Neural Computing and Applications}, 18(8):1021-1029, November 2009.  
\item  M. Filippone, F. Masulli, and S. Rovetta. Clustering in the membership embedding space. \emph{International Journal of Knowledge Engineering and Soft Data Paradigms}, 4(1):363-375, 2009. 
\item  S. Rovetta, F. Masulli, and M. Filippone. Soft ranking in clustering. \emph{Neurocomputing}, 72(7-9):2028-2031, March 2009.  
\item  M. Filippone, F. Camastra, F. Masulli, and S. Rovetta. A survey of kernel and spectral methods for clustering. \emph{Pattern Recognition}, 41(1):176-190, January 2008.  

\end{itemize}\textbf{Conferences}\begin{itemize}\item  M. Heinonen, B.-H. Tran, M. Kampffmeyer, and M. Filippone. Robust classification by coupling data mollification with label smoothing. In \emph{The 28th International Conference on Artificial Intelligence and Statistics}, 2025.  
\item  A. Lh\'eritier and M. Filippone. Unconditionally calibrated priors for beta mixture density networks. In \emph{The 28th International Conference on Artificial Intelligence and Statistics}, 2025.  
\item  A. Benechehab, Y. A. E. Hili, A. Odonnat, O. Zekri, A. Thomas, G. Paolo, M. Filippone, I. Redko, and B. K\'egl. Zero-shot model-based reinforcement learning using large language models. In \emph{The Thirteenth International Conference on Learning Representations}, 2025.  
\item  T. Papamarkou, M. Skoularidou, K. Palla, L. Aitchison, J. Arbel, D. Dunson, M. Filippone, V. Fortuin, P. Hennig, J. M. Hern\'andez-Lobato, A. Hubin, A. Immer, T. Karaletsos, M. E. Khan, A. Kristiadi, Y. Li, S. Mandt, C. Nemeth, M. A. Osborne, T. G. J. Rudner, D. R\"ugamer, Y. W. Teh, M. Welling, A. G. Wilson, and R. Zhang. Position: Bayesian deep learning is needed in the age of large-scale AI. In R. Salakhutdinov, Z. Kolter, K. Heller, A. Weller, N. Oliver, J. Scarlett, and F. Berkenkamp, editors, \emph{Proceedings of the 41st International Conference on Machine Learning}, volume 235 of \emph{Proceedings of Machine Learning Research}, pages 39556-39586. PMLR, 21-27 Jul 2024.  
\item  B.-H. Tran, G. Franzese, P. Michiardi, and M. Filippone. One-line-of-code data mollification improves optimization of likelihood-based generative models. In A. Oh, T. Neumann, A. Globerson, K. Saenko, M. Hardt, and S. Levine, editors, \emph{Advances in Neural Information Processing Systems}, volume 36, pages 6545-6567. Curran Associates, Inc., 2023.  
\item  G. Franzese, G. Corallo, S. Rossi, M. Heinonen, M. Filippone, and P. Michiardi. Continuous-time functional diffusion processes. In A. Oh, T. Neumann, A. Globerson, K. Saenko, M. Hardt, and S. Levine, editors, \emph{Advances in Neural Information Processing Systems}, volume 36, pages 37370-37400. Curran Associates, Inc., 2023.  
\item  J. Wacker, R. Ohana, and M. Filippone. Complex-to-real sketches for tensor products with applications to the polynomial kernel. In F. Ruiz, J. Dy, and J.-W. van de Meent, editors, \emph{Proceedings of The 26th International Conference on Artificial Intelligence and Statistics}, volume 206 of \emph{Proceedings of Machine Learning Research}, pages 5181-5212. PMLR, 25-27 Apr 2023.  
\item  G. Franzese, D. Milios, M. Filippone, and P. Michiardi. Revisiting the effects of stochasticity for Hamiltonian samplers. In K. Chaudhuri, S. Jegelka, L. Song, C. Szepesvari, G. Niu, and S. Sabato, editors, \emph{Proceedings of the 39th International Conference on Machine Learning}, volume 162 of \emph{Proceedings of Machine Learning Research}, pages 6744-6778. PMLR, 17-23 Jul 2022.  
\item  B.-H. Tran, S. Rossi, D. Milios, P. Michiardi, E. V. Bonilla, and M. Filippone. Model selection for Bayesian autoencoders. In A. Beygelzimer, Y. Dauphin, P. Liang, and J. W. Vaughan, editors, \emph{Advances in Neural Information Processing Systems}, 2021.  
\item  G.-L. Tran, D. Milios, P. Michiardi, and M. Filippone. Sparse within Sparse Gaussian Processes using Neighbor Information. In M. Meila and T. Zhang, editors, \emph{Proceedings of the 38th International Conference on Machine Learning}, volume 139 of \emph{Proceedings of Machine Learning Research}, pages 10369-10378. PMLR, 18-24 Jul 2021.  
\item  G. Mita, M. Filippone, and P. Michiardi. An Identifiable Double VAE For Disentangled Representations. In M. Meila and T. Zhang, editors, \emph{Proceedings of the 38th International Conference on Machine Learning}, volume 139 of \emph{Proceedings of Machine Learning Research}, pages 7769-7779. PMLR, 18-24 Jul 2021.  
\item  S. Rossi, M. Heinonen, E. Bonilla, Z. Shen, and M. Filippone. Sparse Gaussian Processes Revisited: Bayesian Approaches to Inducing-Variable Approximations. In A. Banerjee and K. Fukumizu, editors, \emph{Proceedings of The 24th International Conference on Artificial Intelligence and Statistics}, volume 130 of \emph{Proceedings of Machine Learning Research}, pages 1837-1845. PMLR, 13-15 Apr 2021.  
\item  S. Rossi, S. Marmin, and M. Filippone. Walsh-Hadamard variational inference for Bayesian deep learning. In H. Larochelle, M. Ranzato, R. Hadsell, M. Balcan, and H. Lin, editors, \emph{Advances in Neural Information Processing Systems}, volume 33, pages 9674-9686. Curran Associates, Inc., 2020.  
\item  G. Mita, P. Papotti, M. Filippone, and P. Michiardi. LIBRE: Learning Interpretable Boolean Rule Ensembles. In \emph{AISTATS 2020, Palermo, Italy}, 2020.  
\item  S. Rossi, S. Marmin, and M. Filippone. Efficient approximate inference with walsh-hadamard variational inference. In \emph{Bayesian Deep Learning Workshop, NeurIPS}, 2019.  
\item  C. Nemeth, F. Lindsten, M. Filippone, and J. Hensman. Pseudo-extended Markov chain Monte Carlo. In \emph{Advances in Neural Information Processing Systems 32: Annual Conference on Neural Information Processing Systems 2019, 9-12 December 2019, Vancouver, British Columbia, Canada}, 2019.  
\item  S. Rossi, P. Michiardi, and M. Filippone. Good Initializations of Variational Bayes for Deep Models. In \emph{Proceedings of the 36th International Conference on Machine Learning, ICML 2019, Long Beach, USA, 2019}, 2019.  
\item  G.-L. Tran, E. V. Bonilla, J. P. Cunningham, P. Michiardi, and M. Filippone. Calibrating Deep Convolutional Gaussian Processes. In \emph{AISTATS 2019, Naha, Japan, 2019}, 2019.  
\item  D. Nguyen, M. Filippone, and P. Michiardi. Exact Gaussian process regression with distributed computations. In C. Hung and G. A. Papadopoulos, editors, \emph{Proceedings of the 34th ACM/SIGAPP Symposium on Applied Computing, SAC 2019, Limassol, Cyprus, April 8-12, 2019}, pages 1286-1295. ACM, 2019.  
\item  D. Milios, R. Camoriano, P. Michiardi, L. Rosasco, and M. Filippone. Dirichlet-based Gaussian Processes for Large-scale Calibrated Classification. In \emph{Advances in Neural Information Processing Systems 31: Annual Conference on Neural Information Processing Systems 2018, December 3-7 2018, Montreal, Quebec, Canada}, 2018.  
\item  M. Lorenzi and M. Filippone. Constraining the Dynamics of Deep Probabilistic Models. In \emph{Proceedings of the 35th International Conference on Machine Learning, ICML 2018, Stockholm, Sweden, 2018}, 2018.  
\item  J. Fitzsimons, D. Granziol, K. Cutajar, M. Osborne, M. Filippone, and S. Roberts. Entropic Trace Estimates for Log Determinants. In \emph{Machine Learning and Knowledge Discovery in Databases - European Conference, ECML PKDD 2017, Skopje, Macedonia, September 18-22, 2017}, 2017.  
\item  J. Fitzsimons, K. Cutajar, M. Osborne, S. Roberts, and M. Filippone. Bayesian Inference of Log Determinants. In \emph{Thirty-Third Conference on Uncertainty in Artificial Intelligence, UAI 2017, August 11-15, 2017, Sydney, Australia}, 2017.  
\item  K. Krauth, E. V. Bonilla, K. Cutajar, and M. Filippone. AutoGP: Exploring the capabilities and limitations of Gaussian process models. In \emph{Thirty-Third Conference on Uncertainty in Artificial Intelligence, UAI 2017, August 11-15, 2017, Sydney, Australia}, 2017.  
\item  K. Cutajar, E. V. Bonilla, P. Michiardi, and M. Filippone. Random feature expansions for deep Gaussian processes. In \emph{Proceedings of the 34th International Conference on Machine Learning, ICML 2017, Sydney, Australia, August 6-11, 2017}, 2017. 
\item  Y. Han and M. Filippone. Mini-batch spectral clustering. In \emph{2016 International Joint Conference on Neural Networks, IJCNN 2017, Anchorage, AK, USA, May 14-19, 2017}. IEEE, 2017. 
\item  K. Cutajar, E. V. Bonilla, P. Michiardi, and M. Filippone. Accelerating deep Gaussian processes inference with arc-cosine kernels. In \emph{Bayesian Deep Learning Workshop, NIPS}, 2016.  
\item  X. Xiong, M. Filippone, and A. Vinciarelli. Looking good with flickr faves: Gaussian processes for finding difference makers in personality impressions. In \emph{ACM Multimedia}, 2016. 
\item  K. Cutajar, M. A. Osborne, J. P. Cunningham, and M. Filippone. Preconditioning kernel matrices. In \emph{Proceedings of the 33rd International Conference on Machine Learning, ICML 2016, New York City, USA, June 19-24, 2016}, 2016. 
\item  M. Niu, S. Rogers, M. Filippone, and D. Husmeier. Fast inference in nonlinear dynamical systems using gradient matching. In \emph{Proceedings of the 33rd International Conference on Machine Learning, ICML 2016, New York City, USA, June 19-24, 2016}, 2016. 
\item  J. Hensman, A. G. de G. Matthews, M. Filippone, and Z. Ghahramani. MCMC for variationally sparse Gaussian processes. In \emph{Advances in Neural Information Processing Systems 28: Annual Conference on Neural Information Processing Systems 2015, December 7-12 2015, Montreal, Quebec, Canada}, 2015. 
\item  M. Dell'Amico and M. Filippone. Monte Carlo strength evaluation: Fast and reliable password checking. In \emph{Proceedings of the 22nd ACM Conference on Computer and Communications Security}, 2015. 
\item  M. Filippone and R. Engler. Enabling scalable stochastic gradient-based inference for Gaussian processes by employing the Unbiased LInear System SolvEr (ULISSE). In \emph{Proceedings of the 32nd International Conference on Machine Learning, ICML 2015, Lille, France, July 6-11, 2015}, 2015. 
\item  M. Filippone. Bayesian inference for Gaussian process classifiers with annealing and pseudo-marginal MCMC. In \emph{22nd International Conference on Pattern Recognition, ICPR 2014, Stockholm, Sweden, August 24-28, 2014}, pages 614-619, 2014.  
\item  A. D. O'Harney, A. Marquand, K. Rubia, K. Chantiluke, A. B. Smith, A. Cubillo, C. Blain, and M. Filippone. Pseudo-marginal Bayesian multiple-class multiple-kernel learning for neuroimaging data. In \emph{22nd International Conference on Pattern Recognition, ICPR 2014, Stockholm, Sweden, August 24-28, 2014}, pages 3185-3190, 2014.  
\item  F. Dondelinger, M. Filippone, S. Rogers, and D. Husmeier. ODE parameter inference using adaptive gradient matching with Gaussian processes. In \emph{AISTATS}, 2013. 
\item  S. Kim, M. Filippone, F. Valente, and A. Vinciarelli. Predicting the conflict level in television political debates: an approach based on crowdsourcing, nonverbal communication and Gaussian processes. In \emph{Proceedings of the 20th ACM Multimedia Conference, MM '12, Nara, Japan, October 29 - November 02, 2012}, pages 793-796. ACM, 2012.  
\item  G. Mohammadi, A. Origlia, M. Filippone, and A. Vinciarelli. From speech to personality: mapping voice quality and intonation into personality differences. In \emph{Proceedings of the 20th ACM Multimedia Conference, MM '12, Nara, Japan, October 29 - November 02, 2012}, pages 789-792. ACM, 2012.  
\item  D. Barbar\'a, C. Domeniconi, Z. Duric, M. Filippone, R. Mansfield, and E. Lawson. Detecting suspicious behavior in surveillance images. In \emph{Workshops Proceedings of the 8th IEEE International Conference on Data Mining (ICDM 2008), December 15-19, 2008, Pisa, Italy}, pages 891-900. IEEE, 2008.  
\item  M. Filippone, F. Masulli, and S. Rovetta. Stability and performances in biclustering algorithms. In \emph{Computational Intelligence Methods for Bioinformatics and Biostatistics, 5th International Meeting, CIBB 2008, Vietri sul Mare, Italy, October 3-4, 2008, Revised Selected Papers}, volume 5488 of \emph{Lecture Notes in Computer Science}, pages 91-101. Springer, 2008.  
\item  M. Filippone, F. Masulli, and S. Rovetta. An experimental comparison of kernel clustering methods. In \emph{New Directions in Neural Networks - 18th Italian Workshop on Neural Networks: WIRN 2008, Vietri sul Mare, Italy, May 22-24, 2008, Revised Selected Papers}, volume 193 of \emph{Frontiers in Artificial Intelligence and Applications}, pages 118-126. IOS Press, 2008.  
\item  F. Camastra and M. Filippone. SVM-based time series prediction with nonlinear dynamics methods. In \emph{Knowledge-Based Intelligent Information and Engineering Systems, 11th International Conference, KES 2007, XVII Italian Workshop on Neural Networks, Vietri sul Mare, Italy, September 12-14, 2007, Proceedings, Part III}, volume 4694 of \emph{Lecture Notes in Computer Science}, pages 300-307. Springer, 2007.  
\item  S. Rovetta, F. Masulli, and M. Filippone. Membership embedding space approach and spectral clustering. In \emph{Knowledge-Based Intelligent Information and Engineering Systems, 11th International Conference, KES 2007, XVII Italian Workshop on Neural Networks, Vietri sul Mare, Italy, September 12-14, 2007, Proceedings, Part III}, volume 4694 of \emph{Lecture Notes in Computer Science}, pages 901-908. Springer, 2007.  
\item  E. Canestrelli, P. Canestrelli, M. Corazza, M. Filippone, S. Giove, and F. Masulli. Local learning of tide level time series using a fuzzy approach. In \emph{Proceedings of the International Joint Conference on Neural Networks, IJCNN 2007, Celebrating 20 years of neural networks, Orlando, Florida, USA, August 12-17, 2007}, pages 1813-1818. IEEE, 2007.  
\item  M. Filippone, F. Masulli, and S. Rovetta. Possibilistic clustering in feature space. In \emph{Applications of Fuzzy Sets Theory, 7th International Workshop on Fuzzy Logic and Applications, WILF 2007, Camogli, Italy, July 7-10, 2007, Proceedings}, volume 4578 of \emph{Lecture Notes in Computer Science}, pages 219-226. Springer, 2007.  
\item  M. Filippone, F. Masulli, S. Rovetta, S. Mitra, and H. Banka. Possibilistic approach to biclustering: An application to oligonucleotide microarray data analysis. In \emph{Computational Methods in Systems Biology, International Conference, CMSB 2006, Trento, Italy, October 18-19, 2006, Proceedings}, volume 4210 of \emph{Lecture Notes in Computer Science}, pages 312-322. Springer, 2006.  
\item  M. Filippone, F. Masulli, S. Rovetta, and S.-P. Constantinescu. Input selection with mixed data sets: A simulated annealing wrapper approach. In \emph{CISI 06 - Conferenza Italiana Sistemi Intelligenti}, Ancona - Italy, 27-29 September 2006. 
\item  M. Filippone, F. Masulli, and S. Rovetta. Gene expression data analysis in the membership embedding space: A constructive approach. In \emph{CIBB 2006 - Third International Meeting on Computational Intelligence Methods for Bioinformatics and Biostatistics}, Genova - Italy, 29-31 August 2006. 
\item  M. Filippone, F. Masulli, and S. Rovetta. Supervised classification and gene selection using simulated annealing. In \emph{Proceedings of the International Joint Conference on Neural Networks, IJCNN 2006, part of the IEEE World Congress on Computational Intelligence, WCCI 2006, Vancouver, BC, Canada, 16-21 July 2006}, pages 3566-3571. IEEE, 2006.  
\item  M. Filippone, F. Masulli, and S. Rovetta. Unsupervised gene selection and clustering using simulated annealing. In \emph{Fuzzy Logic and Applications, 6th International Workshop, WILF 2005, Crema, Italy, September 15-17, 2005, Revised Selected Papers}, volume 3849 of \emph{Lecture Notes in Computer Science}, pages 229-235. Springer, 2005.  
\item  F. Masulli, S. Rovetta, and M. Filippone. Clustering genomic data in the membership embedding space. In \emph{CI-BIO - Workshop on Computational Intelligence Approaches for the Analysis of Bioinformatics Data}, Montreal - Canada, 5 August 2005. 
\item  S. Rovetta, F. Masulli, and M. Filippone. Soft rank clustering. In \emph{Neural Nets, 16th Italian Workshop on Neural Nets, WIRN 2005, and International Workshop on Natural and Artificial Immune Systems, NAIS 2005, Vietri sul Mare, Italy, June 8-11, 2005, Revised Selected Papers}, volume 3931 of \emph{Lecture Notes in Computer Science}, pages 207-213. Springer, 2005.  
\item  M. Filippone, F. Masulli, and S. Rovetta. ERAF: a R package for regression and forecasting. In \emph{Biological and Artificial Intelligence Environments}, pages 165-173, Secaucus, NJ, USA, 2004. Springer-Verlag New York, Inc. 

\end{itemize}\textbf{Discussions}\begin{itemize}\item  S. Rossi, C. Rusu, L. A. Rosasco, and M. Filippone. Contributed discussion on ``A Bayesian conjugate gradient method''. \emph{Bayesian Analysis, 14(3), 2019}, 10 2019.  
\item  M. Filippone, A. Mira, and M. Girolami. Discussion of the paper: ”Sampling schemes for generalized linear Dirichlet process random effects models” by M. Kyung, J. Gill, and G. Casella. \emph{Statistical Methods \& Applications}, 20:295-297, 2011.  
\item  M. Filippone. Discussion of the paper ”Riemann manifold Langevin and Hamiltonian Monte Carlo methods” by Mark Girolami and Ben Calderhead. \emph{Journal of the Royal Statistical Society, Series B (Statistical Methodology)}, 73(2):164-165, 2011.  
\item  V. Stathopoulos and M. Filippone. Discussion of the paper ”Riemann manifold Langevin and Hamiltonian Monte Carlo methods” by Mark Girolami and Ben Calderhead. \emph{Journal of the Royal Statistical Society, Series B (Statistical Methodology)}, 73(2):167-168, March 2011.  

\end{itemize}\textbf{Theses}\begin{itemize}\item  M. Filippone. \emph{Central Clustering in Kernel-Induced Spaces}. Phd thesis in computer science, University of Genova, February 2008. 
\item  M. Filippone. Metodi di ensemble per la previsione di serie storiche. Master's degree thesis in physics, University of Genova, July 2004. 

\end{itemize}



\end{document}

\subsection*{Education}
\begin{itemize}
\item 2008 - Ph.D. in Computer Science - University of Genoa \\
Thesis title: Central Clustering in Kernel-Induced Spaces \\
Keywords: kernel methods for clustering, spectral clustering, relational clustering.

\item 2004 - Master's Degree in Physics (full marks: 110/110) - University of Genoa \\
Thesis title: Ensemble methods for time series analysis and forecasting \\
Keywords: Nonlinear systems, regression, ensemble of learning machines, signal processing.
\end{itemize}

\subsection*{Research Experience}
\begin{itemize}
\item From Spring 2018 to present - {\em Associate Professor} \\
  EURECOM, Sophia Antipolis, France \\
  Keywords: Bayesian inference, Nonparametric modeling, Scalable inference

\item From Fall 2015 to Spring 2018 - {\em Ma\^{i}tre de Conf\'{e}rence} \\
  EURECOM, Sophia Antipolis, France \\
  Keywords: Bayesian inference, Nonparametric modeling, Scalable inference

\item From Fall 2011 to Fall 2015 - {\em Lecturer} \\
  School of Computing Science - University of Glasgow \\
  Keywords: Bayesian inference, Gaussian Processes, Markov chain Monte Carlo

\item From Fall 2010 to Fall 2011 - {\em Research Associate} (PI: Prof. Mark Girolami) \\
  Department of Statistical Science - University College London (2011) \\
  School of Computing Science - University of Glasgow (2010) \\
%  Grant: The Synthesis of Probabilistic Prediction \& Mechanistic Modelling within a Computational \& Systems Biology Context (
  Keywords: Bayesian inference, Gaussian Processes, Markov chain Monte Carlo
  
\item From Spring 2008 to Fall 2009 - {\em Research Associate} (PI: Dr G. Sanguinetti) \\
  Department of Computer Science - University of Sheffield \\
%  Grant: ALMS: Advanced Lifestyle Monitoring Systems 
  Keywords: novelty detection, statistical testing, Bayesian inference

\item From Spring 2007 to Fall 2007 - {\em Research Scholar} (PIs: Profs. D. Barbar\`a, C. Domeniconi) \\
  Department of Information and Software Engineering - George Mason University \\
%  Grant: Detecting Suspicious Behavior in Reconnaissance Images 
  Keywords: outlier detection, density estimation, relational clustering
  
\end{itemize}

\subsection*{Professional Activities}
\begin{itemize}
\item {\em Associate Editor} for Pattern Recognition (end 2012 - end 2016)
\item {\em Associate Editor} for the IEEE Transactions on Neural Networks and Learning Systems (2013 - end 2016)
\item {\em Technical Program Chair} for IJCNN 2014
\end{itemize}

\subsection*{Research Grants}
\begin{itemize}
  \item \emph{ECO-ML: Rethinking Modern Machine Learning Tools for a New Generation of Low-Power Large-Scale Modeling Systems} (300K\euro), 2018--2021, ANR-JCJC 
  \item AXA Chair of Computational Statistics: \emph{New Computational Approaches to Risk Modeling} (600K\euro), 2016--2023, AXA Research Fund 
  \item Co-PI: \emph{Computational inference of biopathway dynamics and structures} (\pounds 340K), 2014--2017, (PI) D.~Husmeier and (Co-PI) S.~Rogers - EPSRC (UK) research grant 
\end{itemize}

\subsection*{Contracts with Industry}
\begin{itemize}
  \item SAP, CIFRE PhD scholarship (from Fall 2017)
  \item Amadeus, CIFRE PhD scholarship (from Spring 2016) + PhD scholarship (from Spring 2018)
  \item Huawei, 6-months Internship (Spring 2018)
\end{itemize}

\subsection*{Selected Publications}
%\subsubsection*{Journals}



\begin{itemize}
\item  D. Milios, R. Camoriano, P. Michiardi, L. Rosasco, and M. Filippone. Dirichlet-based Gaussian Processes for Large-scale Calibrated Classification. In \emph{Advances in Neural Information Processing Systems 31: Annual Conference on Neural Information Processing Systems 2018, December 3-7 2018, Montreal, Quebec, Canada}, 2018.  
\item  M. Lorenzi and M. Filippone. Constraining the Dynamics of Deep Probabilistic Models. In \emph{Proceedings of the 35th International Conference on Machine Learning, ICML 2018, Stockholm, Sweden, 2018}, 2018.  
\item  K. Cutajar, E. V. Bonilla, P. Michiardi, and M. Filippone. Random feature expansions for deep Gaussian processes. In \emph{Proceedings of the 34th International Conference on Machine Learning, ICML 2017, Sydney, Australia, August 6-11, 2017}, 2017.  
\item  K. Cutajar, M. A. Osborne, J. P. Cunningham, and M. Filippone. Preconditioning kernel matrices. In \emph{Proceedings of the 33rd International Conference on Machine Learning, ICML 2016, New York City, USA, June 19-24, 2016}, 2016.  
\item  J. Hensman, A. G. de G. Matthews, M. Filippone, and Z. Ghahramani. MCMC for variationally sparse Gaussian processes. In \emph{Advances in Neural Information Processing Systems 28: Annual Conference on Neural Information Processing Systems 2015, December 7-12 2015, Montreal, Quebec, Canada}, 2015.  
\item  M. Filippone and R. Engler. Enabling scalable stochastic gradient-based inference for Gaussian processes by employing the Unbiased LInear System SolvEr (ULISSE). In \emph{Proceedings of the 32nd International Conference on Machine Learning, ICML 2015, Lille, France, July 6-11, 2015}, 2015.  
\item  M. Filippone and M. Girolami. Pseudo-marginal Bayesian inference for Gaussian processes. \emph{IEEE Transactions on Pattern Analysis and Machine Intelligence}, 36(11):2214-2226, 2014.  
\item  F. Dondelinger, M. Filippone, S. Rogers, and D. Husmeier. ODE parameter inference using adaptive gradient matching with Gaussian processes. In \emph{AISTATS}, 2013.  
\item  M. Filippone, M. Zhong, and M. Girolami. A comparative evaluation of stochastic-based inference methods for Gaussian process models. \emph{Machine Learning}, 93(1):93-114, 2013.  
\item  M. Filippone, A. F. Marquand, C. R. V. Blain, S. C. R. Williams, J. Mour\~ao-Miranda, and M. Girolami. Probabilistic prediction of neurological disorders with a statistical assessment of neuroimaging data modalities. \emph{Annals of Applied Statistics}, 6(4):1883-1905, 2012.  
%% %% \item  M. Filippone and G. Sanguinetti. Approximate inference of the bandwidth in multivariate kernel density estimation. \emph{Computational Statistics \& Data Analysis}, 55(12):3104-3122, 2011.  
%% \item  M. Filippone, F. Masulli, and S. Rovetta. Applying the possibilistic c-means algorithm in kernel-induced spaces. \emph{IEEE Transactions on Fuzzy Systems}, 18(3):572-584, June 2010.  
\item  M. Filippone and G. Sanguinetti. Information theoretic novelty detection. \emph{Pattern Recognition}, 43(3):805-814, March 2010.  
%% %% \item  M. Filippone. Dealing with non-metric dissimilarities in fuzzy central clustering algorithms. \emph{International Journal of Approximate Reasoning}, 50(2):363-384, February 2009.  
%% \item  M. Filippone, F. Camastra, F. Masulli, and S. Rovetta. A survey of kernel and spectral methods for clustering. \emph{Pattern Recognition}, 41(1):176-190, January 2008.  
\end{itemize}

%% \begin{itemize}\item  M. Filippone and M. Girolami. Pseudo-marginal Bayesian inference for Gaussian processes. \emph{IEEE Transactions on Pattern Analysis and Machine Intelligence}, 36(11):2214--2226, 2014.  
\item  M. Filippone, M. Zhong, and M. Girolami. A comparative evaluation of stochastic-based inference methods for Gaussian process models. \emph{Machine Learning}, 93(1):93--114, 2013.  
\item  M. Filippone, A. F. Marquand, C. R. V. Blain, S. C. R. Williams, J. Mour\~ao-Miranda, and M. Girolami. Probabilistic prediction of neurological disorders with a statistical assessment of neuroimaging data modalities. \emph{Annals of Applied Statistics}, 6(4):1883--1905, 2012.  
\item  M. Filippone and G. Sanguinetti. Approximate inference of the bandwidth in multivariate kernel density estimation. \emph{Computational Statistics \& Data Analysis}, 55(12):3104--3122, 2011.  
\item  M. Filippone and G. Sanguinetti. A perturbative approach to novelty detection in autoregressive models. \emph{IEEE Transactions on Signal Processing}, 59(3):1027--1036, 2011.  
\item  M. Filippone, F. Masulli, and S. Rovetta. Applying the possibilistic c-means algorithm in kernel-induced spaces. \emph{IEEE Transactions on Fuzzy Systems}, 18(3):572--584, June 2010.  
\item  M. Filippone and G. Sanguinetti. Information theoretic novelty detection. \emph{Pattern Recognition}, 43(3):805--814, March 2010.  
\item  M. Filippone. Dealing with non-metric dissimilarities in fuzzy central clustering algorithms. \emph{International Journal of Approximate Reasoning}, 50(2):363--384, February 2009.  
\item  M. Filippone, F. Camastra, F. Masulli, and S. Rovetta. A survey of kernel and spectral methods for clustering. \emph{Pattern Recognition}, 41(1):176--190, January 2008.  
\item  K. Cutajar, E. V. Bonilla, P. Michiardi, and M. Filippone. Random feature expansions for deep Gaussian processes. In \emph{Proceedings of the 34th International Conference on Machine Learning, ICML 2017, Sydney, Australia, August 6-11, 2017}, 2017.  
\item  K. Cutajar, M. A. Osborne, J. P. Cunningham, and M. Filippone. Preconditioning kernel matrices. In \emph{Proceedings of the 33rd International Conference on Machine Learning, ICML 2016, New York City, USA, June 19-24, 2016}, 2016.  
\item  J. Hensman, A. G. de G. Matthews, M. Filippone, and Z. Ghahramani. MCMC for variationally sparse Gaussian processes. In \emph{Advances in Neural Information Processing Systems 28: Annual Conference on Neural Information Processing Systems 2015, December 7-12 2015, Montreal, Quebec, Canada}, 2015.  
\item  M. Filippone and R. Engler. Enabling scalable stochastic gradient-based inference for Gaussian processes by employing the Unbiased LInear System SolvEr (ULISSE). In \emph{Proceedings of the 32nd International Conference on Machine Learning, ICML 2015, Lille, France, July 6-11, 2015}, 2015.  
\item  F. Dondelinger, M. Filippone, S. Rogers, and D. Husmeier. ODE parameter inference using adaptive gradient matching with Gaussian processes. In \emph{AISTATS}, 2013.  

\end{itemize}


\subsection*{Awards}
\begin{itemize}
     \item International Association of Pattern Recognition best paper award: 
       %% \begin{itemize}
       %% \item

       M. Filippone, F. Camastra, F. Masulli, and S. Rovetta.
       \textbf{A survey of kernel and spectral methods for clustering}.
       \emph{Pattern Recognition}, 41(1):176-190, January 2008.
       %% \end{itemize}
%        Manuscripts published in volume 41 (year 2008) were judged by the Editor-in-Chief and the members of the Editorial and Advisory Boards of the journal based on the following criteria: originality of the contribution, presentation and exposition of the manuscript, and citations by other researchers.
\end{itemize}

\subsection*{Media Coverage}
\begin{itemize}
\item {\em MIT Technology Review website} - 20 October 2015 based on ``Monte Carlo strength evaluation: Fast and reliable password checking''
\item {\em New Scientist website} - 03 March 2012 based on ``Predicting the conflict level in television political debates: an approach based on crowdsourcing, nonverbal communication and Gaussian processes''
\end{itemize}


\subsection*{Keynote Presentations}
\begin{itemize}
\item 5 Dec 2013, Conference on Electronics, Telecommunications and Computers 2013, Lisbon, Portugal.
\item 9 Jun 2011, Italian Statistical Society Conference, Bologna, Italy. %- \emph{Bayesian inference in latent variable models and applications}.
\end{itemize}


\subsection*{Referee Activity}
\begin{itemize}
\item Funding bodies:
Leverhulme Trust (\pounds100K+)
%German-Israeli Foundation (\pounds30K+)

\item Journals:
IEEE Transactions on Pattern Analysis and Machine Intelligence, 
Journal of Machine Learning Research,
Bioinformatics, 
Signal Processing, 
Pattern Recognition,
Pattern Recognition Letters,
IEEE Transactions on Neural Networks,
IEEE Transactions on Signal Processing, 
IEEE Signal Processing Letters, 
Computational Statistics \& Data Analysis, 
Computational Intelligence,
Neural Processing Letters
%% Soft Computing.
\item Conferences: 
  NIPS (2014--2018),
  ICML (2015--2018),
  ECML (2016--2017),
  AISTATS (2012, 2013, 2016--2018), 
  IJCAI (2016),
  IJCNN (2006--2010, 2015),
  ICPRAM (2012--2015),
  ICANN (2014).
  %% PRIB (22013, 2010, 2009)
\end{itemize}


\subsection*{Selected Conference Presentations}
\begin{itemize}
\item 9 Jul 2015, ICML 2015, Lille, France
\item 26 Aug 2014, ICPR 2014, Stockholm, Sweden
\item 25 Sep 2013, ECML/PKDD 2013, Prague, Czech Republic
\item 30 May 2012, LGM2012, NTNU, Trondheim, Norway
\end{itemize}

\subsection*{Selected Invited Presentations}
\begin{itemize}
\item Imperial College (2018), Google Research NYC (2017), Yandex (2017), University of Oxford (2015), University of Sheffield (2015), Columbia University (2014, 2009), Bristol University (2014), University of Edinburgh (2014, 2009), UTIA Prague (2014), University of Turin (2014, 2012)
\end{itemize}


\subsection*{Teaching Activity}
\begin{itemize}
\item Spring 2016--2018 - Lecturer (42 h) 
  Advanced Statistical Inference (MSc) - EURECOM
\item Spring 2013--2015 - Lecturer (30 h) 
  Machine Learning (Year 4) - University of Glasgow
\item Fall 2012--2014 - Lecturer (30 h) 
  Artificial Intelligence (Year 4) - University of Glasgow
\item Fall 2014 - Lecturer (30 h) 
  Algorithmic Foundations (Year 2) - University of Glasgow
\end{itemize}

\subsection*{Post-Doc Supervision}
\begin{itemize}
\item Dr Sebastien Marmin: EURECOM. Fall 2017 - present
\item Dr Dimitrios Milios: EURECOM. Fall 2017 - present
\item Dr Mu Niu: School of Mathematics and Statistics, University of Glasgow. Fall 2014 - Fall 2017
\item Dr Roberto Visintainer: Fondazione Bruno Kessler, Trento. Visiting: Fall 2015 - Spring 2016
\end{itemize}

\subsection*{Ph.D. Supervision}
\begin{itemize}
\item Rosa Candela: Dept. of Data Science, EURECOM and Amadeus. Spring 2018 - Spring 2021
\item Simone Rossi: Dept. of Data Science, EURECOM. Spring 2018 - Spring 2021
\item Graziano Mita: Dept. of Data Science, EURECOM and SAP. Fall 2017 - Fall 2020
\item Gia-Lac Tran: Dept. of Data Science, EURECOM. Fall 2017 - Fall 2020
\item Kurt Cutajar: Dept. of Data Science, EURECOM. Fall 2015 - Spring 2019
\item Remi Domingues: Dept. of Data Science, EURECOM and Amadeus. Spring 2016 - Spring 2019
\item Xiaoyu Xiong: School of Computing Science, University of Glasgow. Fall 2013 - Spring 2017
\end{itemize}

\subsection*{Ph.D. Committee}
\begin{itemize}
\item Daniel Trejo Ba\~{n}os: School of Informatics, University of Edinburgh. Fall 2015
\item Anna Polychroniou: School of Computing Science, University of Glasgow. Spring 2014
\end{itemize}


