\documentclass[a4paper,10pt]{article}

\usepackage{url}
\usepackage{graphicx}
\usepackage{eurosym}

\usepackage[left=1.6cm,top=1.6cm,bottom=1.6cm,right=1.6cm]{geometry}

%% \addtolength{\topmargin}{-60pt}
%% \addtolength{\textheight}{60pt}
%% \addtolength{\oddsidemargin}{-70pt}
%% \addtolength{\textwidth}{140pt}

\usepackage{xspace}
\newcommand{\ml}{\name{ml}}
\newcommand{\name}[1]{{\textsc{#1}}\xspace}


\begin{document}

\begin{center}
{\bf \LARGE Curriculum Vitae}
\end{center}

%% \vspace{0.5cm}

\subsection*{Maurizio Filippone}
\begin{itemize}
%\item Nationality:        Italian
\item {\em AXA Chair of Computational Statistics \& Associate Professor} at EURECOM, Sophia Antipolis, France
\item Email:              \texttt{maurizio.filippone@eurecom.fr}
\item Web Page:           \url{http://www.eurecom.fr/~filippon}
\end{itemize}



\subsection*{Education}

I received a Master's degree in Physics and a Ph.D. in Computer Science from the University of Genoa, Italy, in 2004 and 2008, respectively.
The reason to pursue a Ph.D. in Computer Science was to deepen my interest in machine learning, which sparked towards the end of my studies in Physics.
I also hold the French abilitation ``Habilitation \'a diriger des recherches'', which is the highest degree in French academic system to allow someone to become a professor.

%% \begin{itemize}
%% \item Ph.D. in Computer Science - University of Genoa -- 2008 \\
%% Thesis title: Central Clustering in Kernel-Induced Spaces 
%% %% Keywords: kernel methods for clustering, spectral clustering, relational clustering.

%% \item Master's Degree in Physics % (full marks: 110/110) 
%% -- University of Genoa -- 2004
%% %% Thesis title: Ensemble methods for time series analysis and forecasting \\
%% %% Keywords: Nonlinear systems, regression, ensemble of learning machines, signal processing.
%% \end{itemize}
%% •	EDUCATION

%% 199? 	Ph.D.
%% 	Name of Faculty/ Department, Name of University/ Institution, Country
%% 	Name of Ph.D. Supervisor 
%% 199? 	Master
%% 	Name of Faculty/ Department, Name of University/ Institution, Country
	 

\subsection*{Academic Positions}
%% •	CURRENT POSITION(S)

%% 201? –  	Current Position
%% 	Name of Faculty/ Department, Name of University/ Institution/ Country
	
%% 200? – 	Current Position
%% 	Name of Faculty/ Department, Name of University/ Institution/ Country
                          
\begin{itemize}
\item Current position -- {\bf Associate Professor} -- % \\
  %% Department of Data Science, 
EURECOM, Sophia Antipolis, France % \\
  %% Keywords: Bayesian inference, Nonparametric modeling, Scalable inference

%% \vspace{3mm}

%% \item Current position -- {\em Associate Professor} -- % \\
%%   %% Department of Data Science, 
%% EURECOM, Sophia Antipolis, France % \\
%%   %% Keywords: Bayesian inference, Nonparametric modeling, Scalable inference

%% \vspace{3mm}

\item From Fall 2015 to Spring 2018 -- {\em Assistant Professor} -- % \\
  EURECOM, Sophia Antipolis, France % \\
  %% Keywords: Bayesian inference, Nonparametric modeling, Scalable inference

\item From Fall 2011 to Fall 2015 -- {\em Lecturer} -- % \\
  School of Computing Science -- University of Glasgow, UK % \\
  %% Keywords: Bayesian inference, Gaussian Processes, Markov chain Monte Carlo

\item From Fall 2010 to Fall 2011 -- {\em Research Associate} (PI: Prof. Mark Girolami) -- % \\
  Department of Statistical Science -- University College London (2011), UK and % \\
  School of Computing Science -- University of Glasgow (2010), UK % \\
%  Grant: The Synthesis of Probabilistic Prediction \& Mechanistic Modelling within a Computational \& Systems Biology Context (
  %% Keywords: Bayesian inference, Gaussian Processes, Markov chain Monte Carlo
  
\item From Spring 2008 to Fall 2009 -- {\em Research Associate} (PI: Dr Guido Sanguinetti) -- % \\
  Department of Computer Science -- University of Sheffield, UK % \\
%  Grant: ALMS: Advanced Lifestyle Monitoring Systems 
  %% Keywords: novelty detection, statistical testing, Bayesian inference

\item From Spring 2007 to Fall 2007 -- {\em Research Scholar} (PIs: Profs. D. Barbar\`a, C. Domeniconi) -- % \\
  Department of Information and Software Engineering -- George Mason University, VA, USA % \\
%  Grant: Detecting Suspicious Behavior in Reconnaissance Images 
  %% Keywords: outlier detection, density estimation, relational clustering
  
\end{itemize}


%% •	PREVIOUS POSITIONS

%% 200? – 200? 	Position held 
%% 	Name of Faculty/ Department, Name of University/ Institution/ Country
	
%% 200? – 200? 	Position held
%% 	Name of Faculty/ Department, Name of University/ Institution/ Country
	
\subsection*{Research contributions and impact}

My research interest is in the field of Bayesian \ml. 
In the last ten years, I've been focusing on nonparametric Bayesian models based on Gaussian processes, proposing a number of fundamental contributions to their applicability to large-scale problems.
Due to the advancements in the field, myself and other groups working in the domain have established ways to approximate Gaussian processes and Deep Gaussian processes as Bayesian Deep Neural Networks. 
As a result, my research is currently focusing on Bayesian Deep Learning, and techniques at the interface between Gaussian processes and Deep Neural Networks. 

After my Ph.D., I've published 100+ papers almost equally split between journals and conference, and I have been the leading author (either first or last) in most of these.
As of Jan 2022, I've received 3000+ citations and my $h$-index is 26 (source Google Scholar).

\subsection*{Research Grants}
\begin{itemize}
  \item PI: \emph{ECO-ML: Rethinking Modern Machine Learning Tools for a New Generation of Low-Power Large-Scale Modeling Systems} (300K\euro), 2018--2021, ANR-JCJC (French) research grant
  \item PI: AXA Chair of Computational Statistics: \emph{New Computational Approaches to Risk Modeling} (600K\euro), 2016--2023, AXA Research Fund 
  \item Co-PI: \emph{Computational inference of biopathway dynamics and structures} (\pounds 340K), 2014--2017, (PI) D.~Husmeier and (Co-PI) S.~Rogers - EPSRC (UK) research grant 
\end{itemize}


%% •	FELLOWSHIPS 
%% 200? – 200? 	Scholarship, Name of Faculty/ Department/Centre, Name of University/ Institution/ Country 
%% 199? – 199? 	Scholarship, Name of Faculty/ Department/Centre, Name of University/ Institution/ Country

%% •	SUPERVISION OF GRADUATE STUDENTS AND POSTDOCTORAL FELLOWS
%% \subsection*{Supervision of Postdoctoral Fellows and Graduate Students}

%% %% \paragraph{Post-Doc Supervision:}
%% I'm currently supervising two {\bf post-docs} at EURECOM, {\bf Giulio Franzese} and {\bf Dimitrios Milios}, who are partially funded by two ongoing grants on fundamentals of Bayesian \ml.
%% %% Previously, I hosted a visiting post-doc {\bf Roberto Visintainer} shortly after I moved to EURECOM in 2015.
%% %% Finally, 
%% Shortly after joining EURECOM, I supervised another post-doc {\bf Sebastien Marmin} on the same grants, and before joining EURECOM, I co-supervised {\bf Mu Niu} as a post-doc for three years at the University of Glasgow, funded by a grant from the UK research council EPSRC. 
%% %% Previously, I hosted a visiting post-doc {\bf Roberto Visintainer} shortly after I moved to EURECOM in 2015.
%% %% Finally, before joining EURECOM I co-supervised {\bf Mu Niu} as a post-doc for three years at the University of Glasgow, funded by a grant from the UK research council EPSRC. 
%% %% \begin{itemize}
%% %% \item Dr Sebastien Marmin: EURECOM. Fall 2017 - Fall 2020
%% %% \item Dr Dimitrios Milios: EURECOM. Fall 2017 - Fall 2020
%% %% \item Dr Mu Niu: School of Mathematics and Statistics, University of Glasgow. Fall 2014 - Fall 2017
%% %% \item Dr Roberto Visintainer: Fondazione Bruno Kessler, Trento. Visiting: Fall 2015 - Spring 2016
%% %% \end{itemize}

%% %% \paragraph{Ph.D. Supervision:}
%% I'm currently supervising five {\bf Ph.D. students} at EURECOM, {\bf Jonas Wacker} (ends in Spring 2022), {\bf Simone Rossi} (ends in Spring 2022), {\bf Bogdan Kozyrskiy} (ends in Spring 2023), and {\bf Ba-Hien Tran} (ends in Fall 2023) who are partially funded by two ongoing grants on fundamentals of Bayesian \ml.
%% Recently, three Ph.D. students under my supervision at EURECOM, {\bf Kurt Cutajar}, {\bf Gia-Lac Tran}, and {\bf Remi Domingues}, successfully defended their theses, the first two on fundamentals of Bayesian \ml, and the third on machine learning for fraud detection in collaboration with the company Amadeus in Sophia Antipolis, France. 
%% %% \paragraph*{Ph.D. Co-Supervision:}
%% Together with my colleague {\bf Pietro Michiardi}, I've co-supervised Ph.D. students funded by industry with Amadeus on time series ({\bf Rosa Candela}), with SAP on interpretable \ml ({\bf Graziano Mita}), and with Renault Software Labs on \ml for vehicular technologies ({\bf Ugo Lecerf} and {\bf Matthieu Da Silva Filarder}).
%% %% , I'm co-supervising {\bf Rosa Candela} on time-series with Amadeus, {\bf Graziano Mita} on interpretable \ml with SAP, and {\bf Ugo Lecerf} and {\bf Matthieu Da Silva Filarder} on machine learning for vehicular technologies with Renault Software Labs.
%% Prior to joining EURECOM, I supervised a self-funded Ph.D. student {\bf Xiaoyu Xiong} at the University of Glasgow.

%% %% \begin{itemize}
%% %% \item Jonas Wacker: Dept. of Data Science, EURECOM. Spring 2019 - Spring 2022
%% %% \item Simone Rossi: Dept. of Data Science, EURECOM. Spring 2018 - Spring 2021
%% %% \item Gia-Lac Tran: Dept. of Data Science, EURECOM. Fall 2017 - Fall 2020
%% %% \item Kurt Cutajar: Dept. of Data Science, EURECOM. Fall 2015 - Spring 2019
%% %% \item Remi Domingues: Dept. of Data Science, EURECOM and Amadeus. Spring 2016 - Spring 2019
%% %% \item Xiaoyu Xiong: School of Computing Science, University of Glasgow. Fall 2013 - Spring 2017
%% %% \end{itemize}

%% %% \subsubsection*{Ph.D. Co-Supervision}
%% %% \begin{itemize}
%% %% \item Rosa Candela: Dept. of Data Science, EURECOM and Amadeus. Spring 2018 - Spring 2021
%% %% \item Graziano Mita: Dept. of Data Science, EURECOM and SAP. Fall 2017 - Fall 2020
%% %% \end{itemize}

%% %% 200? – 200? 	Number of Postdocs/ Ph.D./ Master Students
%% %% Name of Faculty/ Department/ Centre, Name of University/ Institution/ Country


\subsection*{Teaching Activities}
%% •	TEACHING ACTIVITIES (if applicable) 

I have started teaching when I joined the University of Glasgow as a lecturer in 2011, where I taught under-graduate and post-graduate courses in {\bf Algorithmic Foundations} and {\bf Machine Learning}.
In Glasgow, I also designed and created the material of a new course on {\bf Artificial Intelligence}. 
After joining EURECOM, I have given lectures on Bayesian \ml in a course named {\bf Advanced Statistical Inference}.
Between 2018 and 2019, I have delivered lectures at the MLCC {\bf summer school} in Genoa, Italy, and I designed and created the material for a {\bf tutorial} on Gaussian processes at the IJCNN 2019 conference in collaboration with E.~V.~Bonilla, and another one on Bayesian Deep Learning at the IJCAI 2021 conference in collaboration with my Ph.D. student S.~Rossi. 

%% \begin{itemize}
%% \item Spring 2016--2018 - Lecturer (42 h) 
%%   Advanced Statistical Inference (MSc) - EURECOM
%% \item Spring 2013--2015 - Lecturer (30 h) 
%%   Machine Learning (Year 4) - University of Glasgow
%% \item Fall 2012--2014 - Lecturer (30 h) 
%%   Artificial Intelligence (Year 4) - University of Glasgow
%% \item Fall 2014 - Lecturer (30 h) 
%%   Algorithmic Foundations (Year 2) - University of Glasgow
%% \end{itemize}


%% 200? – 	Teaching position – Topic, Name of University/ Institution/ Country
%% 200? – 200? 	Teaching position – Topic, Name of University/ Institution/ Country



\subsection*{Service to the Scientific Community}

I've served as a {\bf Program committee} member for several conference. 
Here is a selection including the most prestigious ones:
  NeurIPS (2014--2019),
  ICML (2015--2020),
  ECML (2016--2017),
  AISTATS (2012--2013, 2016--2021), 
  IJCAI (2016),
  IJCNN (2006--2010, 2015).
I'm currently covering more senior roles, such as {\bf Area Chair} for AISTATS, and {\bf Guest Editor} for the ECML/PKDD Machine Learning Journal track.
Between 2013 and 2016 I've served as an {\bf Associate Editor} for the journals Pattern Recognition and the IEEE Transactions on Neural Networks and Learning Systems.

%% \begin{itemize}
%% \item {\em Area Chair} for AISTATS 2020
%% \item {\em Guest Editor} of ECML/PKDD journal track, Machine Learning Journal, 2020
%% \item {\em Program committee} member of
%%   NuerIPS (2014--2019),
%%   ICML (2015--2019),
%%   ECML (2016--2017),
%%   AISTATS (2012, 2013, 2016--2019), 
%%   IJCAI (2016),
%%   IJCNN (2006--2010, 2015),
%% \item {\em Associate Editor}: Pattern Recognition (end 2012 - end 2016)
%% \item {\em Associate Editor}: IEEE Transactions on Neural Networks and Learning Systems (2013 - end 2016)
%% \item {\em Technical Program Chair} for IJCNN 2014
%% \end{itemize}

%% •	ORGANISATION OF SCIENTIFIC MEETINGS (if applicable)
%% \subsection*{Organisation of Scientific Meetings}

%% 201?	Please specify your role and the name of event / Country 
%% 200? 	Please specify type of event / number of participants / Country


%% •	INSTITUTIONAL RESPONSIBILITIES (if applicable)

%% 201? – 	Faculty member, Name of University/ Institution/ Country
%% 201? – 201? 	Graduate Student Advisor, Name of University/ Institution/ Country
%% 200? – 200? 	Member of the Faculty Committee, Name of University/ Institution/ Country 
%% 200? – 200? 	Organiser of the Internal Seminar, Name of University/ Institution/ Country
%% 200? – 200? 	Member of a Committee; role, Name of University/ Institution/ Country


%% •	COMMISSIONS OF TRUST (if applicable)

%% 201? – 	Scientific Advisory Board, Name of University/ Institution/ Country
%% 201? – 	Review Board, Name of University/ Institution/ Country
%% 201? –	Review panel member, Name of University/ Institution/ Country
%% 201? – 	Editorial Board, Name of University/ Institution/ Country
%% 200? – 	Scientific Advisory Board, Name of University/ Institution/ Country
%% 200? –	Reviewer, Name of University/ Institution/ Country 
%% 200? –	Scientific Evaluation, Name of University/ Institution/ Country
%% 200? –	Evaluator, Name of University/ Institution/ Country


%% •	MEMBERSHIPS OF SCIENTIFIC SOCIETIES (if applicable)

%% 201? –	Member, Research Network “Name of Research Network”
%% 200? –	Associated Member, Name of Faculty/ Department/Centre, Name of University/ Institution/ Country
%% 200? –	Funding Member, Name of Faculty/ Department/Centre, Name of University/ Institution/ Country 

\subsection*{Selected Presentations}
%% \begin{itemize}
%% \item Aug 2018 and Aug 2019 - Deep Bayes Summer School, Moscow, Russia.
%% \item June 2019 - Machine Learning Crash Course MLCC 2019, Genova, Italy

I receive regular invitations to deliver {\bf keynote} presentations at international events. 
Recently, I presented at the Northern Lights Deep Learning Workshop in Troms\o, Norway, and at the Workshop on Surrogate models for UQ in complex systems, Cambridge, UK.
%% \item 5 Dec 2013, Conference on Electronics, Telecommunications and Computers 2013, Lisbon, Portugal.
%% \item 9 Jun 2011, Italian Statistical Society Conference, Bologna, Italy. %- \emph{Bayesian inference in latent variable models and applications}.
%% \end{itemize}

I've also been actively promoting my research through {\bf invited talks}. 
Here is a selected list over the past five years:
University of Oxford (2019, 2015), Imperial College (2018), Google Research NYC (2017), Yandex Moscow (2017), University of Sheffield (2015), Columbia University (2014), Bristol University (2014), University of Edinburgh (2014), UTIA Prague (2014), University of Turin (2014, 2012).

In complement to these, I gave a talk at the Deep Bayes summer school in Moscow, Russia (2018, 2019), at the MLCC summer school in Genoa, Italy (2019),
%% In 2019 I presented a tutorial entitled \href{https://ebonilla.github.io/gaussianprocesses/}{``Modern Gaussian Processes: Scalable Inference and Novel Applications''} 
and I delivered a tutorial on Gaussian processes at the IJCNN 2019 conference and a tutorial on Bayesian Deep Learning at the IJCAI 2021 conference. 



%% \begin{itemize}
%% \item Aug 2018 and Aug 2019 - Deep Bayes Summer School, Moscow, Russia.
%% \item June 2019 - Machine Learning Crash Course MLCC 2019, Genova, Italy
%% \item Jan 2019 - Northern Lights Deep Learning Workshop in Troms\o, Norway.
%% \item Feb 2018 - Workshop on Surrogate models for UQ in complex systems, Cambridge, UK.
%% \item 5 Dec 2013, Conference on Electronics, Telecommunications and Computers 2013, Lisbon, Portugal.
%% \item 9 Jun 2011, Italian Statistical Society Conference, Bologna, Italy. %- \emph{Bayesian inference in latent variable models and applications}.
%% \end{itemize}

%% \subsection*{Selected Invited Presentations}
%% \begin{itemize}
%% \item University of Oxford (2019, 2015), Imperial College (2018), Google Research NYC (2017), Yandex Moscow (2017), University of Sheffield (2015), Columbia University (2014, 2009), Bristol University (2014), University of Edinburgh (2014, 2009), UTIA Prague (2014), University of Turin (2014, 2012)
%% \end{itemize}


\subsection*{Media Coverage}

%% My research has been the subject of some media articles in the past.
%% In particular, in July 2019, the article ``Light, a possible solution for a sustainable AI'' featured in {\em The Conversation}, and it is about the topics put forward by \myacro.  
%% Previously, in October 2015 my paper on ``Monte Carlo strength evaluation: Fast and reliable password checking'' has been discussed in an article in the {\em MIT Technology Review website}, 
%% and in March 2012 another one on ``Predicting the conflict level in television political debates: an approach based on crowdsourcing, nonverbal communication and Gaussian processes'' in the {\em New Scientist website}.


\begin{itemize}
\item {\em The Conversation} - 26 July 2019 - ``Light, a possible solution for a sustainable AI''
\item {\em MIT Technology Review website} - 20 October 2015 based on ``Monte Carlo strength evaluation: Fast and reliable password checking''
\item {\em New Scientist website} - 03 March 2012 based on ``Predicting the conflict level in television political debates: an approach based on crowdsourcing, nonverbal communication and Gaussian processes''
\end{itemize}

\subsection*{Major Collaborations}
%% •	MAJOR COLLABORATIONS (if applicable)

I have a number of international collaborations, which developed out of shared scientific interests with {\bf John P. Cunningham} (Department of Statistics, Columbia University), {\bf Lorenzo A. Rosasco}, (University of Genoa and MIT), {\bf Edwin V. Bonilla} (Data61, Sydney, Australia), {\bf Michael A. Osborne} (University of Oxford, UK), and {\bf Markus Heinonen} (Aalto University, Helsinki, Finland).
I'm also named collaborator in two grants to develop the application of Bayesian \ml to neuroscience and spatial statistics. 
In particular, the Wellcome trust grant ``BRAINCHART: Normative brain charting for predicting and stratifying psychosis'' with PI {\bf Andre Marquand} (Donders Institute, Nijmegen, The Netherlands), and on the Australian Research Council Discovery Early Career Researcher Award (DECRA) grant with PI {\bf Andrew Zammit-Mangion} (University of Wollongong, Australia).

%% \begin{itemize}
%% \item 2019-2021 - Named Collaborator in Wellcome trust grant: ``BRAINCHART: Normative brain charting for predicting and stratifying psychosis'' - PI: Andre Marquand, Donders Institute, Nijmegen, Netherlands
%% \item 2017-2021 - Named Collaborator in Australian Research Council Discovery Early Career Researcher Award (DECRA) grant - PI: Andrew Zammit-Mangion, University of Wollongong, Australia.
%% \item John Patrick Cunningham, Department of Statistics, Columbia University
%% \item Lorenzo A. Rosasco, University of Genoa and MIT
%% \item Edwin V. Bonilla, Data61, Sydney, Australia
%% \item Michael Osborne, University of Oxford, UK
%% \item James Hensman, Prowler.io, Cambridge, UK
%% \end{itemize}

%% Name of collaborators, Topic, Name of Faculty/ Department/Centre, Name of University/ Institution/ Country

\subsection*{Awards}
%% \begin{itemize}
%%      \item 
International Association of Pattern Recognition best paper award: 
       %% \begin{itemize}
       %% \item
       M. Filippone, et al. % F. Camastra, F. Masulli, and S. Rovetta.
       \textbf{A survey of kernel and spectral methods for clustering}.
       \emph{Pattern Recognition}, 41(1):176-190, January 2008.
       %% \end{itemize}
%        Manuscripts published in volume 41 (year 2008) were judged by the Editor-in-Chief and the members of the Editorial and Advisory Boards of the journal based on the following criteria: originality of the contribution, presentation and exposition of the manuscript, and citations by other researchers.
%% \end{itemize}
%
I also received a ``Special Mention'' award for a poster at the Autumn meeting on Latent Gaussian Models in Trondheim, Norway in 2015. 


\subsection*{Selected Publications}
%\subsubsection*{Journals}

\begin{itemize}
\item  A. Zammit-Mangion, T.-L. J. Ng, Q. Vu, and M. Filippone. Deep compositional spatial models. \emph{Journal of the American Statistical Association}, to appear, 2021.
\item  B.-H. Tran, S. Rossi, D. Milios, and M. Filippone. Model Selection for Bayesian Autoencoders. In \emph{Advances in Neural Information Processing Systems 34: NeurIPS 2021}, 2021.  
\item  G.-L. Tran, D. Milios, P. Michiardi, and M. Filippone. Sparse within Sparse Gaussian Processes using Neighbor Information. In \emph{Proceedings of the 38th International Conference on Machine Learning}, volume 139 of \emph{Proceedings of Machine Learning Research}, pages 10369--10378. PMLR, 18--24 Jul 2021.
\item  S. Rossi, S. Marmin, and M. Filippone. Walsh-Hadamard Variational Inference for Bayesian Deep Learning. In \emph{Advances in Neural Information Processing Systems 33: NeurIPS 2020}, 2020.  
\item  C. Nemeth, F. Lindsten, M. Filippone, and J. Hensman. Pseudo-extended Markov chain Monte Carlo. In \emph{Advances in Neural Information Processing Systems 32: NeurIPS 2019, 9-12 December 2019, Vancouver, British Columbia, Canada}, 2019.  
\item  S. Rossi, P. Michiardi, and M. Filippone. Good Initializations of Variational Bayes for Deep Models. In \emph{Proceedings of the 36th International Conference on Machine Learning, ICML 2019, Long Beach, USA, 2019}, 2019.  
\item  G.-L. Tran, E. V. Bonilla, J. P. Cunningham, P. Michiardi, and M. Filippone. Calibrating Deep Convolutional Gaussian Processes. In \emph{AISTATS 2019, Naha, Japan, 2019}, 2019.  
\item  D. Milios, R. Camoriano, P. Michiardi, L. Rosasco, and M. Filippone. Dirichlet-based Gaussian Processes for Large-scale Calibrated Classification. In \emph{Advances in Neural Information Processing Systems 31: NeurIPS 2018, December 3-7 2018, Montreal, Quebec, Canada}, 2018.  
\item  M. Lorenzi and M. Filippone. Constraining the Dynamics of Deep Probabilistic Models. In \emph{Proceedings of the 35th International Conference on Machine Learning, ICML 2018, Stockholm, Sweden, 2018}, 2018.  
\item  K. Cutajar, E. V. Bonilla, P. Michiardi, and M. Filippone. Random feature expansions for deep Gaussian processes. In \emph{Proceedings of the 34th International Conference on Machine Learning, ICML 2017, Sydney, Australia, August 6-11, 2017}, 2017.  
\item  K. Cutajar, M. A. Osborne, J. P. Cunningham, and M. Filippone. Preconditioning kernel matrices. In \emph{Proceedings of the 33rd International Conference on Machine Learning, ICML 2016, New York City, USA, June 19-24, 2016}, 2016.  
\item  J. Hensman, A. G. de G. Matthews, M. Filippone, and Z. Ghahramani. MCMC for variationally sparse Gaussian processes. In \emph{Advances in Neural Information Processing Systems 28: NeurIPS 2015, December 7-12 2015, Montreal, Quebec, Canada}, 2015.  
\item  M. Filippone and R. Engler. Enabling scalable stochastic gradient-based inference for Gaussian processes by employing the Unbiased LInear System SolvEr (ULISSE). In \emph{Proceedings of the 32nd International Conference on Machine Learning, ICML 2015, Lille, France, July 6-11, 2015}, 2015.  
\item  M. Filippone and M. Girolami. Pseudo-marginal Bayesian inference for Gaussian processes. \emph{IEEE Transactions on Pattern Analysis and Machine Intelligence}, 36(11):2214-2226, 2014.  
\item  F. Dondelinger, M. Filippone, S. Rogers, and D. Husmeier. ODE parameter inference using adaptive gradient matching with Gaussian processes. In \emph{AISTATS}, 2013.  
\item  M. Filippone, M. Zhong, and M. Girolami. A comparative evaluation of stochastic-based inference methods for Gaussian process models. \emph{Machine Learning}, 93(1):93-114, 2013.  
\item  M. Filippone, A. F. Marquand, C. R. V. Blain, S. C. R. Williams, J. Mour\~ao-Miranda, and M. Girolami. Probabilistic prediction of neurological disorders with a statistical assessment of neuroimaging data modalities. \emph{Annals of Applied Statistics}, 6(4):1883-1905, 2012.  
%% %% \item  M. Filippone and G. Sanguinetti. Approximate inference of the bandwidth in multivariate kernel density estimation. \emph{Computational Statistics \& Data Analysis}, 55(12):3104-3122, 2011.  
%% \item  M. Filippone, F. Masulli, and S. Rovetta. Applying the possibilistic c-means algorithm in kernel-induced spaces. \emph{IEEE Transactions on Fuzzy Systems}, 18(3):572-584, June 2010.  
\item  M. Filippone and G. Sanguinetti. Information theoretic novelty detection. \emph{Pattern Recognition}, 43(3):805-814, March 2010.  
%% %% \item  M. Filippone. Dealing with non-metric dissimilarities in fuzzy central clustering algorithms. \emph{International Journal of Approximate Reasoning}, 50(2):363-384, February 2009.  
%% \item  M. Filippone, F. Camastra, F. Masulli, and S. Rovetta. A survey of kernel and spectral methods for clustering. \emph{Pattern Recognition}, 41(1):176-190, January 2008.  
\end{itemize}


%% •	CAREER BREAKS (if applicable)

%% Exact dates	Please indicate the reason and the duration in months.

%% ************************************************** OLD SHORT CV ***************************************************


\end{document}

\subsection*{Education}
\begin{itemize}
\item 2008 - Ph.D. in Computer Science - University of Genoa \\
Thesis title: Central Clustering in Kernel-Induced Spaces \\
Keywords: kernel methods for clustering, spectral clustering, relational clustering.

\item 2004 - Master's Degree in Physics (full marks: 110/110) - University of Genoa \\
Thesis title: Ensemble methods for time series analysis and forecasting \\
Keywords: Nonlinear systems, regression, ensemble of learning machines, signal processing.
\end{itemize}

\subsection*{Research Experience}
\begin{itemize}
\item From Spring 2018 to present - {\em Associate Professor} \\
  EURECOM, Sophia Antipolis, France \\
  Keywords: Bayesian inference, Nonparametric modeling, Scalable inference

\item From Fall 2015 to Spring 2018 - {\em Ma\^{i}tre de Conf\'{e}rence} \\
  EURECOM, Sophia Antipolis, France \\
  Keywords: Bayesian inference, Nonparametric modeling, Scalable inference

\item From Fall 2011 to Fall 2015 - {\em Lecturer} \\
  School of Computing Science - University of Glasgow \\
  Keywords: Bayesian inference, Gaussian Processes, Markov chain Monte Carlo

\item From Fall 2010 to Fall 2011 - {\em Research Associate} (PI: Prof. Mark Girolami) \\
  Department of Statistical Science - University College London (2011) \\
  School of Computing Science - University of Glasgow (2010) \\
%  Grant: The Synthesis of Probabilistic Prediction \& Mechanistic Modelling within a Computational \& Systems Biology Context (
  Keywords: Bayesian inference, Gaussian Processes, Markov chain Monte Carlo
  
\item From Spring 2008 to Fall 2009 - {\em Research Associate} (PI: Dr G. Sanguinetti) \\
  Department of Computer Science - University of Sheffield \\
%  Grant: ALMS: Advanced Lifestyle Monitoring Systems 
  Keywords: novelty detection, statistical testing, Bayesian inference

\item From Spring 2007 to Fall 2007 - {\em Research Scholar} (PIs: Profs. D. Barbar\`a, C. Domeniconi) \\
  Department of Information and Software Engineering - George Mason University \\
%  Grant: Detecting Suspicious Behavior in Reconnaissance Images 
  Keywords: outlier detection, density estimation, relational clustering
  
\end{itemize}

\subsection*{Professional Activities}
\begin{itemize}
\item {\em Associate Editor} for Pattern Recognition (end 2012 - end 2016)
\item {\em Associate Editor} for the IEEE Transactions on Neural Networks and Learning Systems (2013 - end 2016)
\item {\em Technical Program Chair} for IJCNN 2014
\end{itemize}

\subsection*{Research Grants}
\begin{itemize}
  \item \emph{ECO-ML: Rethinking Modern Machine Learning Tools for a New Generation of Low-Power Large-Scale Modeling Systems} (300K\euro), 2018--2021, ANR-JCJC 
  \item AXA Chair of Computational Statistics: \emph{New Computational Approaches to Risk Modeling} (600K\euro), 2016--2023, AXA Research Fund 
  \item Co-PI: \emph{Computational inference of biopathway dynamics and structures} (\pounds 340K), 2014--2017, (PI) D.~Husmeier and (Co-PI) S.~Rogers - EPSRC (UK) research grant 
\end{itemize}

\subsection*{Contracts with Industry}
\begin{itemize}
  \item SAP, CIFRE PhD scholarship (from Fall 2017)
  \item Amadeus, CIFRE PhD scholarship (from Spring 2016) + PhD scholarship (from Spring 2018)
  \item Huawei, 6-months Internship (Spring 2018)
\end{itemize}

\subsection*{Selected Publications}
%\subsubsection*{Journals}

\begin{itemize}
\item  D. Milios, R. Camoriano, P. Michiardi, L. Rosasco, and M. Filippone. Dirichlet-based Gaussian Processes for Large-scale Calibrated Classification. In \emph{Advances in Neural Information Processing Systems 31: Annual Conference on Neural Information Processing Systems 2018, December 3-7 2018, Montreal, Quebec, Canada}, 2018.  
\item  M. Lorenzi and M. Filippone. Constraining the Dynamics of Deep Probabilistic Models. In \emph{Proceedings of the 35th International Conference on Machine Learning, ICML 2018, Stockholm, Sweden, 2018}, 2018.  
\item  K. Cutajar, E. V. Bonilla, P. Michiardi, and M. Filippone. Random feature expansions for deep Gaussian processes. In \emph{Proceedings of the 34th International Conference on Machine Learning, ICML 2017, Sydney, Australia, August 6-11, 2017}, 2017.  
\item  K. Cutajar, M. A. Osborne, J. P. Cunningham, and M. Filippone. Preconditioning kernel matrices. In \emph{Proceedings of the 33rd International Conference on Machine Learning, ICML 2016, New York City, USA, June 19-24, 2016}, 2016.  
\item  J. Hensman, A. G. de G. Matthews, M. Filippone, and Z. Ghahramani. MCMC for variationally sparse Gaussian processes. In \emph{Advances in Neural Information Processing Systems 28: Annual Conference on Neural Information Processing Systems 2015, December 7-12 2015, Montreal, Quebec, Canada}, 2015.  
\item  M. Filippone and R. Engler. Enabling scalable stochastic gradient-based inference for Gaussian processes by employing the Unbiased LInear System SolvEr (ULISSE). In \emph{Proceedings of the 32nd International Conference on Machine Learning, ICML 2015, Lille, France, July 6-11, 2015}, 2015.  
\item  M. Filippone and M. Girolami. Pseudo-marginal Bayesian inference for Gaussian processes. \emph{IEEE Transactions on Pattern Analysis and Machine Intelligence}, 36(11):2214-2226, 2014.  
\item  F. Dondelinger, M. Filippone, S. Rogers, and D. Husmeier. ODE parameter inference using adaptive gradient matching with Gaussian processes. In \emph{AISTATS}, 2013.  
\item  M. Filippone, M. Zhong, and M. Girolami. A comparative evaluation of stochastic-based inference methods for Gaussian process models. \emph{Machine Learning}, 93(1):93-114, 2013.  
\item  M. Filippone, A. F. Marquand, C. R. V. Blain, S. C. R. Williams, J. Mour\~ao-Miranda, and M. Girolami. Probabilistic prediction of neurological disorders with a statistical assessment of neuroimaging data modalities. \emph{Annals of Applied Statistics}, 6(4):1883-1905, 2012.  
%% %% \item  M. Filippone and G. Sanguinetti. Approximate inference of the bandwidth in multivariate kernel density estimation. \emph{Computational Statistics \& Data Analysis}, 55(12):3104-3122, 2011.  
%% \item  M. Filippone, F. Masulli, and S. Rovetta. Applying the possibilistic c-means algorithm in kernel-induced spaces. \emph{IEEE Transactions on Fuzzy Systems}, 18(3):572-584, June 2010.  
\item  M. Filippone and G. Sanguinetti. Information theoretic novelty detection. \emph{Pattern Recognition}, 43(3):805-814, March 2010.  
%% %% \item  M. Filippone. Dealing with non-metric dissimilarities in fuzzy central clustering algorithms. \emph{International Journal of Approximate Reasoning}, 50(2):363-384, February 2009.  
%% \item  M. Filippone, F. Camastra, F. Masulli, and S. Rovetta. A survey of kernel and spectral methods for clustering. \emph{Pattern Recognition}, 41(1):176-190, January 2008.  
\end{itemize}

%% \begin{itemize}\item  M. Filippone and M. Girolami. Pseudo-marginal Bayesian inference for Gaussian processes. \emph{IEEE Transactions on Pattern Analysis and Machine Intelligence}, 36(11):2214--2226, 2014.  
\item  M. Filippone, M. Zhong, and M. Girolami. A comparative evaluation of stochastic-based inference methods for Gaussian process models. \emph{Machine Learning}, 93(1):93--114, 2013.  
\item  M. Filippone, A. F. Marquand, C. R. V. Blain, S. C. R. Williams, J. Mour\~ao-Miranda, and M. Girolami. Probabilistic prediction of neurological disorders with a statistical assessment of neuroimaging data modalities. \emph{Annals of Applied Statistics}, 6(4):1883--1905, 2012.  
\item  M. Filippone and G. Sanguinetti. Approximate inference of the bandwidth in multivariate kernel density estimation. \emph{Computational Statistics \& Data Analysis}, 55(12):3104--3122, 2011.  
\item  M. Filippone and G. Sanguinetti. A perturbative approach to novelty detection in autoregressive models. \emph{IEEE Transactions on Signal Processing}, 59(3):1027--1036, 2011.  
\item  M. Filippone, F. Masulli, and S. Rovetta. Applying the possibilistic c-means algorithm in kernel-induced spaces. \emph{IEEE Transactions on Fuzzy Systems}, 18(3):572--584, June 2010.  
\item  M. Filippone and G. Sanguinetti. Information theoretic novelty detection. \emph{Pattern Recognition}, 43(3):805--814, March 2010.  
\item  M. Filippone. Dealing with non-metric dissimilarities in fuzzy central clustering algorithms. \emph{International Journal of Approximate Reasoning}, 50(2):363--384, February 2009.  
\item  M. Filippone, F. Camastra, F. Masulli, and S. Rovetta. A survey of kernel and spectral methods for clustering. \emph{Pattern Recognition}, 41(1):176--190, January 2008.  
\item  K. Cutajar, E. V. Bonilla, P. Michiardi, and M. Filippone. Random feature expansions for deep Gaussian processes. In \emph{Proceedings of the 34th International Conference on Machine Learning, ICML 2017, Sydney, Australia, August 6-11, 2017}, 2017.  
\item  K. Cutajar, M. A. Osborne, J. P. Cunningham, and M. Filippone. Preconditioning kernel matrices. In \emph{Proceedings of the 33rd International Conference on Machine Learning, ICML 2016, New York City, USA, June 19-24, 2016}, 2016.  
\item  J. Hensman, A. G. de G. Matthews, M. Filippone, and Z. Ghahramani. MCMC for variationally sparse Gaussian processes. In \emph{Advances in Neural Information Processing Systems 28: Annual Conference on Neural Information Processing Systems 2015, December 7-12 2015, Montreal, Quebec, Canada}, 2015.  
\item  M. Filippone and R. Engler. Enabling scalable stochastic gradient-based inference for Gaussian processes by employing the Unbiased LInear System SolvEr (ULISSE). In \emph{Proceedings of the 32nd International Conference on Machine Learning, ICML 2015, Lille, France, July 6-11, 2015}, 2015.  
\item  F. Dondelinger, M. Filippone, S. Rogers, and D. Husmeier. ODE parameter inference using adaptive gradient matching with Gaussian processes. In \emph{AISTATS}, 2013.  

\end{itemize}


\subsection*{Awards}
\begin{itemize}
     \item International Association of Pattern Recognition best paper award: 
       %% \begin{itemize}
       %% \item

       M. Filippone, F. Camastra, F. Masulli, and S. Rovetta.
       \textbf{A survey of kernel and spectral methods for clustering}.
       \emph{Pattern Recognition}, 41(1):176-190, January 2008.
       %% \end{itemize}
%        Manuscripts published in volume 41 (year 2008) were judged by the Editor-in-Chief and the members of the Editorial and Advisory Boards of the journal based on the following criteria: originality of the contribution, presentation and exposition of the manuscript, and citations by other researchers.
\end{itemize}

\subsection*{Media Coverage}
\begin{itemize}
\item {\em MIT Technology Review website} - 20 October 2015 based on ``Monte Carlo strength evaluation: Fast and reliable password checking''
\item {\em New Scientist website} - 03 March 2012 based on ``Predicting the conflict level in television political debates: an approach based on crowdsourcing, nonverbal communication and Gaussian processes''
\end{itemize}


\subsection*{Keynote Presentations}
\begin{itemize}
\item 5 Dec 2013, Conference on Electronics, Telecommunications and Computers 2013, Lisbon, Portugal.
\item 9 Jun 2011, Italian Statistical Society Conference, Bologna, Italy. %- \emph{Bayesian inference in latent variable models and applications}.
\end{itemize}


\subsection*{Referee Activity}
\begin{itemize}
\item Funding bodies:
Leverhulme Trust (\pounds100K+)
%German-Israeli Foundation (\pounds30K+)

\item Journals:
IEEE Transactions on Pattern Analysis and Machine Intelligence, 
Journal of Machine Learning Research,
Bioinformatics, 
Signal Processing, 
Pattern Recognition,
Pattern Recognition Letters,
IEEE Transactions on Neural Networks,
IEEE Transactions on Signal Processing, 
IEEE Signal Processing Letters, 
Computational Statistics \& Data Analysis, 
Computational Intelligence,
Neural Processing Letters
%% Soft Computing.
\item Conferences: 
  NIPS (2014--2018),
  ICML (2015--2018),
  ECML (2016--2017),
  AISTATS (2012, 2013, 2016--2018), 
  IJCAI (2016),
  IJCNN (2006--2010, 2015),
  ICPRAM (2012--2015),
  ICANN (2014).
  %% PRIB (22013, 2010, 2009)
\end{itemize}


\subsection*{Selected Conference Presentations}
\begin{itemize}
\item 9 Jul 2015, ICML 2015, Lille, France
\item 26 Aug 2014, ICPR 2014, Stockholm, Sweden
\item 25 Sep 2013, ECML/PKDD 2013, Prague, Czech Republic
\item 30 May 2012, LGM2012, NTNU, Trondheim, Norway
\end{itemize}

\subsection*{Selected Invited Presentations}
\begin{itemize}
\item Imperial College (2018), Google Research NYC (2017), Yandex (2017), University of Oxford (2015), University of Sheffield (2015), Columbia University (2014, 2009), Bristol University (2014), University of Edinburgh (2014, 2009), UTIA Prague (2014), University of Turin (2014, 2012)
\end{itemize}


\subsection*{Teaching Activity}
\begin{itemize}
\item Spring 2016--2018 - Lecturer (42 h) 
  Advanced Statistical Inference (MSc) - EURECOM
\item Spring 2013--2015 - Lecturer (30 h) 
  Machine Learning (Year 4) - University of Glasgow
\item Fall 2012--2014 - Lecturer (30 h) 
  Artificial Intelligence (Year 4) - University of Glasgow
\item Fall 2014 - Lecturer (30 h) 
  Algorithmic Foundations (Year 2) - University of Glasgow
\end{itemize}

\subsection*{Post-Doc Supervision}
\begin{itemize}
\item Dr Sebastien Marmin: EURECOM. Fall 2017 - present
\item Dr Dimitrios Milios: EURECOM. Fall 2017 - present
\item Dr Mu Niu: School of Mathematics and Statistics, University of Glasgow. Fall 2014 - Fall 2017
\item Dr Roberto Visintainer: Fondazione Bruno Kessler, Trento. Visiting: Fall 2015 - Spring 2016
\end{itemize}

\subsection*{Ph.D. Supervision}
\begin{itemize}
\item Rosa Candela: Dept. of Data Science, EURECOM and Amadeus. Spring 2018 - Spring 2021
\item Simone Rossi: Dept. of Data Science, EURECOM. Spring 2018 - Spring 2021
\item Graziano Mita: Dept. of Data Science, EURECOM and SAP. Fall 2017 - Fall 2020
\item Gia-Lac Tran: Dept. of Data Science, EURECOM. Fall 2017 - Fall 2020
\item Kurt Cutajar: Dept. of Data Science, EURECOM. Fall 2015 - Spring 2019
\item Remi Domingues: Dept. of Data Science, EURECOM and Amadeus. Spring 2016 - Spring 2019
\item Xiaoyu Xiong: School of Computing Science, University of Glasgow. Fall 2013 - Spring 2017
\end{itemize}

\subsection*{Ph.D. Committee}
\begin{itemize}
\item Daniel Trejo Ba\~{n}os: School of Informatics, University of Edinburgh. Fall 2015
\item Anna Polychroniou: School of Computing Science, University of Glasgow. Spring 2014
\end{itemize}


